
\prob{0022}{一元三次方程}

推导标准一元三次方程$ax^3 + bx^2 + cx + d = 0\quad(a \ne0)$的求根公式。
\problabels{yellow/代数, green/方程相关问题}

\ans{参见解法末尾。}

\subsection{换元法}

\subsubsection{消去二次项}

为了消去二次项,令$x = y + s$,$y$为未知数,则
\begin{align*}
  a(y + s)^3 + b(y + s)^2 + c(y + s) + d &= 0 \\
  ay^3 + (3as + b)y^2 + (3as^2 + 2bs + c)y & \\
  + (as^3 + bs^2 + cs + d) &= 0
\end{align*}
若要消去二次项,则必须使$3as + b = 0$,此时$s = -b/(3a)$。因此令$s = -b/(3a)$,于是$y = x - b/(3a)$,则
\begin{align*}
  ay^3 + \left(-\frac{b^2}{3a} + c\right)y + \left(\frac{2b^3}{27a^2} - \frac{bc}{3a} + d\right) &= 0 \\
  y^3 + \left(-\frac{b^2}{3a^2} + \frac ca\right)y + \left(\frac{2b^3}{27a^3} - \frac{bc}{3a^2} + \frac da\right) &= 0
\end{align*}
令
\begin{align*}
  p &= -\frac{b^2}{3a^2} + \frac ca \\
  q &= \frac{2b^3}{27a^3} - \frac{bc}{3a^2} + \frac da
\end{align*}
则可得不含二次项的一元三次方程
\[ y^3 + py + q = 0 \]

\subsubsection{消去一次项}

为了消去一次项,令$y = z + t/z$,$z$为未知数,则
\begin{align*}
  \left(z + \frac tz\right)^3 + p\left(z + \frac tz\right) + q &= 0 \\
  z^3 + (3t + p)z + q + \frac{3t^2 + pt}z + \frac{t^3}{z^3} &= 0
\end{align*}
此时,为了消去一次项,必须使$3t + p = 0$,此时$t = -p/3$。因此令$t = -p/3$,则
\begin{align*}
  y &= z - \frac p{3z} \\
  z^3 - \frac{p^3}{27z^3} + q &= 0
\end{align*}
值得注意的是,此时$1/z$项也被消去了。

\subsubsection{得出求根公式}

将上述等式两边同时乘以$z^3$,可得
\[ (z^3)^2 + qz^3 - \frac{p^3}{27} = 0 \]
解一元二次方程得
\begin{align*}
  z^3 &= \frac12\left(-q\pm\sqrt{q^2 + \frac{4p^3}{27}}\right) \\
  &= -\frac q2 \pm\sqrt{\frac{p^3}{27} + \frac{q^2}4}
\end{align*}
开三次根号可得6个$z$:
\begin{align*}
  \left\{ \begin{aligned}
    z_{1,2} &= \sqrt[3]{-\frac q2 \pm\sqrt{\frac{p^3}{27} + \frac{q^2}4}} \\
    z_{3,4} &= \omega\sqrt[3]{-\frac q2 \pm\sqrt{\frac{p^3}{27} + \frac{q^2}4}} \\
    z_{5,6} &= \omega^2\sqrt[3]{-\frac q2 \pm\sqrt{\frac{p^3}{27} + \frac{q^2}4}} \\
  \end{aligned} \right.
\end{align*}
其中
\[ \omega = -\frac12 + \frac{\sqrt3}2\mathi \]
由于
\begin{align*}
  &\mathrel{\phantom=} \sqrt[3]{-\frac q2 + \sqrt{\frac{p^3}{27} + \frac{q^2}4}}\cdot\sqrt[3]{-\frac q2 - \sqrt{\frac{p^3}{27} + \frac{q^2}4}} \\
  &= \sqrt[3]{\left(-\frac q2 + \sqrt{\frac{p^3}{27} + \frac{q^2}4}\right)\left(-\frac q2 - \sqrt{\frac{p^3}{27} + \frac{q^2}4}\right)} \\
  &= \sqrt[3]{\left(-\frac q2\right)^2 - \left(\frac{p^3}{27} + \frac{q^2}4\right)} \\
  &= \sqrt[3]{\frac{n^2}4 + \frac{p^3}{27} - \frac{q^2}4} \\
  &= -\frac p3
\end{align*}
又因为
\[ \omega^0 = \omega^3 = \omega^6 = \dots = 1 \]
所以
\[ z_1z_2 = z_3z_6 = z_4z_5 = -\frac p3 \]
又由于
\[ y = z - \frac p{3z} = z + \left(-\frac p3\right)\cdot\frac 1z \]
所以将$z = z_{1,2}$代入$y = z - p/(3z)$,可得$y$的一个解为
\begin{align*}
  y_1 &= z_1 + \left(-\frac p3\right)\cdot\frac 1{z_1} \\
  &= z_1 + z_1z_2\cdot\frac 1{z_1} = z_1 + z_2 \\
  y_1 &= z_2 + \left(-\frac p3\right)\cdot\frac 1{z_1} \\
  &= z_2 + z_1z_2\cdot\frac 1{z_1} = z_2 + z_1
\end{align*}
同理可得
\begin{align*}
  \left\{ \begin{aligned}
    y_1 &= z_1 + z_2 \\
    y_2 &= z_3 + z_6 \\
    y_3 &= z_4 + z_5
  \end{aligned} \right.
\end{align*}
即
\[
  \left\{ \begin{aligned}
    y_1 &= \sqrt[3]{-\frac q2 + \sqrt{\frac{p^3}{27} + \frac{q^2}4}} \\
    &+ \sqrt[3]{-\frac q2 - \sqrt{\frac{p^3}{27} + \frac{q^2}4}} \\
    y_2 &= \omega\sqrt[3]{-\frac q2 + \sqrt{\frac{p^3}{27} + \frac{q^2}4}} \\
    &+ \omega^2\sqrt[3]{-\frac q2 - \sqrt{\frac{p^3}{27} + \frac{q^2}4}} \\
    y_3 &= \omega^2\sqrt[3]{-\frac q2 + \sqrt{\frac{p^3}{27} + \frac{q^2}4}} \\
    &+ \omega\sqrt[3]{-\frac q2 - \sqrt{\frac{p^3}{27} + \frac{q^2}4}}
  \end{aligned} \right.
\]
将$y_{1,2,3}$分别代入$x = y - b/(3a)$即可得求根公式
\[
  \left\{ \begin{aligned}
    x_1 &= \sqrt[3]{-\frac q2 + \sqrt{\frac{p^3}{27} + \frac{q^2}4}} \\
    &+ \sqrt[3]{-\frac q2 - \sqrt{\frac{p^3}{27} + \frac{q^2}4}} - \frac b{3a} \\
    x_2 &= \omega\sqrt[3]{-\frac q2 + \sqrt{\frac{p^3}{27} + \frac{q^2}4}} \\
    &+ \omega^2\sqrt[3]{-\frac q2 - \sqrt{\frac{p^3}{27} + \frac{q^2}4}} - \frac b{3a} \\
    x_3 &= \omega^2\sqrt[3]{-\frac q2 + \sqrt{\frac{p^3}{27} + \frac{q^2}4}} \\
    &+ \omega\sqrt[3]{-\frac q2 - \sqrt{\frac{p^3}{27} + \frac{q^2}4}} - \frac b{3a}
  \end{aligned} \right.
\]
其中
\begin{align*}
  p &= -\frac{b^2}{3a^2} + \frac ca \\
  q &= \frac{2b^3}{27a^3} - \frac{bc}{3a^2} + \frac da \\
  \omega &= -\frac12 + \frac{\sqrt3}2\mathi
\end{align*}

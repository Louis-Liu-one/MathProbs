
\prob{00A2}{Monge圆}

\begin{figure}[htbp]
  \centering
  \image{00A2}
  \caption{总第~\ref{sec:00A2} 题图} \label{fig:00A2}
\end{figure}

如图~\ref{fig:00A2},以$O$为中心的椭圆外有一点$P$;从$P$引两条椭圆的切线$PM, PN$,切椭圆于点$M, N$,满足$PM \perp PN$。设椭圆的长轴半径为$a$,短轴半径为$b$,求证:$P$总是在一个圆心为$O$、半径为$r$的圆上,其中$r^2 = a^2 + b^2$。
\problabels{yellow/平面几何, green/证明题}

\subsection{解析几何} \label{subsec:00A2-dec}

\begin{figure}[htbp]
  \centering
  \image{00A2-dec}
  \caption{方法~\ref{subsec:00A2-dec} 图} \label{fig:00A2-dec}
\end{figure}

如图~\ref{fig:00A2-dec},以$O$为原点、椭圆两轴为坐标轴建立平面直角坐标系。设$P(x_P, y_P)$,切点坐标为$(x, y)$,切线方程为$y - y_P = k(x - x_P)$,其中$k$为斜率。

考虑关于$x, y$的方程组
\[ \left\{ \begin{aligned}
  y - y_P &= k(x - x_P) \\
  \frac{x^2}{a^2} + \frac{y^2}{b^2} &= 1
\end{aligned} \right. \]
由于切线切于椭圆,故以上方程组有两组相同解。将方程组中前者约为
\begin{align}
  y = kx - kx_P + y_P \label{eq:00A2-dec}
\end{align}
后者约为
\[ b^2x^2 + a^2y^2 - a^2b^2 = 0 \]
将式~\ref{eq:00A2-dec} 代入上式,得
\begin{align*}
  & b^2x^2 + a^2(kx - kx_P + y_P)^2 - a^2b^2 \\
  ={}& b^2x^2 + a^2k^2x^2 + a^2k^2x_P^2 + a^2y_P^2 \\
  & - 2a^2k^2x_Px + 2a^2ky_Px - 2a^2kx_Py_P \\
  & -a^2b^2 \\
  ={}& \left(b^2 + a^2k^2\right)x^2 - 2a^2k(kx_P - y_P)x \\
  & + a^2\left(k^2x_P^2 + y_P^2 - 2kx_Py_P - b^2\right) = 0
\end{align*}
上方程有两相同实数根,故
\begin{align*}
  \Delta ={}& 4a^4k^2\left(kx_P - y_P\right)^2 \\
  & - 4a^2\left(b^2 + a^2k^2\right) \\
  & \cdot \left(k^2x_P^2 + y_P^2 - 2kx_Py_P - b^2\right) \\
  \frac{\Delta}{4a^2}={}& a^2k^2\left(k^2x_P^2 + y_P^2 - 2kx_Py_P\right) \\
  & - b^2\left(k^2x_P^2 + y_P^2 - 2kx_Py_P - b^2\right) \\
  & - a^2k^2\left(k^2x_P^2 + y_P^2 - 2kx_Py_P - b^2\right) \\
  ={}& a^2x_P^2k^4 + a^2y_P^2k^2 - 2a^2x_Py_Pk^3 \\
  & - b^2x_P^2k^2 - b^2y_P^2 + 2b^2x_Py_Pk + b^4 \\
  & - a^2x_P^2k^4 - a^2y_P^2k^2 \\
  & + 2a^2x_Py_Pk^3 + a^2b^2k^2 \\
  ={}& b^2\left(a^2 - x_P^2\right)k^2 + 2b^2x_Py_Pk \\
  & + b^2\left(b^2 - y_P^2\right) = 0
\end{align*}
于是得一关于$k$的二次方程,可知其有两实数根$k_1, k_2$,分别表示$PM, PN$的斜率。

由Vieta定理知,
\[ k_1k_2 = \frac{b^2\left(b^2 - y_P^2\right)}{b^2\left(a^2 - x_P^2\right)} = \frac{b^2 - y_P^2}{a^2 - x_P^2} \]
而$PM \perp PN$,故有$k_1k_2 = -1$,代入上式化简得
\[ x_P^2 + y_P^2 = a^2 + b^2 \]
故点$P$恒在圆$x^2 + y^2 = a^2 + b^2$上,此圆的圆心为$O$,半径$r$满足$r^2 = a^2 + b^2$。证毕。

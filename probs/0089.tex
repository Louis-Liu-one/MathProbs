
\prob{0089}{欠债还钱}

甲欠乙共$x$元,其中$0 < x < 1$。然而,甲只有1元硬币,乙不能找钱。甲决定按如下方式还钱:

第1步:若欠的钱多于0.5元,则硬币换到另一人手中,该人欠对方$1 - x$元,否则不做操作。然后进行第2步。

第2步:拥有硬币的人掷硬币,若掷出正面,则硬币属于掷者,欠的钱相当于还清,停止操作;若掷出反面,则所欠金额加倍,然后进行第1步。

证明:这种还钱方式是公平的,即乙平均来说将从甲得到$x$元。
\problabels{yellow/概率论, green/证明题}

\subsection{二进制}

若每次所欠钱数改变后,亦改变$x$,则步骤可以写成:在第$n - 1$次掷硬币后,若为正面,则拥有硬币的人获得硬币,结束;若为反面,则$x \leftarrow 2x$,若$x > \sfrac12$,则交换硬币,同时$x \leftarrow 1 - x$。

将$x$写为二进制数,即将其表示为
\[ x = 2^{-1}x_1 + 2^{-2}x_2 + 2^{-3}x_3 + 2^{-4}x_4 + \cdots \]
的形式,其中所有的$x_i$非0即1。

将$x$加倍1次,相当于将$x$的二进制表示左移1位,此时
\[ x \leftarrow 2x = 2^{-1}x_2 + 2^{-2}x_3 + 2^{-3}x_4 + \cdots \]
而且进行此操作时$x \le \sfrac12$,即$x_1 = 0$,故可省略。继续可以推知,$x$加倍$n - 1$次,其间又交换硬币多次后,
\[ x = 2^{-1}x_n + 2^{-2}x_{n + 1} + 2^{-3}x_{n + 2} + 2^{-4}x_{n + 3} + \cdots \]
此时若$x_n = 1$,则$x > \sfrac12$。而操作$x \leftarrow 1 - x$可以表示为对于所有的$i$,$x_i \leftarrow 1 - x_i$,这是由于循环$n - 1$次后,
\begin{align*}
  1 - x ={}& \left(2^{-1} + 2^{-2} + 2^{-3} + 2^{-4} + \cdots\right)\cdot1 \\
  &- \left(2^{-1}x_n + 2^{-2}x_{n + 1} + 2^{-3}x_{n + 2} + \cdots\right) \\
  ={}& 2^{-1}(1 - x_n) + 2^{-2}(1 - x_{n + 1}) \\
  &+ 2^{-3}(1 - x_{n + 2})+ 2^{-4}(1 - x_{n + 3}) + \cdots
\end{align*}
记原始的$x_i$为$X_i$。

于是步骤可以写成:在第$n - 1$次掷硬币后,若为正面,则拥有硬币的人获得硬币,结束;若为反面,则将$x$左移1位,若$x_n = 1$,则交换硬币,同时所有的$i$,$x_i \leftarrow 1 - x_i$。

若在第$n$次掷硬币前均掷出反面,考虑第$n - 1$次掷硬币前,硬币的交换次数。若该次数为偶数,则第$n - 1$次掷硬币后,硬币在甲手中,且$x_i = X_i$。于是,只有$X_n = x_n = 1$时,乙才能在第$n$次掷硬币前拥有硬币;

若在第$n$次掷硬币前均掷出反面,且该次数为奇数,则则第$n - 1$次掷硬币后,硬币在乙手中,且$x_i = 1 - X_i$。于是,只有$1 - X_n = x_n = 0$,即$X_n = 1$时,乙才能在第$n$次掷硬币前拥有硬币。

因此,若且唯若$X_n = 1$且已掷出$n - 1$个反面时,乙才能在第$n$次掷硬币前拥有硬币。掷出$n - 1$个反面的概率为$\sfrac1{2^{n - 1}} = 2^{1 - n}$,故乙能在第$n$次掷硬币前拥有硬币的概率为$2^{1 - n}X_n$。而第$n$次掷出正面的概率为$\sfrac12 = 2^{-1}$,故乙恰在掷$n$次硬币后获得1元的概率为$2^nX_n$。考虑所有的$n$,可知乙最终获得1元的概率为
\[ 2^{-1}X_1 + 2^{-2}X_2 + 2^{-3}X_3 + 2^{-4}X_4 + \dots = x \]
同时乙有$1 - x$的概率从甲获得0元,故乙平均可以从甲得到
\[ 1\cdot x + 0\cdot(1 - x) = x \]
元。于是,这种还钱方式是公平的。证毕。

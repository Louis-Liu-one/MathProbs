
\prob{0002}{40度角}

\begin{figure}[htbp]
  \centering
  \image{0002}
  \caption{0002:40度角} \label{fig:0002}
\end{figure}

如图~\ref{fig:0002},$\angle C = 40^\circ$,$\angle BDC = 60^\circ$,$AB = CD$,求$\angle A$。
\problabels{yellow/平面几何, green/角度问题}

\ans{$\angle A = 30^\circ$}

\subsection{构造等边三角形} \label{subsec:0002-eqtri}

\begin{figure}[htbp]
  \centering
  \image{0002-eqtri}
  \caption{\nameref{subsec:0002-eqtri}:通过构造等边三角形找出一系列等腰三角形。} \label{fig:0002-eqtri}
\end{figure}

基本思路:构造等边三角形,然后通过一系列等腰三角形求解。

如图~\ref{fig:0002-eqtri},延长$DB$到$E$使得$DE = CE$,$E$在直线$AC$下方。在直线$AC$上找到一点$F$使得$EF = EB$,连接$EA$、$EC$、$EF$。

\begin{align*}
  &\because   DC = DE, \angle CDE = 60^\circ \\
  &\therefore \triangle CDE \text{是等边三角形} \\
  &\therefore DC = DE = CE,\\
  &\mathalignsep \angle CDE = \angle DCE = \angle DEC = 60^\circ \\
  &\because   AB = CD \\
  &\therefore AB = CE \\
  &\because   \angle ACD = 40^\circ \\
  &\therefore \angle ACE = 20^\circ \\
  &\therefore \angle ABE = 80^\circ \\
  &\because   EF = EB \\
  &\therefore \angle ABE = \angle CFE \\
  &\therefore \angle CFE = 80^\circ \\
  &\therefore \angle DEF = 20^\circ \\
  &\therefore \angle CEF = 80^\circ \\
  &\therefore \angle CFE = \angle CEF \\
  &\therefore \angle CEF = \angle ABE, CE = CF \\
  &\because   \text{在}\triangle ABE\text{与}\triangle CEF\text{中} \\
  &\mathalignsep \begin{cases}
    AB = CE \\
    \angle ABE = \angle CEF \\
    BE = EF \\
  \end{cases} \\
  &\therefore \triangle ABE \cong \triangle CEF \\
  &\therefore AE = CF, \angle AED = \angle CFE, \angle EAC = \angle ECA \\
  &\therefore DC = DE = CE = CF = AB = AE \\
  &\therefore EA = ED \\
  &\therefore \angle EAD = \angle EDA \\
  &\because   \angle AED = 80^\circ \\
  &\therefore \angle EAD = 50^\circ \\
  &\because   \angle EAC = 20^\circ \\
  &\therefore \angle CAD = 30^\circ \\
\end{align*}

综上,$\angle CAD = 30^\circ$。

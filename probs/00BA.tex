
\prob{00BA}{双平行作图}

\begin{figure}[htbp]
  \centering \image{00BA}
  \caption{总第~\ref{sec:00BA} 题图}
  \label{fig:00BA}
\end{figure}

如图~\ref{fig:00BA},直线$l_1 \parallel l_2$;在两直线之间有一点$P$,其中$P$到两直线距离不相等。仅使用无刻度直尺,过$P$作直线$l \parallel l_1$,并证明之。
\problabels{yellow/平面几何, green/作图问题}

\subsection{Ceva定理} \label{subsec:00BA-ceva}

\begin{figure}[htbp]
  \centering \image{00BA-ceva}
  \caption{方法~\ref{subsec:00BA-ceva} 图}
  \label{fig:00BA-ceva}
\end{figure}

如图~\ref{fig:00BA-ceva},自$P$引出两条不重合的直线$II', JJ'$,分别交$l_2$于$I, J$,交$l_1$与$I', J'$;作直线$IJ', I'J$,交于$K$;作直线$PK$,交$l_2$于$M$;连接$MJ'$,交$II'$于$N$,作直线$NJ$,交$IK$于$P'$。作直线$PP'$,此即为所求直线$l$。

现在证明$PP' \parallel IJ$。在$\triangle IJK$中,由(边元)Ceva定理和平行线等比例分线段,
\[ \frac{IJ'}{J'K}\cdot\frac{I'K}{I'J}\cdot\frac{JM}{IM} = 1, \frac{IJ'}{J'K} = \frac{I'J}{I'K} \Rightarrow IM = JM \]
在$\triangle IJJ'$中,由Ceva定理,
\[ \frac{PJ}{PJ'}\cdot\frac{P'J'}{P'I}\cdot\frac{IM}{JM} = 1, IM = JM \Rightarrow \frac{JJ'}{PJ'} = \frac{IJ'}{P'J'} \]
从而显然有$\triangle P'J'P \sim \triangle IJ'J \Rightarrow PP' \parallel IJ$。证毕。

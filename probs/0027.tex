
\prob{0027}{切割线定理}

\begin{figure}[htbp]
  \centering
  \image{0027}
  \caption{0027:切割线定理} \label{fig:0027}
\end{figure}

证明切割线定理:从圆外一点引圆的切线和割线,切线长是这点到割线与圆交点的两条线段长的比例中项。
\problabels{yellow/平面几何, green/证明题}

\subsection{相似三角形} \label{subsec:0027-sim}

\begin{figure}[htbp]
  \centering
  \image{0027-sim}
  \caption{\nameref{subsec:0027-sim}:通过弦切角定理证明三角形相似。}
  \label{fig:0027-sim}
\end{figure}

如图~\ref{fig:0027-sim},连接$AC, BC$。

\begin{align*}
  \because  {}& \text{据弦切角定理\footnotemark 得,$\angle CAP = \angle BCP$}, \\
  & \angle P = \angle P \\
  \therefore{}& \triangle CAP \sim \triangle BCP \\
  \therefore{}& AP:CP = CP:BP \\
\end{align*}

\footnotetext{弦切角定理的证明参见第~\ref{sec:0026} 题。}

证毕。

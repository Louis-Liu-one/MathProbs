
\prob{004D}{巧解四次方程I}

求关于$x$的方程

\begin{align*}
  & (x - 2)(x - 4)(x - 6)(x - 8) \\
  ={}& 4\left((x - 2)^2 + (x - 4)^2 + (x - 6)^2 + (x - 8)^2\right) \\
\end{align*}

的实数根。
\problabels{yellow/代数, green/方程相关问题}

\ans{$x = 5 \pm\sqrt{13 + 4\sqrt{15}}$}

\subsection{展开后换元}

基本思路:先适当展开原式两边,然后换元。

原方程左边可变为

\begin{align*}
  & (x - 2)(x - 4)(x - 6)(x - 8) \\
  ={}& (x - 2)(x - 8)\cdot(x - 4)(x - 6) \\
  ={}& (x^2 - 10x + 16)(x^2 - 10x + 24) \\
\end{align*}

原方程右边可变为

\begin{align*}
  & 4\left((x - 2)^2 + (x - 4)^2 + (x - 6)^2 + (x - 8)^2\right) \\
  ={}& 4(4x^2 - 40x + 120) = 16(x^2 - 10x + 30) \\
\end{align*}

设$y = x^2 - 10x$,则

\begin{align*}
  (y + 16)(y + 24) &= 16(y + 30) \\
  y^2 + 40y + 384 &= 16y + 480 \\
  y^2 + 24y - 96 &= 0 \\
\end{align*}

解得$y = \pm4(\sqrt{15} \mp3)$。当$y = -4(\sqrt{15} + 3)$时,

\[ x^2 - 10x = -4(\sqrt{15} + 3) \]

无解,故$y = 4(\sqrt{15} - 3)$。当$y = 4(\sqrt{15} - 3)$时,

\begin{align*}
  x^2 - 10x &= 4(\sqrt{15} - 3) \\
  x &= 5 \pm\sqrt{13 + 4\sqrt{15}} \\
\end{align*}

综上,原方程实数根为

\[ x = 5 \pm\sqrt{13 + 4\sqrt{15}} \]

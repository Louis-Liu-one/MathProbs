
\prob{0013}{角平分线定理}

\begin{figure}[htbp]
  \centering
  \image{0013}
  \caption{0013:角平分线定理} \label{fig:0013}
\end{figure}

如图~\ref{fig:0013},在$\triangle ABC$中,$AP$平分$\angle BAC$,证明角平分线定理:$AB:AC = PB:PC$。
\problabels{yellow/平面几何, green/证明题}

\subsection{面积法} \label{subsec:0013-S}

\begin{figure}[htbp]
  \centering
  \image{0013-S}
  \caption{\nameref{subsec:0013-S}:通过面积比例证明命题。}
  \label{fig:0013-S}
\end{figure}

基本思路:运用面积和角平分线的一些基本性质,根据左右两三角形面积之比证明。

如图~\ref{fig:0013-S},作$PP_1 \perp AB$于$P_1$,$PP_2 \perp AC$于$P_2$,$AP_3 \perp BC$于$P_3$。

\begin{align*}
  &\because   PP_1 \perp AB \\
  &\therefore \angle AP_1P = 90^\circ \\
  &\because   PP_2 \perp AC \\
  &\therefore \angle AP_2P = 90^\circ \\
  &\because   AP\text{平分}\angle BAC \\
  &\therefore PP_1 = PP_2 \\
  &\therefore S_{\triangle APB}:S_{\triangle APC} = \frac12AB\cdot PP_1:\frac12AC\cdot PP_2 \\& = AB:AC \\
  &\because   S_{\triangle APB}:S_{\triangle APC} = \frac12PB\cdot AP_3:\frac12PC\cdot AP_3 \\& = PB:PC \\
  &\therefore AB:AC = PB:PC \\
  &\text{证毕。} \\
\end{align*}

\subsection{正弦定理}

基本思路:利用角平分线和正弦定理推出比例。

根据正弦定理可得$AB:\sin\angle APB = PB:\sin\angle BAP$,$AC:\sin\angle APC = PC:\sin\angle CAP$。

因为$\angle APB + \angle APC = 180^\circ$,由此可得$\sin\angle APB = \sin\angle APC$,又因为$AP$平分$\angle BAC$,可得$\angle BAP = \angle CAP$,即$\sin\angle BAP = \sin\angle CAP$。

代入$AB:\sin\angle APB = PB:\sin\angle BAP$与$AC:\sin\angle APC = PC:\sin\angle CAP$可得

\[ AB:AC = PB:PC \]

证毕。

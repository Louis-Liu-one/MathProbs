
\prob{0036}{10的倍数}

若$a$、$b$、$n$为整数且$n > 1$,证明:$a^{2^n+1}b - ab^{2^n+1}$是$10$的倍数。
\problabels{yellow/数论, green/证明题}

\subsection{奇偶分析法拓展}

基本思路:通过奇偶分析法证明待求公式是$2$的倍数,然后通过奇偶分析法拓展到任意倍数的方法证明待求公式是$5$的倍数,从而证明命题。

\subsubsection{命题转化}

通过简单的因式分解可知

\vspace{-16pt}

\begin{align*}
  a^{2^n+1}b - ab^{2^n+1} &= ab(a^{2^n} - b^{2^n}) \\
  &= ab(a - b)(a + b)(a^2 + b^2) \\
  &\cdots(a^{2^{n-1}} + b^{2^{n-1}}) \\
\end{align*}

\vspace{-16pt}

由于$n > 1$,所以其中前5项$a$、$b$、$a - b$、$a + b$、$a^2 + b^2$总是在待求公式中,因此只要证明$ab(a - b)(a + b)(a^2 + b^2)$是$10$的倍数即可证明命题。

\subsubsection{证明原式为2的倍数}

现在运用奇偶分析法的思想分类讨论$a$、$b$。

若$a$、$b$中有一个是偶数,则$ab$必为偶数,因此待求公式是2的倍数。

若$a$、$b$都是奇数,不妨设$a = 2x + 1$,$b = 2y + 1$,其中$x$、$y$是整数,则$a - b = 2x - 2y = 2(x - y)$是2的倍数,因此待求公式是2的倍数。

\subsubsection{证明原式为5的倍数}

现在再次运用奇偶分析法的思想,并将其拓展到任意倍数,而不是只有2的倍数。

为了证明待求公式是$5$的倍数,不妨设$a = 5u + p$,$b = 5v + q$,其中$m$、$n$是整数,且

\begin{align*}
  p &\in \{0, 1, 2, 3, 4\} \\
  q &\in \{0, 1, 2, 3, 4\} \\
\end{align*}

\vspace{-19pt}

由此可知

\vspace{-19pt}

\begin{align*}
  &\phantom{=} ab(a - b)(a + b)(a^2 + b^2) \\
  &= (5u + p)(5v + q)\Big(5(u - v) + (p - q)\Big) \\
  &\cdot\Big(5(u + v) + (p + q)\Big)\Big((5u + p)^2 + (5v + q)^2\Big) \\
  &= (5u + p)(5v + q)\Big(5(u - v) + (p - q)\Big) \\
  &\cdot\Big(5(u + v) + (p + q)\Big)\Big(5(\dots) + (p^2 + q^2)\Big) \\
\end{align*}

\vspace{-19pt}

可以看到,每项都是形如$5r + s$的形式。将上式乘开后,只有$pq(p - q)(p + q)(p^2 + q^2)$一项中没有$5$,其它项中均有$5$,因此

\vspace{-16pt}

\begin{align*}
  &\phantom{=} ab(a - b)(a + b)(a^2 + b^2) \\
  &= pq(p - q)(p + q)(p^2 + q^2) + 5(\dots) \\
\end{align*}

\vspace{-19pt}

因此只要证明$pq(p - q)(p + q)(p^2 + q^2)$是$5$的倍数即可证明$ab(a - b)(a + b)(a^2 + b^2)$是$5$的倍数。由于$p$、$q$的取值有有限个,因此可以使用穷举法,参见表~\ref{tab:0036-eo-5mul}。

\begin{table}[htbp]
  \centering
  \begin{tabular}{ccl}
    \toprule
    $p$ & $q$ & 原式中 \\
    \midrule
    \multicolumn{2}{c}{有一个为$0$} & $pq = 0$ \\
    \multicolumn{2}{c}{相等} & $p - q = 0$ \\ \midrule
    $1$ & $4$ & \multirow{2}*{$p + q = 5$} \\
    $2$ & $3$ & \\ \midrule
    $1$ & $2$ & $p^2 + q^2 = 5$ \\
    $1$ & $3$ & $p^2 + q^2 = 10$ \\
    $2$ & $4$ & $p^2 + q^2 = 20$ \\
    $3$ & $4$ & $p^2 + q^2 = 25$ \\
    \bottomrule
  \end{tabular}
  \caption{由于将$p$和$q$交换后,原式绝对值不变,符号改变,因此我们只考虑$p \le q$的情况。}
  \label{tab:0036-eo-5mul}
\end{table}

表~\ref{tab:0036-eo-5mul} 中涵盖了所有$p$和$q$的可能组合,因此$pq(p - q)(p + q)(p^2 + q^2)$是$5$的倍数,因此待求公式是$5$的倍数。

证毕。


\prob{00B2}{平行线夹等腰}

\begin{figure}[htbp]
  \centering \image{00B2}
  \caption{总第~\ref{sec:00B2} 题图} \label{fig:00B2}
\end{figure}

如图~\ref{fig:00B2},有$\triangle ABC$;过$C$作直线$CD$使得$\angle BCD = \angle A$,$D$在$BC$右侧且满足$BC = BD$;作$DE \perp AB$于$E$;直线$AC, DE$交于$F$;过$D$作$DG \parallel AE$,交$AF$于$G$。求线段$FG$与$AC$的数量关系。
\problabels{yellow/平面几何, green/数量关系问题}

\ans{$FG = 2AC$}

\subsection{正弦定理} \label{subsec:00B2-sin}

\begin{figure}[htbp]
  \centering \image{00B2-sin}
  \caption{方法~\ref{subsec:00B2-sin} 图} \label{fig:00B2-sin}
\end{figure}

如图~\ref{fig:00B2-sin},作$\triangle DFG$的中线$DM$,设$\angle A = \alpha, \angle DCG = \beta$,则易知
\begin{align*}
  \angle A = \angle BCD &= \angle BDC \\
  = \angle MGD &= \angle MDG = \alpha \\
  \angle DCG &= \angle ABC = \beta \\
  FG &= 2MG
\end{align*}
于是易知$\triangle BCD \sim \triangle MGD$,故有
\[ \frac{MG}{BC} = \frac{GD}{CD} \]
结合上式,在$\triangle CDG$中应用正弦定理知
\[ \frac{MG}{BC} = \frac{GD}{CD} = \frac{\sin\beta}{\sin\alpha} \]
在$\triangle ABC$中应用正弦定理知
\[ \frac{BC}{AC} = \frac{\sin\alpha}{\sin\beta} \]
于是
\[ \frac{MG}{AC} = \frac{MG}{BC}\cdot\frac{BC}{AC} = 1 \Rightarrow MG = AC \]
即$FG = 2AC$。

\subsection{相似三角形} \label{subsec:00B2-sim}

\begin{figure}[htbp]
  \centering \image{00B2-sim}
  \caption{方法~\ref{subsec:00B2-sim} 图} \label{fig:00B2-sim}
\end{figure}

如图~\ref{fig:00B2-sim},作$\triangle DFG$的中线$DM$;在射线$GF$上找一点$G'$使得$DG = DG'$,$G'$不与$G$重合;设$\angle A = \alpha, \angle DCG = \beta$,则由方法~\ref{subsec:00B2-sin} 知
\[ \left\{ \begin{aligned}
  \angle A &= \angle G' = \alpha \\
  \angle ABC &= \angle G'CD = \beta
\end{aligned} \right. \Rightarrow \triangle ABC \sim \triangle G'CD \]
于是
\[ \frac{AC}{DG} = \frac{AC}{DG'} = \frac{BC}{CD} \]
而由方法~\ref{subsec:00B2-sin} 知$\triangle BCD \sim \triangle MGD$,于是有
\[ \frac{MG}{DG} = \frac{BC}{CD} \Rightarrow \frac{MG}{DG} = \frac{AC}{DG} \Rightarrow MG = AC \]
即$FG = 2AC$。

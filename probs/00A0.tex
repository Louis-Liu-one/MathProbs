
\prob{00A0}{Peaucellier-Lipkin连杆}

\begin{figure}[htbp]
  \centering
  \image{00A0}
  \caption{总第~\ref{sec:00A0} 题图} \label{fig:00A0}
\end{figure}

如图~\ref{fig:00A0},定$\odot O$上有一定点$A$和一动点$B$;在某时刻,存在两点$C, D$,连接$AC, AD, BC, BD$,满足$AC = AD, BC = BD$;以$BC, BD$为边作菱形$BCED$。证明:当$B$在圆上运动时,$E$的运动轨迹为一条垂直于$OA$的线段。
\problabels{yellow/平面几何, green/证明题}

\subsection{乘积恒定} \label{subsec:00A0-mul}

\begin{lemma} \label{lemma:00A0}
  \begin{figure}[htbp]
    \centering
    \image{009E}
    \caption{引理~\ref{lemma:00A0} 图} \label{fig:00A0-lemma}
  \end{figure}

  如图~\ref{fig:00A0-lemma},若定$\odot O$上有一定点$A$和一动点$B$,$C$是射线$AB$上一点,且满足$AB \cdot AC$为某定值,则$C$的运动轨迹为一条垂直于$OA$的直线。
\end{lemma}

\begin{proof}
  参见总第~\ref{sec:009E} 题。
\end{proof}

\begin{figure}[htbp]
  \centering
  \image{00A0-mul}
  \caption{方法~\ref{subsec:00A0-mul} 图} \label{fig:00A0-mul}
\end{figure}

如图~\ref{fig:00A0-mul},显然$A, B, E$都在$CD$的中垂线上,连接$AE, CD$,交于$H$,易知$AE \perp CD, BH = EH$。设$AC = a, BC = b, \angle CAH = \alpha$。

于是可知,在$\triangle ACH$中,
\[ CH = a\sin\alpha, AH = a\cos\alpha \]
由勾股定理可知,
\[ EH = BH = \sqrt{b^2 - a^2\sin^2\alpha} \]
于是
\begin{align*}
  AB \cdot AE &= (AH - BH)(AH + EH) \\
  &= (AH - BH)(AH + BH) = AH^2 - BH^2 \\
  &= (a\cos\alpha)^2 - \left(b^2 - a^2\sin^2\alpha\right) \\
  &= \left(\cos^2\alpha + \sin^2\alpha\right)a^2 + b^2 = a^2 + b^2
\end{align*}
可知$AB \cdot AE$为定值,于是由引理~\ref{lemma:00A0} 知$E$的运动轨迹为一条垂直于$OA$的线段。证毕。


\prob{00C1}{四边形三边到一边}

\begin{figure}[htbp]
  \centering \image{00C1}
  \caption{总第~\ref{sec:00C1} 题图} \label{fig:00C1}
\end{figure}

如图~\ref{fig:00C1},在四边形 $ABCD$ 中,对角线 $AC, BD$ 交于 $O$,$AC = BD, \angle BOC = 60^\circ$。若 $AB = \sqrt7, AD = 2, BC = \sqrt3$,求 $CD$。
\problabels{yellow/平面几何, green/长度问题}

\emph{LTY 提供的题目。}

\ans{$CD = \sqrt{13}$}

\subsection{证全等} \label{subsec:00C1-eq}

\begin{figure}[htbp]
  \centering \image{00C1-eq}
  \caption{方法~\ref{subsec:00C1-eq} 图} \label{fig:00C1-eq}
\end{figure}

如图~\ref{fig:00C1-eq},以 $AB$ 为边向上做等边 $\triangle APB$,有 $PA = PB = AB = \sqrt7, \angle APB = 60^\circ$;连接 $PC, PD$。

显然有
\[ \angle BAC + \angle PAC = \angle BAC + \angle ABD = 60^\circ \]
故 $\angle PAC = \angle ABD$。于是在 $\triangle PAC$ 与 $\triangle ABD$ 中,
\[ \left\{ \begin{aligned}
  PA &= AB \\ \angle PAC &= \angle ABD \\ AC &= BD
\end{aligned} \right. \Rightarrow \triangle PAC \cong \triangle ABD \]
故 $PC = AD = 2$,于是
\[ PC^2 + BC^2 = PB^2 \Rightarrow \angle BCP = 90^\circ \]
故 $\angle BPC + \angle PBC = 90^\circ$。

同理可得 $\triangle PBD \cong \triangle BAC \Rightarrow PD = BC = \sqrt3$,于是在 $\triangle ADP$ 与 $\triangle PCB$ 中,
\[ \left\{ \begin{aligned}
  PD = BC \\ AD = PC \\ AP = PB
\end{aligned} \right. \Rightarrow \triangle ADP \cong \triangle PCB \]
故 $\angle APD = \angle PBC \Rightarrow \angle APD + \angle BPC = 90^\circ$。而 $\angle APB = 60^\circ$,故 $\angle CPD = 150^\circ$。

由余弦定理知,
\begin{align*}
  CD^2 &= PC^2 + PD^2 - 2PC\cdot PD \cos\angle CPD \\
  &= PC^2 + PD^2 + \sqrt3PC\cdot PD \\
  &= 2^2 + \left(\sqrt3\right)^2 + \sqrt3\cdot2\cdot\sqrt3 = 13
\end{align*}
故 $CD = \sqrt{13}$。

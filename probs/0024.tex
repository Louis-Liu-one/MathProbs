
\prob{0024}{平行线上的等边}

\begin{figure}[htbp]
  \centering
  \image{0024}
  \caption{0024:平行线上的等边} \label{fig:0024}
\end{figure}

如图~\ref{fig:0024},$l_1 \parallel l_2 \parallel l_3$且$l_1$与$l_2$的距离为$a$,$l_2$与$l_3$的距离为$b$,$a < b$,边长为$c$的等边$\triangle ABC$的三个顶点$A$、$B$、$C$分别在$l_1$、$l_2$、$l_3$上,求$c$(用含$a$、$b$的代数式表示)。
\problabels{yellow/平面几何, green/数量关系问题}

\ans{$c = \sfrac23\sqrt{3(a^2 + ab + b^2)}$}

\subsection{勾股定理} \label{subsec:0024-pyth}

\begin{figure}[htbp]
  \centering
  \image{0024-pyth}
  \caption{\nameref{subsec:0024-pyth}:通过作垂线构造直角三角形,然后利用勾股定理求解。}
  \label{fig:0024-pyth}
\end{figure}

基本思路:通过作垂线构造三个直角三角形,然后利用勾股定理,结合等边三角形三边相等的性质列方程求解。

如图~\ref{fig:0024-pyth},过$B$作$UV \perp l_2$,交$l_1$于$U$,交$l_3$于$V$,过$A$作$AW \perp l_3$于$W$。

\begin{align*}
  &\because   UV \perp l_2, l_1 \parallel l_2 \\
  &\therefore UV \perp l_1 \\
  &\therefore \angle AUB = 90^\circ \\
  &\therefore UA^2 + UB^2 = AB^2 \\
  &\because   UV \perp l_2, l_3 \parallel l_2 \\
  &\therefore UV \perp l_3 \\
  &\therefore \angle BVC = 90^\circ \\
  &\therefore VB^2 + VC^2 = BC^2 \\
  &\because   AW \perp l_3 \\
  &\therefore \angle AWC = 90^\circ \\
  &\therefore WA^2 + WC^2 = AC^2, \\
  &\mathalignsep \angle AUB = \angle BVC = \angle AWC = 90^\circ \\
  &\therefore \text{四边形}AUVW\text{是矩形} \\
  &\therefore UV = WA, AU = VW \\
  &\therefore UB + VB = WA, UA = VC + WC \\
  &\because   UB = a, VB = b, AB = AC = BC = c \\
  &\therefore \begin{cases}
    WA = a + b \\
    UA^2 + a^2 = c^2 \\
    VC^2 + b^2 = c^2 \\
    WC^2 + WA^2 = c^2 \\
    UA - VC = WC \\
  \end{cases}
\end{align*}

因此可得以下关于$c$的方程:

\[ \sqrt{c^2 - a^2} - \sqrt{c^2 - b^2} = \sqrt{c^2 - (a + b)^2} \]

两边平方后移项得

\begin{align*}
  \sqrt{c^2 - a^2} - \sqrt{c^2 - b^2} &= \sqrt{c^2 - (a + b)^2} \\
  2c^2 - a^2 - b^2 & \\
  - 2\sqrt{(c^2 - a^2)(c^2 - b^2)} &= c^2 - a^2 - b^2 - 2ab \\
  2\sqrt{(c^2 - a^2)(c^2 - b^2)} &= c^2 + 2ab \\
\end{align*}

设$x = c^2$,两边平方可得一个关于$x$的一元二次方程:

\begin{align*}
  4(x - a^2)(x - b^2) &= (x + 2ab)^2 \\
  4x^2 - 4(a^2 + b^2)x + 4a^2b^2 &= x^2 + 4abx + 4a^2b^2 \\
  3x^2 - 4(a^2 + ab + b^2)x &= 0 \\
\end{align*}

此时由于$x \ne0$,可得

\begin{align*}
  3x - 4(a^2 + ab + b^2) &= 0 \\
  x &= \frac43(a^2 + ab + b^2) \\
\end{align*}

代入$x = c^2$,得

\begin{align*}
  c^2 &= \frac43(a^2 + ab + b^2) \\
  c &= \frac23\sqrt{3(a^2 + ab + b^2)} \\
\end{align*}

综上,$c = \sfrac23\sqrt{3(a^2 + ab + b^2)}$。

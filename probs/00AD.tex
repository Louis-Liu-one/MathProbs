
\prob{00AD}{面积恒等求最值}

\begin{figure}[htbp]
  \centering \image{00AD}
  \caption{总第~\ref{sec:00AD} 题图} \label{fig:00AD}
\end{figure}

如图~\ref{fig:00AD},在四边形$ABCD$中,连接$AC, BD$,有$\angle ABC = 90^\circ, \angle CAD = 60^\circ$;若$AB = 2, S_{\triangle ACD} = 6\sqrt3$,求$BD$的最大值。
\problabels{yellow/平面几何, green/最值问题}

\emph{JYH提供的题目。}

\ans{$BD$的最大值为$6 + 2\sqrt7$。}

\subsection{求轨迹} \label{subsec:00AD-l}

\begin{figure}[htbp]
  \centering \image{00AD-l}
  \caption{方法~\ref{subsec:00AD-l} 图} \label{fig:00AD-l}
\end{figure}

如图~\ref{fig:00AD-l},将$\triangle ABC$绕$A$逆时针旋转$60^\circ$至$\triangle AB'C'$处;延长$AB'$至$E$使得$AB' \cdot AE = AD \cdot AC'$,连接$DE$;作$AE$中点$O$,连接$OB, OD$。

显然$A, D, C'$共线,$\angle AB'C' = \angle ABC = 90^\circ$。由于
\[ AB' \cdot AE = AD \cdot AC' \Rightarrow \frac{AB'}{AC'} = \frac{AD}{AE} \]
同时$\angle B'AC' = \angle DAE$,故
\[ \triangle B'AC' \sim \triangle DAE \Rightarrow \angle ADE = \angle AB'C' = 90^\circ \]

又由于$S_{\triangle ACD} = 6\sqrt3, \angle CAD = 60^\circ$,故
\[ \frac12 AC \cdot AD \sin 60^\circ = 6\sqrt3 \]
即
\[ AB' \cdot AE = AC' \cdot AD = AC \cdot AD = 24 \]
而$AB' = AB = 2$,故$AE = 12$为一定值。由于$\angle ADE = 90^\circ$,故$D$的轨迹为一个圆,该圆以$AE$为直径,圆心为$O$,半径$OA = OD = OE = 6$。

由于$\angle OAB = 60^\circ$,由余弦定理知
\[ AB^2 + OA^2 - OB^2 = 2 AB \cdot OA \cos 60^\circ \]
即$OB = 2\sqrt7$。因此,当$B, O, D$共线时,$BD$取得最大值$6 + 2\sqrt7$。

\subsection{求导} \label{subsec:00AD-d}

\begin{figure}[htbp]
  \centering \image{00AD-d}
  \caption{方法~\ref{subsec:00AD-d} 图} \label{fig:00AD-d}
\end{figure}

如图~\ref{fig:00AD-d},作$DH \perp AB$于$H$;设$AC = x, \angle BAC = \alpha$,则$\angle DAH = 120^\circ - \alpha$。由方法~\ref{subsec:00AD-l} 知$AC\cdot AD = 24$。

由于$AB = 2, AC = x, BC = \sqrt{x^2 - 4}$,于是有
\[ \sin\alpha = \frac{\sqrt{x^2 - 4}}x, \cos\alpha = \frac2x \]
因此
\begin{align*}
  & \cos\angle DAH = \cos(120^\circ - \alpha) \\
  ={}& \cos120^\circ\cos\alpha + \sin120^\circ\sin\alpha \\
  ={}& \frac{\sqrt3}2\sin\alpha - \frac12\cos\alpha \\
  ={}& \frac{\sqrt{3x^2 - 12}}{2x} - \frac1x = \frac{\sqrt{3x^2 - 12} - 2}{2x}
\end{align*}
于是
\begin{align*}
  BD^2 &= BH^2 + DH^2 = (AH + 2)^2 + DH^2 \\
  &= AH^2 + DH^2 + 4AH + 4 \\
  &= AD^2 + 4AH + 4 \\
  &= \left(\frac{24}x\right)^2 + 4\cdot\frac{24}x\cos\angle DAH + 4 \\
  &= \frac{24}{x^2}\left(24 + 4\cdot\frac{\sqrt{3x^2 - 12} - 2}2\right) + 4 \\
  &= \frac{24}{x^2}\left(20 + 2\sqrt{3x^2 - 12}\right) + 4 \\
\end{align*}
令$y = 1/x^2$,则
\begin{align*}
  BD^2 &= 24\left(20y + 2\sqrt{\frac3y - 12}\cdot\sqrt{y^2}\right) + 4 \\
  &= 24\left(20y + 2\sqrt{3y - 12y^2}\right) + 4
\end{align*}
对$y$求导,得
\[ \frac\dif{\dif y}BD^2 = 24\left(20 + \frac{3 - 24y}{\sqrt{3y - 12y^2}}\right) \]

当$BD$取极值时,
\[ \frac\dif{\dif y}BD^2 = 0 \]
故而
\[ 20 + \frac{3 - 24y}{\sqrt{3y - 12y^2}} = 0 \]
即
\begin{align*}
  24y - 3 &= 20\sqrt{3y - 12y^2} \\
  24^2y^2 - 2\cdot3\cdot24y + 3^2 &= 20^2\left(3y - 12y^2\right)
\end{align*}
整理得
\begin{align}
  1792y^2 - 448y = -3 \label{eq:00AD-d}
\end{align}
解得
\[ y_1 = \frac18 + \frac5{16\sqrt7}, y_2 = \frac18 - \frac5{16\sqrt7}\ \text{(舍)} \]
故而
\[ 24\cdot20y = 60 + \frac{150}{\sqrt7} \]
同时,由式~\ref{eq:00AD-d} 可得
\[ 3y - 12y^2 = -\frac3{448}\left(1792y^2 - 448y\right) = \frac9{448} \]
因此,当$BD$取最大值时,
\begin{align*}
  BD^2 &= 24\left(20y + 2\sqrt{3y - 12y^2}\right) + 4 \\
  &= 24\cdot20y + 48\sqrt{3y - 12y^2} + 4 \\
  &= 60 + \frac{150}{\sqrt7} + 48\sqrt{\frac9{448}} + 4 \\
  &= 60 + \frac{150}{\sqrt7} + \frac{18}{\sqrt7} + 4 \\
  &= 64 + \frac{168}{\sqrt7} = 64 + 24\sqrt7 \\
  BD &= \sqrt{64 + 24\sqrt7} = 2\sqrt{16 + 6\sqrt7} \\
  &= 2\sqrt{3^2 + \left(\sqrt7\right)^2 + 2\cdot3\cdot\sqrt7} \\
  &= 2\left(3 + \sqrt7\right) = 6 + 2\sqrt7
\end{align*}
因此,$BD$的最大值为$6 + 2\sqrt7$。

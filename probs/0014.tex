
\prob{0014}{分数最小值}

已知$x, y$为正数且$x + y = 1$,求$\sfrac{x^2}{x + 1} + \sfrac{y^2}{y + 1}$的最小值。
\problabels{yellow/代数, green/最值问题}

\ans{$\sfrac{x^2}{x + 1} + \sfrac{y^2}{y + 1}$的最小值为$\sfrac13$。}

\subsection{代数变换} \label{subsec:0014-alg}

基本思路:通过代数变换将问题转换成求$xy$的最大值,然后求解。

依题意可得

\begin{align*}
  &\mathrel{\phantom{=}} \frac{x^2}{x + 1} + \frac{y^2}{y + 1} \\
  &= \frac{x^2(y + 1) + y^2(x + 1)}{(x + 1)(y + 1)} \\
  &= \frac{x^2y + xy^2 + x^2 + y^2}{xy + x + y + 1} \\
  &= \frac{xy(x + y) + x^2 + y^2}{xy + 2} \\
  &= \frac{x^2 + xy + y^2}{xy + 2} \\
  &= \frac{x^2 + 2xy + y^2 - xy}{xy + 2} \\
  &= \frac{(x + y)^2 - xy}{xy + 2} \\
  &= \frac{1 - xy}{xy + 2} \\
\end{align*}

因为$xy$总是正数,分母总是正数,所以当$xy$越大时,分母越大,分子越小,原式就越小。于是原问题就转化为当$x + y = 1$时,求$xy$的最大值。我们令$xy = k$,即求$k$的最大值。

根据二元一次方程组

\[ \begin{cases}
  x + y = 1 \\
  xy = k \\
\end{cases} \]

可得$x - y = \pm\sqrt{1 - 4k}$。由于$x - y$是实数,所以$1 - 4k \le 0$,即$k \ge \sfrac14$。\footnote{这一步可以理解为平面直角坐标系里有两个函数:$x + y = 1$和$xy = k$,我们要使$k$尽可能的大同时保证这两个函数有交点,此时$k = \sfrac14$。} 因此,$xy$的最大值为$\sfrac14$。代入原式可得其最小值为

\[ \frac{1 - \sfrac14}{\sfrac14 + 2} = \frac39 = \frac13 \]

综上,$\sfrac{x^2}{x + 1} + \sfrac{y^2}{y + 1}$的最小值为$\sfrac13$。

\subsection{三角函数}

基本思路:运用三角函数求解$xy$的最大值,进而求出原式最小值。

同样按照\nameref{subsec:0014-alg}的做法将原式变成

\[ \frac{1 - xy}{xy + 2} \]

因为$x + y = 1$,所以我们可以令$x = \sin^2\alpha, y = \cos^2\alpha$,即要求$(\sin\alpha \cos\alpha)^2$的最小值。积化和差可得

\begin{align*}
  xy &= (\sin\alpha \cos\alpha)^2 \\
  &= (\frac12\times 2\sin\alpha \cos\alpha)^2 \\
  &= (\frac12\sin2\alpha)^2 \\
  &= \frac14\sin^2 2\alpha \\
\end{align*}

当$\alpha = \pm\sfrac\pi 4$时$\sin^2 2\alpha$取得其最大值$\sin^2(\pm\sfrac\pi 2) = 1$,此时$xy$取得最大值$1\times\sfrac14 = \sfrac14$。代入原式可得原式最小值为$\sfrac13$。

综上,$\sfrac{x^2}{x + 1} + \sfrac{y^2}{y + 1}$的最小值为$\sfrac13$。

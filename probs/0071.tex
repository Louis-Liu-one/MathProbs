
\prob{0071}{四线一点}

\begin{figure}[htbp]
  \centering
  \image{0071}
  \caption{0071:四线一点} \label{fig:0071}
\end{figure}

如图~\ref{fig:0071},在锐角$\triangle ABC$中,$BD \perp AC$于$D$,$CE \perp AB$于$E$;分别过$D, E$作$BC$的垂线,交$CE$于$F$,交$BD$于$G$,交$BA$的延长线于$P$,交$CA$的延长线于$Q$。证明:直线$BC, DE, FG, PQ$交于一点。
\problabels{yellow/平面几何, green/证明题}

\subsection{解析几何} \label{subsec:0071-dec}

基本思路:先证明$DE, FG, PQ$交于一点,再利用解析几何证明$BC, DE, FG$交于一点。

\begin{figure}[htbp]
  \centering
  \image{0071-dec}
  \caption{\nameref{subsec:0071-dec}} \label{fig:0071-dec}
\end{figure}

如图~\ref{fig:0071-dec},设$BD$与$CE$交于$O$,$DU \perp BC$于$U$,$EV \perp BC$于$V$。以$BC$的中点为原点,$BC$为$x$轴建立平面直角坐标系。

\subsubsection{相似三角形}

先证明$DE, FG, PQ$交于一点。

由$PU \perp BC, QV \perp BC$易知$PU \parallel QV$,于是有
\begin{align*}
  \triangle DAP &\sim \triangle QAE \\
  \triangle DOF &\sim \triangle GOE \\
\end{align*}

而由$O$是$\triangle ABC$的垂心易知$AO \perp BC$,故$PU \parallel QV \parallel AO$。由此可知$A$到$PU$的距离与$O$到$PU$的距离相同,$A$到$QV$的距离与$O$到$QV$的距离相同,故两对相似三角形的相似比相同,即
\[ \frac{PD}{QE} = \frac{DF}{EG} \]
由此易知直线$DE, FG, PQ$交于一点。

\subsubsection{三角函数}

再证明$BC, DE, FG$交于一点。

设$BC = 2r$,于是有$D(r\cos\alpha, r\sin\alpha)$,$E(r\cos\beta, r\sin\beta)$,$B(-r, 0), C(r, 0)$。于是易知

\begin{align*}
  BD&:y = \frac{\sin\alpha}{\cos\alpha + 1}(x + r) \\
  CE&:y = \frac{\sin\beta}{\cos\beta - 1}(x - r) \\
  DU&:x = r\cos\alpha \\
  EV&:x = r\cos\beta \\
\end{align*}

将$x = r\cos\alpha$代入$CE$的函数解析式可得$F$的纵坐标为

\begin{align*}
  FU = y_F &= \frac{\sin\beta}{\cos\beta - 1}(r\cos\alpha - r) \\
  &= \frac{r\sin\beta(\cos\alpha - 1)}{\cos\beta - 1} \\
\end{align*}
同理,有
\[ GV = y_G = \frac{r\sin\alpha(\cos\beta + 1)}{\cos\alpha + 1} \]
因此有
\begin{align*}
  \frac{FU}{GV} &= \frac{\sin\alpha(\cos\beta + 1)(\cos\beta - 1)}{\sin\beta(\cos\alpha + 1)(\cos\alpha - 1)} \\
  &= \frac{\sin\alpha(1 - \cos^2\beta)}{\sin\beta(1 - \cos^2\alpha)} \\
  &= \frac{\sin\alpha\sin^2\beta}{\sin\beta\sin^2\alpha} = \frac{\sin\beta}{\sin\alpha} \\
  \frac{DU}{EV} &= \frac{r\sin\beta}{r\sin\alpha} = \frac{\sin\beta}{\sin\alpha} \\
\end{align*}
故
\[ \frac{FU}{GV} = \frac{DU}{EV} \]
因此,直线$BC, DE, FG$交于一点。

综上可知,直线$BC, DE, FG, PQ$交于一点。证毕。

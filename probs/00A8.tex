
\prob{00A8}{从等腰到等腰}

\begin{figure}[htbp]
  \centering \image{00A8}
  \caption{总第~\ref{sec:00A8} 题图} \label{fig:00A8}
\end{figure}

如图~\ref{fig:00A8},在等腰$\rttri ABC$中,$\angle ACB = 90^\circ, AC = BC$,$D$是边$BC$上一点;延长$AD$到$E$,连接$BE, CE$,使得$AB = AE$且$CE \parallel AB$。求$\angle BAE$。
\problabels{yellow/平面几何, green/角度问题}

\ans{$\angle BAE = 30^\circ$}

\subsection{作垂线} \label{subsec:00A8-h}

\begin{figure}[htbp]
  \centering \image{00A8-h}
  \caption{方法~\ref{subsec:00A8-h} 图} \label{fig:00A8-h}
\end{figure}

如图~\ref{fig:00A8-h},作$CP \perp AB$于$P$,$EQ \perp AB$于$Q$,显然四边形$PQEC$为矩形,故$CP = EQ$。同时,$\triangle ABC$为等腰直角三角形,故$AP = BP = CP$,于是$AB = 2EQ$。由于$AB = AE$,于是$AE = 2EQ$,即$\angle BAE = 30^\circ$。

\subsection{正弦定理}

不失一般性,设$AC = 1, AB = \sqrt2$,于是$AE = \sqrt2$。由$CE \parallel AB$易知$\angle ACE = 135^\circ$。

在$\triangle ACE$中应用正弦定理得
\[ \frac{\sin\angle ACE}{AE} = \frac{\sin\angle AEC}{AC} \Rightarrow \sin\angle AEC = \frac12 \]
于是$\angle BAE = \angle AEC = 30^\circ$。

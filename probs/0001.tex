
\prob{0001}{等长线段}

\begin{figure}[htbp]
  \centering
  \image{0001}
  \caption{0001:等长线段} \label{fig:0001}
\end{figure}

如图~\ref{fig:0001},$AC = AB = BD = DE$,$CD = CE$,求$\angle DCE$。
\problabels{yellow/平面几何, green/角度问题}

\ans{$\angle DCE = 100^\circ$}

\subsection{构造等边三角形} \label{subsec:0001-eqtri}

\begin{figure}[htbp]
  \centering
  \image{0001-eqtri}
  \caption{\nameref{subsec:0001-eqtri}:构造等边三角形和平行四边形。} \label{fig:0001-eqtri}
\end{figure}

基本思路:构造等边三角形,通过所得的平行四边形列角度方程求解。

如图~\ref{fig:0001-eqtri},设$\alpha = \angle D = \angle CED$,则$\angle ABC = \angle ACB = 2\alpha$。作直线$AC$右边的$A'$,使得$\angle AEA' = \angle ACB = \angle ABC = 2\alpha$且$A'E = AB$。连接$A'A$、$A'B$、$A'E$、$BE$。

\begin{align*}
  &\because   BD = AC, CD = CE \\
  &\therefore BD - CD = AC - CE \\
  &\therefore BC = EA \\
  &\therefore \triangle ABC \cong \triangle A'EA \ \text{(证明省略)} \\
  &\therefore AC = AB = BD = DE = A'A = A'E \\
  \text{又}&\because \angle AEA' = \angle ACB \\
  &\therefore A'E \parallel BD \\
  &\therefore \angle DBE = \angle A'EB \\
  &\because   \triangle DBE \cong \triangle A'EB \ \text{(证明省略)} \\
  &\therefore AB = A'A = A'B \\
  &\therefore \angle A'AB = 60^\circ \\
  &\therefore \angle EAA' - \angle BAC = 60^\circ \\
  &\because   \angle BAC' = 180^\circ - 4\alpha, \angle EAA' = 2\alpha \\
  &\therefore 6\alpha - 180^\circ = 60^\circ \\
  &\therefore \alpha = 40^\circ \\
  &\therefore \angle DCE = 180^\circ - 2\times40^\circ = 100^\circ \\
\end{align*}

综上,$\angle DCE = 100^\circ$。


\prob{00A9}{从中线到直角}

\begin{figure}[htbp]
  \centering \image{00A9}
  \caption{总第~\ref{sec:00A9} 题图} \label{fig:00A9}
\end{figure}

如图~\ref{fig:00A9},在$\triangle ABC$中,$BD$是中线,且$\angle ADB = 45^\circ$。若$\angle A = 3\angle C$,求证:$\angle ABC = 90^\circ$。
\problabels{yellow/平面几何, green/证明题}

\subsection{作垂线} \label{subsec:00A9-h}

\begin{figure}[htbp]
  \centering \image{00A9-h}
  \caption{方法~\ref{subsec:00A9-h} 图} \label{fig:00A9-h}
\end{figure}

如图~\ref{fig:00A9-h},作$BE \perp AC$于$E$;过$D$作$E'D \perp AC$于,交$BC$于$A'$;过$B$作$AC$的平行线,交$E'D$于$E'$。显然,四边形$DEBE'$是正方形,故$\angle EBE' = 90^\circ, BE = BE'$。设$\angle A = \alpha$,则$\angle C = 3\alpha$。

显然$A'D$垂直平分$AC$,于是$\angle A'AD = \angle A = \alpha$,故$\angle AA'B = 2\alpha$。同时,由于$\angle C = 3\alpha$,故$\angle A'AB = 2\alpha = \angle AA'B \Rightarrow AB = A'B$。

由于在$\rttri BAE$与$\rttri BA'E'$中,
\[ \left\{\begin{aligned}
  BA &= BA' \\ BE &= BE'
\end{aligned}\right. \]
故$\rttri BAE \cong \rttri BA'E'$,于是
\[ \angle ABE = \angle A'BE' \Rightarrow \angle ABC = \angle EBE' = 90^\circ \]
证毕。

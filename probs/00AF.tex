
\prob{00AF}{等边内一点II}

\begin{figure}[htbp]
  \centering \image{00AF}
  \caption{总第~\ref{sec:00AF} 题图} \label{fig:00AF}
\end{figure}

如图~\ref{fig:00AF},在等边$\triangle ABC$中,$AB = BC = CA = 2\sqrt3$;$P$是其中一点,作直线$BP, CP$分别交$AC, AB$于$D, E$;作$PF$平分$\angle BPC$,交$BC$于$F$。若$PF = 1, CD = AE$,求$CE$。
\problabels{yellow/平面几何, green/长度问题}

\emph{JYH提供的题目。}

\ans{$CE = 3$}

\subsection{相似三角形} \label{subsec:00AF-sim}

\begin{figure}[htbp]
  \centering \image{00AF-sim}
  \caption{方法~\ref{subsec:00AF-sim} 图} \label{fig:00AF-sim}
\end{figure}

如图~\ref{fig:00AF-sim},以$CP$为边向左作等边$\triangle PCP'$,连接$AP'$。设$AE = CD = x$,则$AD = 2\sqrt3 - x$。

显然$\triangle ACE \cong \triangle CBD$,于是
\[ \angle ACE + \angle CDB = \angle ACE + \angle AEC = 120^\circ \]
于是
\[ \angle CPD = 60^\circ \Rightarrow \angle BPF = \angle CPF = 60^\circ \]
因此$P, D, P'$共线。

显然有$\triangle ACP' \cong \triangle BCP$,故
\[ \angle AP'C = \angle BPC = 2\cdot60^\circ = 120^\circ \]
而由于$\angle BP'C = 60^\circ$,故$P'D$是$\triangle ACP'$的角平分线。而$PF$是$\triangle BCP$的角平分线,故$P'D = PF = 1$。

由$\angle AP'B = \angle CPP' = 60^\circ \Rightarrow AP' \parallel CE$知
\[ \angle DAP' = \angle ECA, \angle AP'D = \angle CAE \]
因此,$\triangle AP'D \sim \triangle CAE$,于是
\[ \frac{AD}{P'D} = \frac{CE}{AE} \Rightarrow CE = \frac{AD\cdot AE}{P'D} = x\left(2\sqrt3 - x\right) \]
故
\begin{align}
  CE^2 = \left(x^2 - 2\sqrt3x\right)^2 \label{eq:00AF-sim}
\end{align}

在$\triangle CAE$中使用余弦定理知
\[ \left(2\sqrt3\right)^2 + x^2 - CE^2 = 2\cdot2\sqrt3x\cos\angle CAE \]
得$CE^2 = x^2 - 2\sqrt3x + 12$。结合式~\ref{eq:00AF-sim} 知
\[ x^2 - 2\sqrt3x + 12 = \left(x^2 - 2\sqrt3x\right)^2 \]
解关于$x^2 - 2\sqrt3x$的二次方程得
\[ x^2 - 2\sqrt3x = -3\ \text{或} x^2 - 2\sqrt3x = 4 \]
而$CE = -\left(x^2 - 2\sqrt3x\right)$,故$CE = 3$或$CE = -4$(舍)。

综上,$CE = 3$。

\subsection{线段旋转最值} \label{subsec:00AF-min}

\begin{lemma} \label{lemma:00AF-min}
  \begin{figure}[htbp]
    \centering \image{00AF-min-l}
    \caption{引理~\ref{lemma:00AF-min} 图} \label{fig:00AF-min-l}
  \end{figure}

  如图~\ref{fig:00AF-min-l},从$P$引出两条射线$PA, PB$,使得$\angle APB = 120^\circ$;作$\angle APB$的角平分线$PQ$,其中$PQ = 1$;过$Q$作一条直线$MN$,交$PA$于$M$,交$PB$于$N$。若$MN = 2\sqrt3$,则$MN \perp PQ$。
\end{lemma}

\begin{proof}
  由总第~\ref{sec:00AE} 题知,当且仅当$MN \perp PQ$时,$MN$最小,此时易知$MN = 2\sqrt3$。而由题知$MN = 2\sqrt3$,故$MN \perp PQ$。
\end{proof}

注意到在$\triangle BPC$中,若将$PB, PC$看作定射线,$PF$看作定线段,则直线$BC$恒过点$F$且绕$F$旋转。由引理~\ref{lemma:00AF-min} 知,$BC \perp PF$,于是$PC = 2PF = 2$。又有$\triangle PBE \cong \triangle PBF$,于是$PE = PF = 1$,因此有$CE = 1 + 2 = 3$。

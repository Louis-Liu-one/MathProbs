
\prob{0085}{勾股数}

证明:若勾股数$a, b, c$互质,则其能表示为如下形式:
\begin{align*}
  a &= 2pq \\
  b &= p^2 - q^2 \\
  c &= p^2 + q^2
\end{align*}
其中$p, q$是互质的正整数,且$p > q$。
\problabels{yellow/数论, green/证明题}

\subsection{奇偶性分析}

\begin{lemma} \label{lemma:0085-abc}
  $a, b, c$两两互质。
\end{lemma}

\begin{proof}
  由$a^2 + b^2 = c^2$知,$a, b$的公因数是$a, b, c$的公因数,故$a, b$互质;由$a^2 = c^2 - b^2 = (c + b)(c - b)$知,$b, c$的公因数是$a, b, c$的公因数,故$b, c$互质;同理知$a, c$互质,故$a, b, c$两两互质。
\end{proof}

\begin{lemma} \label{lemma:0085-aboddeven}
  $a, b$奇偶性不同。
\end{lemma}

\begin{proof}
  显然$a, b$不可能均为偶数。$a, b$也不可能同为奇数,这是由于若$a, b$均为奇数,则存在正整数$m, n$,使得$a = 2m + 1, b = 2n + 1$,此时
  \begin{align*}
    & c^2 = a^2 + b^2 = (2m + 1)^2 + (2n + 1)^2 \\
    ={}& 4\left(m^2 + n^2 + m + n\right) + 2 \\
    \equiv{}& 2 \pmod 4
  \end{align*}
  而此时$c$显然为偶数,故$c^2$被4整除,不可能同余2,因此$a, b$不可能同为奇数。
  故$a, b$必然一奇一偶,即奇偶性不同。
\end{proof}

由引理~\ref{lemma:0085-aboddeven} 知$a, b$一奇一偶,不妨设$a$为偶数,$b, c$为奇数。与此同时,
\[ a^2 = c^2 - b^2 = (c + b)(c - b) = 4\cdot\frac{c + b}2\cdot\frac{c - b}2 \]
设整数$r = (c + b)/2, s = (c - b)/2$,于是$r, s$互质。这是由于$r, s$的每个公因数都是$c = r + s$和$b = r - s$的公因数,而由引理~\ref{lemma:0085-abc},$b, c$互质,故$r, s$互质。而$4rs$是平方数,故$r, s$均为平方数。令$r = p^2, s = q^2$,其中$p, q$是互质的正整数。

代入原式,有
\begin{align*}
  a &= 2pq \\
  b &= p^2 - q^2 \\
  c &= p^2 + q^2
\end{align*}
故$a, b, c$能表示成如上形式。证毕。

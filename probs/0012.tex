
\prob{0012}{Pick定理}

\begin{figure}[htbp]
  \centering
  \image{0012}
  \caption{0012:Pick定理} \label{fig:0012}
\end{figure}

如图~\ref{fig:0012},在某平面直角坐标系中,定义“格点”是横、纵坐标都是整数的的点,“格点多边形”是顶点都是格点的多边形。

求证:任一格点多边形的面积等于其内部(不包括其边缘)的格点数量与其边缘上的格点数量的一半之和减去$1$。
\problabels{yellow/解析几何, green/证明题}

\subsection{数学归纳法} \label{subsec:0012-ind}

基本思路:通过证明公式在边平行于$x$、$y$轴的矩形和直角边平行于$x$、$y$轴的直角三角形的情况下成立,从而推出公式在任意三角形的情况下成立。根据任意多边形都可以分成多个三角形的性质,若可以证明“若公式在两个多边形中都成立,则公式在两个多边形组合成的一个多边形中仍成立”,则命题得证。

设多边形面积为$S$,内点数为$p$,边缘点数为$q$,则原命题可表达为:

\[ S = p + \frac12q - 1 \]

\subsubsection{边平行于$x$、$y$轴的矩形} \label{subsubsec:0012-ind-rect}

\begin{figure}[htbp]
  \centering
  \image{0012-ind-rect}
  \caption{\nameref{subsubsec:0012-ind-rect}:归纳法的第一步,证明命题在边平行于$x$、$y$轴的矩形的情况下成立。}
  \label{fig:0012-ind-rect}
\end{figure}

如图~\ref{fig:0012-ind-rect},设边平行于$x$、$y$轴的矩形长为$a$,宽为$b$,则可知图中浅蓝色点数为$2a$,深蓝色点数为$2b$,因此边缘的点数为

\[ q = 2a + 2b \]

又通过观察图~\ref{fig:0012-ind-rect} 知其内部的红色点组成了一个长$a - 1$个点,宽$b - 1$个点的点阵,因此内点数为

\[ p = (a - 1)(b - 1) \]

代入待证公式得

\begin{align*}
  S &= p + \frac12q - 1 \\
  &= (a - 1)(b - 1) + \frac12(2a + 2b) - 1 \\
  &= ab - a - b + 1 + a + b - 1 \\
  &= ab \\
\end{align*}

即为矩形面积。因此,原命题在多边形为边平行于$x$、$y$轴的矩形的情况下成立。

\subsubsection{直角边平行于$x$、$y$轴的直角三角形} \label{subsubsec:0012-ind-rttri}

\begin{figure}[htbp]
  \centering
  \image{0012-ind-rttri}
  \caption{\nameref{subsubsec:0012-ind-rttri}:归纳法的第二步,证明命题在直角边平行于$x$、$y$轴的三角形的情况下成立。}
  \label{fig:0012-ind-rttri}
\end{figure}

如图~\ref{fig:0012-ind-rttri},设直角边平行于$x$、$y$轴的直角三角形两直角边长分别为$a$、$b$,斜边上的点数(不含首尾)为$k$\footnote{例如在图~\ref{fig:0012-ind-rttri} 中浅绿色点有1个,$k = 1$。}。易知浅蓝色点有$a$个,深蓝色点有$b$个,因此边缘点数为

\[ q = a + b + k + 1 \]

看图易知红色点与灰色点数量相同,又由于红色点、灰色点、浅绿色点组合的点阵即为大矩形的内点,且大矩形的内点有$(a - 1)(b - 1)$个,所以可知直角三角形内点数为

\[ p = \frac12\Big((a - 1)(b - 1) - k\Big) \]

代入待证公式得

\begin{align*}
  S &= p + \frac12q - 1 \\
  &= \frac12\Big((a - 1)(b - 1) - k\Big) \\
  &+ \frac12(a + b + k + 1) - 1 \\
  &= \frac12ab - \frac12a - \frac12b + \frac12 - \frac12k \\
  &+ \frac12a + \frac12b + \frac12k + \frac12 - 1 \\
  &= \frac12ab \\
\end{align*}

即为三角形面积。因此,原命题在多边形为直角边平行于$x$、$y$轴的直角三角形的情况下成立。

\subsubsection{任意三角形} \label{subsubsec:0012-ind-tri}

\begin{figure}[htbp]
  \centering
  \image{0012-ind-tri}
  \caption{\nameref{subsubsec:0012-ind-tri}:归纳法的第三步,证明命题在任意三角形的情况下成立。}
  \label{fig:0012-ind-tri}
\end{figure}

如图~\ref{fig:0012-ind-tri},设大矩形的面积为$S_R$,内点数为$p_R$,边缘点数为$q_R$,上面、右边、左下角三个直角三角形的面积、内点数、边缘点数、斜边点数(不含首尾)分别为$S_1$、$p_1$、$q_1$、$k_1$、$S_2$、$p_2$、$q_2$、$k_3$、$S_3$、$p_3$、$q_3$、$k_3$,则根据前面几步的结论可知

\begin{align*}
  S_R &= p_R + \frac12q_R - 1 \\
  S_1 &= p_1 + \frac12q_1 - 1 \\
  S_2 &= p_2 + \frac12q_2 - 1 \\
  S_3 &= p_3 + \frac12q_3 - 1 \\
\end{align*}

又由三个直角三角形的斜边点数易知中心三角形的边缘点数为

\[ q = k_1 + k_2 + k_3 + 3 \]

看图易知灰点、红点、浅绿色点的数量之和为$p_R$,灰点数量为$p_1 + p_2 + p_3$,浅绿色点数量为$k_1 + k_2 + k_3$,易知中心三角形内点数量,即红点数量为

\[ p = p_R - (p_1 + p_2 + p_3) - (k_1 + k_2 + k_3) \]

代入待证公式可知算得的中心三角形面积为

\begin{align*}
  S &= p + \frac12q - 1 \\
  &= p_R - (p_1 + p_2 + p_3) - (k_1 + k_2 + k_3) \\
  &+ \frac12(k_1 + k_2 + k_3 + 3) - 1 \\
  &= p_R - (p_1 + p_2 + p_3) - \frac12(k_1 + k_2 + k_3) + \frac12 \\
  &= (p_R + \frac12q_R - 1) - \frac12q_R + 1 \\
  &- (p_1 + p_2 + p_3) - \frac12(k_1 + k_2 + k_3) + \frac12 \\
  &= S_R - \frac12q_R - (p_1 + p_2 + p_3) \\
  &- \frac12(k_1 + k_2 + k_3) + \frac32 \\
\end{align*}

因为$q_R$是大矩形边缘点数,易知\footnote{因为$q_1 - k_1$等是每个直角三角形不算斜边的边缘点数,但是由于$k_1$等都不含首尾,因此$q_1 - k_1$等都包含中心三角形的三个顶点(绿点),因此每个绿点都被算了两次,所以要减去3。}

\begin{align*}
  q_R &= (q_1 - k_1) + (q_2 - k_2) + (q_3 - k_3) - 3 \\
  &= (q_1 + q_2 + q_3) - (k_1 + k_2 + k_3) - 3 \\
\end{align*}

将上式代入面积公式得

\begin{align*}
  S &= S_R - \frac12q_R - (p_1 + p_2 + p_3) \\
  &- \frac12(k_1 + k_2 + k_3) + \frac32 \\
  &= S_R - \frac12(q_1 + q_2 + q_3) \\
  &+ \frac12(k_1 + k_2 + k_3) + \frac32 \\
  &-(p_1 + p_2 + p_3) - \frac12(k_1 + k_2 + k_3) + \frac32 \\
  &= S_R - \Bigg((p_1 + p_2 + p_3) \\
  &+ \frac12(q_1 + q_2 + q_3)\Bigg) + 3 \\
  &= S_R - \Bigg(\left(p_1 + \frac12q_1 - 1\right) + \left(p_2 + \frac12q_2 - 1\right) \\
  &+ \left(p_3 + \frac12q_3 - 1\right)\Bigg) \\
  &= S_R - S_1 - S_2 - S_3 \\
\end{align*}

即为三角形的面积。因此,原命题在多边形为任意三角形的情况下均成立。

\subsubsection{任意多边形} \label{subsubsec:0012-ind-poly}

\begin{figure}[htbp]
  \centering
  \image{0012-ind-poly}
  \caption{\nameref{subsubsec:0012-ind-poly}:归纳法的第四步,证明命题在任意多边形的情况下成立。}
  \label{fig:0012-ind-poly}
\end{figure}

如图~\ref{fig:0012-ind-poly},设两个多边形的面积、内点数、边缘点数分别为$S_1$、$S_2$、$p_1$、$p_2$、$q_1$、$q_2$,且这两个多边形满足待证公式,即

\begin{align*}
  S_1 &= p_1 + \frac12q_1 - 1 \\
  S_2 &= p_2 + \frac12q_2 - 1 \\
\end{align*}

设两个多边形的公共边上的点(不含首尾,即浅绿色点)的数量为$k$,则易知两多边形组合成的大多边形的边缘点数为\footnotemark

\[ q = q_1 + q_2 - 2k - 2 \]

又由于红点($p_1 + p_2$)和浅绿色点($k$)的数量之和即为大多边形的内点数,因此内点数为

\[ p = p_1 + p_2 + k \]

因此根据待证公式,大多边形的面积为

\begin{align*}
  S &= p + \frac12q - 1 \\
  &= p_1 + p_2 + k + \frac12(q_1 + q_2 - 2k - 2) - 1 \\
  &= p_1 + p_2 + \frac12q_1 + \frac12q_2 - 1 - 1 \\
  &= \left(p_1 + \frac12q_1 - 1\right) + \left(p_2 + \frac12q_2 - 1\right) \\
  &= S_1 + S_2 \\
\end{align*}

结果正确。因此待证公式成立。

证毕。

\footnotetext{在$q_1 + q_2$中,绿点被多算了一遍,浅绿色点被多算了两遍,所以要减去。}


\prob{0086}{特殊的Fermat大定理}

证明:关于$a, b, c$的方程
\[ a^4 + b^4 = c^4 \]
不存在正整数解。
\problabels{yellow/数论, green/证明题}

\subsection{反证法}

\begin{lemma} \label{lemma:0086}
  若勾股数$a, b, c$互质,则其能表示为如下形式:
  \begin{align*}
    a &= 2pq \\
    b &= p^2 - q^2 \\
    c &= p^2 + q^2
  \end{align*}
  其中$p, q$是互质的正整数,且$p > q$。
\end{lemma}

\begin{proof}
  参见总第~\ref{sec:0085} 题。
\end{proof}

假设存在互质的正整数$a, b, c$满足原方程,其中$a$为偶数,于是得到勾股数$a^2, b^2, c^2$。由引理~\ref{lemma:0086} 知,存在互质的$p, q$,使得
\[ a^2 = 2pq, b^2 + q^2 = p^2, p^2 + q^2 = c^2 \]
于是得到勾股数$q, b, p$。由引理~\ref{lemma:0086} 知,存在互质的$p', q'$,使得
\[ q = 2p'q', b = p'^2 - q'^2, p = p'^2 + q'^2 \]
其中$q$是偶数是由于$b$是奇数。代入$a^2 = 2pq$得
\[ a^2 = 4p'q'\left(p'^2 + q'^2\right) \]
于是$p'q'\left(p'^2 + q'^2\right)$是完全平方数。

$p, q$互质,故$p'q'$的因数不可能同时整除$p'$与$q'$,即不能整除$p'^2 + q'^2$,故$p'q'$与$p'^2 + q'^2$互质,于是$p', q', p'^2 + q'^2$都是完全平方数。不妨设$p' = a'^2, q' = b'^2$,其中$a', b'$互质。于是$a'^4 + b'^4$是完全平方数。

因此,若$a^4 + b^4$是一完全平方数,则可以通过上述步骤构造两数$a', b'$,使得$a'^4 + b'^4$仍是完全平方数。而
\begin{align*}
  a'^4 + b'^4 &= p'^2 + q'^2 = p \\
  &< p^2 + q^2 = c^2 < c^4 = a^4 + b^4
\end{align*}
于是
\[ a'^4 + b'^4 < a^4 + b^4 \]
因此,若存在$a, b$使得$a^4 + b^4$是完全平方数,便可以再由$a, b$构造出$a', b'$、$a'', b''$……直至无穷。然而显然不可能构造无穷的$a, b$,故不存在$a, b$使得$a^4 + b^4$是完全平方数。于是亦不存在$a, b$使得$a^4 + b^4$是四次方数,即原方程不存在正整数解,证毕。


\prob{00B4}{半圆中圆}

\begin{figure}[htbp]
  \centering \image{00B4}
  \caption{总第~\ref{sec:00B4} 题图} \label{fig:00B4}
\end{figure}

如图~\ref{fig:00B4},在半圆$O_1$中,$AB$是其直径,$C$是半圆上一点,过$C$作$CD \perp AB$于$D$;在半圆内、$CD$右侧作$\odot O_2$使之与半圆和$CD$相切,且与$AB$切于$P$;连接$PC$。求证:
\[ PC^2 = 2PA\cdot PD \]
\problabels{yellow/平面几何, green/证明题}

\emph{YCY提供的题目。}

\subsection{硬算角平分线} \label{subsec:00B4-bis}

\begin{figure}[htbp]
  \centering \image{00B4-bis}
  \caption{方法~\ref{subsec:00B4-bis} 图} \label{fig:00B4-bis}
\end{figure}

如图~\ref{fig:00B4-bis},连接$AC, BC, O_1C$;令半$\odot O_1$与$\odot O_2$切于$E$,连接$O_1E, O_2P$;在线段$AD$上找一点$P'$使得$PD = P'D$;作$PH \perp BC$于$H$;过$A, P', C$作圆。设$\odot O_1$的半径为$R$,$\odot O_2$的半径为$r$。显然有$\angle DCP = \angle DCP', DP = O_2P = r$。

考虑过$E$的切线$l$,显然$O_1E \perp l, O_2E \perp l$,故$O_1, O_2, E$共线。而$O_1E = R, O_2E = r$,故$O_1O_2 = R - r$。而$O_2P \perp AB$,由勾股定理知
\[ O_1P = \sqrt{O_1O_2^2 - O_2P^2} = \sqrt{R^2 - 2Rr} \]
设$k = O_1P = \sqrt{R^2 - 2Rr}$。

由于$DP = r, O_1B = R$,故$O_1D = k - r, BP = R - k, BD = R + r - k$,由勾股定理知
\begin{align*}
  CD &= \sqrt{O_1C^2 - O_1D^2} = \sqrt{R^2 - (k - r)^2} \\
  &= \sqrt{R^2 - k^2 + 2kr - r^2} = \sqrt{2Rr + 2kr - r^2} \\
  BC &= \sqrt{CD^2 + BD^2} \\
  &= \sqrt{2Rr + 2kr - r^2 + (R + r - k)^2} \\
  &= \sqrt{4Rr + R^2 + k^2 - 2kR} \\
  &= \sqrt{4Rr + R^2 + R^2 - 2Rr - 2kR} \\
  &= \sqrt{2R^2 + 2Rr - 2kR}
\end{align*}

显然$\triangle PBH \sim \triangle CBD$,故
\begin{align*}
  PH &= \frac{CD}{BC}\cdot PB \\
  &= \frac{\sqrt{2Rr + 2kr - r^2}}{\sqrt{2R^2 + 2Rr - 2kR}}(R - k) \\
  PH^2 &= \frac{r(2R + 2k - r)}{2R(R + r - k)}(R - k)^2 \\
  &= \frac{r(2R + 2k - r)}{2R(R + r - k)}\left(R^2 - 2kR + R^2 - 2Rr\right) \\
  &= \frac{r(2R + 2k - r)}{2R(R + r - k)}2R(R - k - r) \\
  &= \frac{r(2R + 2k - r)(R - k - r)}{R + r - k} \\
\end{align*}
分子除$r$外的部分乘开得
\begin{align*}
  & (2R + 2k - r)(R - k - r) \\
  ={}& 2R^2 - 2kR - 2Rr + 2kR \\
  &- 2k^2 - 2kr - Rr + kr + r^2 \\
  ={}& 2R^2 - 3Rr - 2k^2 - kr + r^2 \\
  ={}& 2R^2 - 3Rr - 2R^2 + 4Rr - kr + r^2 \\
  ={}& r(R - k + r)
\end{align*}
故而
\[ PH^2 = \frac{r\cdot r(R - k + r)}{R + r - k} = r^2 \]
即$PH = r = PD$。故$CP$平分$\angle BCD$,于是
\[ \angle DCP' = \angle DCP = \angle BCP \]
于是有$\angle BCD = \angle PCP'$。

显然$\angle CAP' = \angle BCD$,故$\angle CAP' = \angle PCP'$,由弦切角定理逆定理知$CP$切于$\triangle ACP'$外接的圆,由切割线定理知
\[ PC^2 = PP'\cdot PA \Rightarrow PC^2 = 2PD\cdot PA \]
证毕。

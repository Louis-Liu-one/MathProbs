
\prob{002E}{对边平方和}

\begin{figure}[htbp]
  \centering
  \image{002E}
  \caption{002E:对边平方和} \label{fig:002E}
\end{figure}

证明:若四边形对角线垂直,则其对边平方和相等。
\problabels{yellow/平面几何, green/证明题}

\subsection{勾股定理} \label{subsec:002E-pyth}

\begin{figure}[htbp]
  \centering
  \image{002E-pyth}
  \caption{\nameref{subsec:002E-pyth}:四次运用勾股定理,证明命题。}
  \label{fig:002E-pyth}
\end{figure}

如图~\ref{fig:002E-pyth},设$AC$与$BD$交于点$O$。

在$\rttri AOB$与$\rttri COD$中由勾股定理知
\begin{align*}
  AB^2 &= OA^2 + OB^2 \\
  CD^2 &= OC^2 + OD^2
\end{align*}
因此,
\[ AB^2 + CD^2 = OA^2 + OB^2 + OC^2 + OD^2 \]

又在$\rttri BOC$与$\rttri AOD$中由勾股定理知
\begin{align*}
  BC^2 &= OB^2 + OC^2 \\
  AD^2 &= OA^2 + OD^2
\end{align*}
因此,
\[ BC^2 + AD^2 = OA^2 + OB^2 + OC^2 + OD^2 \]
由此可知,
\[ AB^2 + CD^2 = BC^2 + AD^2 \]

证毕。

\subsection{化积为方} \label{subsec:002E-squ}

\begin{figure}[htbp]
  \centering
  \image{002E-squ}
  \caption{\nameref{subsec:002E-squ}:作四个正方形,然后证明颜色相同的矩形面积相等。}
  \label{fig:002E-squ}
\end{figure}

如图~\ref{fig:002E-squ},设$AC, BD$交于点$O$;分别以线段$AB, BC, CD, AD$为边向外作出四个正方形$ABA_1B_1, BCB_2C_2, CDC_3D_3, ADA_4D_4$;不妨设点$Q_1, Q_2, Q_3, Q_4$分别是点$O$在线段$A_1B_1, B_2C_2, C_3D_3, A_4D_4$上的垂足,分别交$AB, BC, CD, AD$于$P_1, P_2, P_3, P_4$;连接$A_1C_2$,作$A_1M \parallel BD$交$OQ_1$于$M$,$C_2N \parallel BD$交$OQ_2$于$N$;设$L$为$A_1C_2$的中点,连接$BL$。

由于$AB \perp A_1B, BC \perp BC_2$,且$OB$是$\triangle ABC$在$AC$边上的高,$BL$是$\triangle A_1BC_2$的中线,因此应用第~\ref{sec:002D} 题的结论知$O, B, L$三点共线。因此知平行四边形$OBA_1M$和$OBC_2N$在$OB$边上的高相等,又由于它们有公共底边$OB$,所以可知
\[ S_{OBA_1M} = S_{OBC_2N} \]
又看图可知,平行四边形$OBA_1M$与矩形$A_1BP_1Q_1$共底且高相同,所以
\[ S_{OBA_1M} = S_{A_1BP_1Q_1} \]
同理知
\[ S_{OBC_2N} = S_{C_2BP_2Q_2} \]
因此,
\[ S_{A_1BP_1Q_1} = S_{C_2BP_2Q_2} \]
用同样的方法可知
\begin{align*}
  S_{B_2CP_2Q_2} &= S_{D_3CP_3Q_3} \\
  S_{C_3DP_3Q_3} &= S_{A_4DP_4Q_4} \\
  S_{D_4AP_4Q_4} &= S_{B_1AP_1Q_1}
\end{align*}
因此,
\begin{align*}
  &\mathrel{\phantom=} S_{ABA_1B_1} + S_{CDC_3D_3} \\
  &= S_{A_1BP_1Q_1} + S_{B_1AP_1Q_1} + S_{D_3CP_3Q_3} + S_{C_3DP_3Q_3} \\
  &= S_{C_2BP_2Q_2} + S_{D_4AP_4Q_4} + S_{B_2CP_2Q_2} + S_{A_4DP_4Q_4} \\
  &= S_{BCB_2C_2} + S_{ADA_4D_4}
\end{align*}
由于
\begin{align*}
  S_{ABA_1B_1} &= AB^2 \\
  S_{BCB_2C_2} &= BC^2 \\
  S_{CDC_3D_3} &= CD^2 \\
  S_{ADA_4D_4} &= AD^2
\end{align*}
所以可知,
\[ AB^2 + CD^2 = BC^2 + AD^2 \]

证毕。


\prob{000A}{等腰与二倍角}

\begin{figure}[htbp]
  \centering \image{000A}
  \caption{总第~\ref{sec:000A} 题图} \label{fig:000A}
\end{figure}

如图~\ref{fig:000A},在$\triangle ABC$中,$D, E$分别是边$BC, AC$上的点且满足$BD = 8, DE = 7, AD = CD, \angle ADE = 60^\circ, 2\angle BAD = \angle CDE$,求$AE$的长。

\problabels{yellow/平面几何, green/长度问题}

\ans{$AE = 13$}

\subsection{相似三角形} \label{subsec:000A-sim}

\begin{figure}[htbp]
  \centering \image{000A-sim}
  \caption{方法~\ref{subsec:000A-sim} 图} \label{fig:000A-sim}
\end{figure}

\emph{LTY提供的方法。}

如图~\ref{fig:000A-sim},延长$AB, ED$交于$F$;作$FH \perp AD$于$H$。设$\angle BAD = \theta$,则$\angle CDE = \angle BDF = 2\theta$。

由于$AD = CD$,故$\angle C = 60^\circ - \theta$,又有$\angle ADB = 120^\circ - 2\theta$,则$\angle DBF = 120^\circ - \theta$,于是$\angle BFD = 60^\circ - \theta = \angle C$。由于
\[ \angle BFD = \angle C, \angle BDF = \angle EDC \]
故而$\triangle BDF \sim \triangle EDC$,相似比为$8:7$。不妨设$AD = CD = 7x$,则$FD = 8x$。

由于$\angle FDH = \angle ADE = 60^\circ, DH \perp FH$,有
\[ DH = \frac12FD = 4x, FH = \frac{\sqrt3}2FD = 4\sqrt3x \]
故
\[ AH = AD + DH = 7x + 4x = 11x \]
由勾股定理知
\[ AF = \sqrt{AH^2 + FH^2} = \sqrt{169x^2} = 13x \]
由正余弦函数的定义知
\[ \sin\theta = \frac{FH}{AF} = \frac4{13}\sqrt3, \cos\theta = \frac{AH}{AF} = \frac{11}{13} \]
由差角公式知
\begin{align*}
  & \sin\angle CAD = \sin\angle C = \sin(60^\circ - \theta) \\
  ={}& \sin60^\circ\cos\theta - \cos60^\circ\sin\theta \\
  ={}& \frac{11}{26}\sqrt3 - \frac4{26}\sqrt3 = \frac7{26}\sqrt3
\end{align*}
在$\triangle ADE$中应用正弦定理知
\[ \frac{AE}{\sin\angle ADE} = \frac{DE}{\sin\angle CAD} \Rightarrow \frac23\sqrt3AE = \frac{26}3\sqrt3 \]
即$AE = 13$。

\subsection{等边三角形} \label{subsec:000A-eqtri}

\begin{figure}[htbp]
  \centering \image{000A-eqtri}
  \caption{方法~\ref{subsec:000A-eqtri} 图} \label{fig:000A-eqtri}
\end{figure}

\emph{LTY提供的方法。}

如图~\ref{fig:000A-eqtri},以$AD$为边向右作等边$\triangle ADD'$;在线段$AE$上截取$AB' = AB$,连接$B'D'$。

设$\angle BAD = \theta$,则$\angle ADC = 60^\circ + 2\theta$,故
\begin{align*}
  & \angle CAD = 60^\circ - \theta \Rightarrow \angle BAB' = 60^\circ \\
  \Rightarrow{}& \angle BAD = \angle B'AD' \Rightarrow \triangle BAD \cong \triangle B'AD' \\
  \Rightarrow{}& \angle B = \angle AB'D', BD = B'D'
\end{align*}
而$\angle ADC = 60^\circ + 2\theta, \angle BAD = \theta$,故
\[ \angle AB'D' = \angle B = 60^\circ + \theta \]
于是$\angle EB'D' = 120^\circ - \theta$。同时$\angle ADE = 60^\circ, \angle DAE = 60^\circ - \theta$,故
\[ \angle B'ED' = 120^\circ - \theta = \angle EB'D' \]
因此
\[ BD = B'D' = D'E = 8, DE = 7 \]
得
\[ AD = DD' = DE + D'E = 7 + 8 = 15 \]
由余弦定理易得$AE = 13$。

\subsection{翻折法} \label{subsec:000A-fold}

\begin{figure}[htbp]
  \centering \image{000A-fold}
  \caption{方法~\ref{subsec:000A-fold} 图} \label{fig:000A-fold}
\end{figure}

如图~\ref{fig:000A-fold},作$D'$与$D$关于直线$AB$对称,连接$AD', BD'$;在线段$AD'$上找一点$F$使得$AF = DE = 7$,连接$DF$,与$AB$交于$G$;在线段$AD$上找一点$B'$使得$BD = B'D = 8$,连接$B'G$。

\begin{align*}
  \because  {}& \text{$D'$与$D$关于直线$AB$对称} \\
  \therefore{}& AD = AD', BD = BD' \\
  \therefore{}& \triangle ABD \cong \triangle ABD'\ \text{(证明省略)} \\
  \therefore{}& \angle BAD = \angle BAD' \\
  \because  {}& \angle DAF = \angle BAD + \angle BAD' \\
  \therefore{}& \angle DAF = 2\angle BAD \\
  \because  {}& \angle CDE = 2\angle BAD \\
  \therefore{}& \angle DAF = \angle CDE \\
  \because  {}& \text{在$\triangle DAF$与$\triangle CDE$中,} \\
  & \left\{ \begin{aligned}
    DA &= CD \\ \angle DAF &= \angle CDE \\ AF &= DE \\
  \end{aligned} \right. \\
  \therefore{}& \triangle DAF \cong \triangle CDE \\
  \therefore{}& \angle ADF = \angle C \\
  \because  {}& AD = CD \\
  \therefore{}& \angle C = \angle CAD \\
  \therefore{}& \angle ADF = \angle CAD \\
  \therefore{}& AC \parallel DF \\
  \therefore{}& \angle C = \angle BDG \\
  \therefore{}& \angle BDG = \angle B'DG \\
  \because  {}& \text{在$\triangle BDG$与$\triangle B'DG$中,} \\
  & \left\{ \begin{aligned}
    BD &= B'D \\ \angle BDG &= \angle B'DG \\ DG &= DG \\
  \end{aligned} \right. \\
  \therefore{}& \triangle BDG \cong \triangle B'DG \\
  \therefore{}& \angle BGD = \angle B'GD \\
  \text{又}\because{}& \angle ADF + \angle BDF + \angle CDE + \angle ADE = 180^\circ \\
  \therefore{}& 2\angle ADF + 2\angle BAD = 180^\circ - \angle ADE \\
  \because  {}& \angle ADE = 60^\circ \\
  \therefore{}& \angle ADF + \angle BAD = 60^\circ \\
  \therefore{}& \angle BGD = 60^\circ \\
  \because  {}& \angle BGD = \angle AGF \\
  \therefore{}& \angle AGF = \angle B'GD = 60^\circ \\
  \because  {}& \angle AGF + \angle B'GD + \angle AGB' = 180^\circ \\
  \therefore{}& \angle AGB' = 60^\circ \\
  \therefore{}& \angle AGF = \angle AGB' \\
  \because  {}& \text{在$\triangle AGF$与$\triangle AGB'$中,} \\
  &\left\{ \begin{aligned}
    \angle FAG &= \angle B'AG \\ AG &= AG \\ \angle AGF &= \angle AGB' \\
  \end{aligned} \right. \\
  \therefore{}& \triangle AGF \cong \triangle AGB' \\
  \therefore{}& AF = AB' \\
  \because  {}& AF = 7 \\
  \therefore{}& AB' = 7 \\
  \because  {}& AD = AB' + B'D \\
  \therefore{}& AD = 7 + 8 = 15 \\
  \therefore{}& \text{由余弦定理易得$AE = 13$。}
\end{align*}

综上,$AE = 13$。

若不用余弦定理,则得出$AD = 15$后,也可作$EH \perp AD$于$H$。根据$DH:DE:EH = 1:2:\sqrt3$可得$EH = 7\sqrt3/2, DH = 7/2$。因此$AH = 15 - 7/2 = 23/2$,$AE = \sqrt{AH^2 + EH^2} = \sqrt{169} = 13$。作$AH \perp DE$于$H$亦可,方法与前述类似。

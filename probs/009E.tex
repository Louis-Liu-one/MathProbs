
\prob{009E}{直线运动条件}

\begin{figure}[htbp]
  \centering
  \image{009E}
  \caption{总第~\ref{sec:009E} 题图} \label{fig:009E}
\end{figure}

如图~\ref{fig:009E},圆$O$、点$A$固定,$B$在圆$O$上运动,$C$是射线$AB$上一点,且满足$AB \cdot AC$为定值。证明:当$B$运动时,$C$的轨迹为一条垂直于$OA$的直线$CH$,其中$H$为垂足。
\problabels{yellow/平面几何, green/证明题}

\subsection{相似三角形} \label{subsec:009E-sim}

\begin{figure}[htbp]
  \centering
  \image{009E-sim}
  \caption{解法~\ref{subsec:009E-sim} 图} \label{fig:009E-sim}
\end{figure}

如图~\ref{fig:009E-sim},令直线$OA$交圆$O$于点$D$,$D$不与$A$重合。易知$AD$为圆$O$的直径,故$\angle ABD = \angle H = 90^\circ$。于是知$\triangle CAH \sim \triangle DAB$,由此知$AB \cdot AC = AD \cdot AH$。

$AB \cdot AC$与$AD$均为定值,故$AH$为定值,$H$为定点。因此,在$C$的运动过程中,不论$C$在何处,其到直线$OA$的垂足总为定值,故$C$总是在垂直于$OA$的某定直线上。证毕。

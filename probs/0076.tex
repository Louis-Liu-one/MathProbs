
\prob{0076}{两弦相等}

\begin{figure}[htbp]
  \centering
  \image{0076}
  \caption{总第~\ref{sec:0076} 题图} \label{fig:0076}
\end{figure}

如图~\ref{fig:0076},若两圆$O_1, O_2$交于$P, Q$两点,求作一过$P$而不过$Q$的直线,使得其交圆$O_1$于$A$,交圆$O_2$于$B$,且弦$PA = PB$。
\problabels{yellow/平面几何, green/作图问题}

\subsection{旋转圆} \label{subsec:0076-rot}

\begin{figure}[htbp]
  \centering
  \image{0076-rot}
  \caption{总第~\ref{sec:0076} 题解法“\nameref{subsec:0076-rot}”图}
  \label{fig:0076-rot}
\end{figure}

如图~\ref{fig:0076-rot},将圆$O_1$绕点$P$旋转$180^\circ$至圆$O_1'$处,于是圆$O_1'$与圆$O_2$交点$B$即为所求作,由此易得$A$。

易证直线$AB$即为所求作。

旋转圆$O_2$或作对称等作法同理。


\prob{007C}{Khwarizmi乘法}

Al Khwarizmi提出了与竖式不同的一种用于做正整数乘法的算法。例如计算$19\times23$,按如下方式计算:

\begin{table}[htbp]
  \centering
  \begin{tabular}{rrl}
    19 &  23 & \\
     9 &  46 & \\
     4 &  92 & 划掉 \\
     2 & 184 & 划掉 \\
     1 & 368 & \\ \midrule
    \multicolumn{3}{r}{$23 + 46 + 368 = 437$}  \\
  \end{tabular}
  \caption{Al Khwarizmi乘法} \label{tab:007C}
\end{table}

将19与23如表~\ref{tab:007C} 的方式写下,然后每次将左边的数字除以2后向下取整,右边的数字乘以2;当左边减为1时,将左边为偶数的行划掉,然后计算没有划掉的行的右边的和,即为$19\times23 = 437$。

证明该算法的正确性。
\problabels{yellow/算法理论, green/证明题}

\subsection{递推式}

设左边的数为$x$,右边的数为$y$。

运用递归的思想,可以发现当左边是奇数时,算法返回
\[ xy = x + \left\lfloor\frac x2\right\rfloor\cdot2y \]
这是因为$xy$等于右边加上下面的结果;当左边为偶数时,算法返回
\[ xy = \left\lfloor\frac x2\right\rfloor\cdot2y \]
这是因为此时右边被划掉,于是$xy$直接等于下面的结果。也即,该乘法算法实现了如下规则:
\[ xy = \begin{cases}
  x + \left\lfloor\sfrac x2\right\rfloor\cdot2y, & 2 \nmid x \\
  \left\lfloor\sfrac x2\right\rfloor\cdot2y, & 2 \mid x \\
\end{cases} \]
该规则显然是正确的,故算法是正确的。证毕。

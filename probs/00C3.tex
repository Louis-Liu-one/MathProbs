
\prob{00C3}{极线与调和点列}

\begin{figure}[htbp]
  \centering \image{00C3}
  \caption{总第~\ref{sec:00C3} 题图}
  \label{fig:00C3}
\end{figure}

如图~\ref{fig:00C3},从 $\odot O$ 外一点 $A$ 引出其两条切线 $AB, AC$ 分别切 $\odot O$ 于 $B, C$,连接 $BC$;从 $A$ 引出 $\odot O$ 一条割线交 $\odot O$ 于 $P, P'$,交 $BC$ 于 $M$。求证:$A, P, M, P'$ 是调和点列,即
\[ \frac{AP}{AP'} = \frac{MP}{MP'} \]
\problabels{yellow/平面几何, green/证明题}

\emph{YCY 提供的题目。}

\subsection{代数方法} \label{subsec:00C3-ag}

\begin{figure}[htbp]
  \centering \image{00C3-ag}
  \caption{方法~\ref{subsec:00C3-ag} 图}
  \label{fig:00C3-ag}
\end{figure}

如图~\ref{fig:00C3-ag},作直线 $OA$,交 $BC$ 于 $M_0$,显然 $OA \perp BC$;作 $PH \perp OA$ 于 $H$,$P'H' \perp OA$ 于 $H'$。设
\begin{align*}
  AM_0 &= l, AB = AC = kx, AP = x, \\
  BM_0 &= CM_0 = h_1, MM_0 = h_2, \\
  MP &= p, MP' = q
\end{align*}

由切割线定理可知
\[ AC^2 = AP\cdot AP' \Rightarrow AP' = k^2x \]
于是
\[ p + q + x = k^2x \]
又由相交弦定理知
\[ MP\cdot MP' = MB\cdot MC \]
故有
\[ pq = (h_1 + h_2)(h_1 - h_2) = h_1^2 - h_2^2 \]
而
\[ h_1^2 = k^2x^2 - l^2, h_2^2 = (p + x)^2 - l^2 \]
于是有
\begin{align*}
  pq &= k^2x^2 - (p + x)^2 \\
  &= x(p + q + x) - \left(p^2 + 2px + x^2\right) \\
  &= qx - p^2 - px = qx - p(p + x) \\
  qx &= p(p + q + x) = p\cdot k^2x \\
  \frac pq &= \frac1{k^2} = \frac x{k^2x} = \frac{AP}{AP'}
\end{align*}
故
\[ \frac{MP}{MP'} = \frac{AP}{AP'} \]
证毕。

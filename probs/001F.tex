
\prob{001F}{Euler公式}

证明:
\[ \mathe^{\mathi\theta} = \cos\theta + \mathi\sin\theta \]
\problabels{yellow/微积分, green/证明题}

\subsection{Taylor展开}

因为$(\mathe^x)' = \mathe^x$,$\mathe^x$求导后不变,所以可知其Maclaurin展开为
\begin{align*}
  \mathe^x &= \frac{\mathe^0}{0!}x^0 + \frac{\mathe^0}{1!}x^1 + \frac{\mathe^0}{2!}x^2 + \frac{\mathe^0}{3!}x^3 \\
  &+ \dots + \frac{\mathe^0}{n!}x^n \\
  &= \frac1{0!}x^0 + \frac1{1!}x^1 + \frac1{2!}x^2 + \frac1{3!}x^3 \\
  &+ \dots + \frac1{n!}x^n + \cdots \\
\end{align*}
由于$\mathi^0 = 1, \mathi^1 = \mathi, \mathi^2 = -1, \mathi^3 = -\mathi, \mathi^n = \mathi^{n - 4}$,令$x = \mathi\theta$,可得
\begin{align*}
  \mathe^{\mathi\theta} &= \frac1{0!}(\mathi\theta)^0 + \frac1{1!}(\mathi\theta)^1 + \frac1{2!}(\mathi\theta)^2 \\
  &+ \frac1{3!}(\mathi\theta)^3 + \dots + \frac1{n!}(\mathi\theta)^n + \cdots \\
  &= \frac1{0!}\theta^0 + \frac1{1!}\mathi\theta^1 - \frac1{2!}\theta^2 - \frac1{3!}\mathi\theta^3 + \cdots \\
  &= \left(\frac1{0!}\theta^0 - \frac1{2!}\theta^2 + \frac1{4!}\theta^4 - \frac1{6!}\theta^6 + \dots\right) \\
  &+ \mathi\left(\frac1{1!}\theta^1 - \frac1{3!}\theta^3 + \frac1{5!}\theta^5 - \frac1{7!}\theta^7 + \dots\right) \\
\end{align*}

因为
\begin{align*}
  (\cos\theta) &= \cos\theta \qquad& (\cos\theta)' &= -\sin\theta \\
  (\cos\theta)'' &= -\cos\theta \qquad& (\cos\theta)''' &= \sin\theta \\
  (\cos\theta)'''' &= \cos\theta \qquad& \cdots &= \cdots \\
\end{align*}
可得$\cos\theta$的Maclaurin展开式为
\begin{align*}
  \cos\theta &= \frac{\cos0}{0!}\theta^0 + \frac{-\sin0}{1!}\theta^1 \\
  &+ \frac{-\cos0}{2!}\theta^2 + \frac{\sin0}{3!}\theta^3 + \cdots \\
  &= \frac1{0!}\theta^0 - \frac0{1!}\theta^1 - \frac1{2!}\theta^2 + \frac0{3!}\theta^3 + \cdots \\
  &= \frac1{0!}\theta^0 - \frac1{2!}\theta^2 + \frac1{4!}\theta^4 - \frac1{6!}\theta^6 + \cdots \\
\end{align*}
又因为
\begin{align*}
  (\sin\theta) &= \sin\theta \qquad& (\sin\theta)' &= \cos\theta \\
  (\sin\theta)'' &= -\sin\theta \qquad& (\sin\theta)''' &= -\cos\theta \\
  (\sin\theta)'''' &= \sin\theta \qquad& \cdots &= \cdots \\
\end{align*}
可得$\sin\theta$的Maclaurin展开式为
\begin{align*}
  \sin\theta &= \frac{\sin0}{0!}\theta^0 + \frac{\cos0}{1!}\theta^1 \\
  &+ \frac{-\sin0}{2!}\theta^2 + \frac{-\cos0}{3!}\theta^3 + \cdots \\
  &= \frac0{0!}\theta^0 + \frac1{1!}\theta^1 - \frac0{2!}\theta^2 - \frac1{3!}\theta^3 + \cdots \\
  &= \frac1{1!}\theta^1 - \frac1{3!}\theta^3 + \frac1{5!}\theta^5 - \frac1{7!}\theta^7 + \cdots \\
\end{align*}

对比$\mathe^{\mathi\theta}$的展开式,可得
\[ \mathe^{\mathi\theta} = \cos\theta + \mathi\sin\theta \]
证毕。\footnote{公式$\mathe^{\mathi\theta} = \cos\theta + \mathi\sin\theta$是Euler公式在$\theta = \pi$时的特殊情况,把复指数函数与三角函数联系起来,有非常重要的地位。由其导出的$\mathe^{\mathi\pi} + 1 = 0$,包含了两个重要的超越数:$\mathe$和$\pi$、实数和虚数单位:$1$和$\mathi$、数的基石:$0$、三种最基本的运算:加法、乘法和乘方、还有最重要的二元关系之一:相等,使得该公式被誉为“最美公式”之一。}

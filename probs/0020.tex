
\prob{0020}{正方形对角线}

\begin{figure}[htbp]
  \centering
  \image{0020}
  \caption{0020:正方形对角线} \label{fig:0020}
\end{figure}

如图~\ref{fig:0020},在边长为$7$的正方形$ABCD$中,$E$在线段$CD$上且$DE = 3$,$F$在线段$BD$上且$AF \perp EF$,求线段$DF$的长。
\problabels{yellow/平面几何, green/长度问题}

\ans{$DF = 5\sqrt2$}

\subsection{勾股定理} \label{subsec:0020-pyth}

\begin{figure}[htbp]
  \centering
  \image{0020-pyth}
  \caption{\nameref{subsec:0020-pyth}:通过作垂线构造直角三角形,然后利用勾股定理求解。}
  \label{fig:0020-pyth}
\end{figure}

基本思路:通过作垂线构造多个直角三角形,然后利用勾股定理列方程求解。

如图~\ref{fig:0020-pyth},作$FP \perp AD$于$P$,$EQ \perp FP$于$Q$,连接$AE$。设$DP = x$。

\begin{align*}
  &\because   \text{四边形}ABCD\text{是正方形} \\
  &\therefore \angle ADC = 90^\circ \\
  &\therefore AD^2 + DE^2 = AE^2 \\
  &\because   AD = 7, DE = 3 \\
  &\therefore AE = \sqrt{7^2 + 3^2} = \sqrt{58} \\
  &\because   FP \perp AD, EQ \perp FP \\
  &\therefore \angle DPF = \angle EQP = 90^\circ \\
  &\therefore \text{四边形}DPQE\text{是矩形} \\
  &\therefore PQ = DE = 3, EQ = DP = x \\
  &\because   BD\text{是正方形}ABCD\text{的对角线} \\
  &\therefore \angle FDP = 45^\circ \\
  &\because   \angle DPF = 90^\circ \\
  &\therefore \triangle DPF\text{是等腰直角三角形} \\
  &\therefore PD = FP = x, DF = \sqrt2x \\
  &\therefore FQ = FP - PQ = x - 3 \\
  &\because   \angle EQF = 90^\circ \\
  &\therefore EQ^2 + FQ^2 = EF^2 \\
  &\therefore EF = \sqrt{x^2 + (x - 3)^2} = \sqrt{2x^2 - 6x + 9} \\
  \text{又}&\because AD = 7 \\
  &\therefore AP = AD - DP = 7 - x \\
  &\because   \angle APF = 90^\circ \\
  &\therefore FP^2 + AP^2 = AF^2 \\
  &\therefore AF = \sqrt{x^2 + (7 - x)^2} = \sqrt{2x^2 - 14x + 49} \\
  &\because   AF \perp EF \\
  &\therefore \angle AFE = 90^\circ \\
  &\therefore AF^2 + EF^2 = AE^2 \\
  &\therefore (2x^2 - 14x + 49) + (2x^2 - 6x + 9) = 58 \\
  &\therefore x = 5 \\
  &\therefore DF = \sqrt2x = 5\sqrt2 \\
\end{align*}

综上,$DF = 5\sqrt2$。

\subsection{解析几何} \label{subsec:0020-dec}

\begin{figure}[htbp]
  \centering
  \image{0020-dec}
  \caption{\nameref{subsec:0020-dec}:通过建立平面直角坐标系利用斜率列方程求解。}
  \label{fig:0020-dec}
\end{figure}

基本思路:通过建立平面直角坐标系求出$AF$、$EF$的斜率,然后利用垂直列方程求解。

如图~\ref{fig:0020-dec},以$A$为原点,直线$AB$为$x$轴,直线$AD$为$y$轴建立平面直角坐标系$xAy$。依题意得$A(0,0)$、$B(7,0)$、$D(0,7)$、$E(3,7)$,设$x_F = x$,直线$EF$的斜率为$k_1$,直线$AF$的斜率为$k_2$。

\begin{align*}
  &\because   B(7, 0), D(0, 7) \\
  &\therefore \text{直线}BD\text{的解析式为}y = 7 - x \\
  &\because   F\text{在线段}BD\text{上} \\
  &\therefore y_F = 7 - x_F = 7 - x \\
  &\therefore F(x, 7 - x) \\
  &\therefore k_1 = \frac{(7 - x) - 7}{x - 3} = -\frac x{x - 3}, \\
  &\mathalignsep k_2 = \frac{(7 - x) - 0}{x - 0} = \frac{7 - x}x \\
  &\because   AF \perp EF \\
  &\therefore k_1 \cdot k_2 = -1 \\
  &\therefore -\frac x{x - 3} \cdot \frac{7 - x}x = -1 \\
  &\therefore \frac{7 - x}{x - 3} = 1 \\
  &\therefore 7 - x = x - 3 \\
  &\therefore x = 5 \\
  &\therefore y_F = 7 - x = 2 \\
  &\because   D(0, 7) \\
  &\therefore DF = \sqrt{(x_F - 0)^2 + (y_F - 7)^2} \\
  &\therefore DF = \sqrt{2x^2} = \sqrt2x = 5\sqrt2 \\
\end{align*}

综上,$DF = 5\sqrt2$。\footnote{建立平面直角坐标系时,也可以其它点作为原点,方法类似,不加赘述。}

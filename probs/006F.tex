
\prob{006F}{线段比例II}

\begin{figure}[htbp]
  \centering
  \image{006F}
  \caption{006F:线段比例II} \label{fig:006F}
\end{figure}

如图~\ref{fig:006F},在$\triangle ABC$中,$D$是$AB$的中点,$E$在$CD$上且$CE:DE = 1:3$,延长$AE$与$BC$交于点$F$,求$CF:BF$。
\problabels{yellow/平面几何, green/长度问题}

\ans{$CF:BF = 1:6$}

\subsection{作过$D$的平行线} \label{subsec:006F-parD}

基本思路:作过$D$的平行线,然后通过平行线等比例分线段的性质求解。

\begin{figure}[htbp]
  \centering
  \image{006F-parD}
  \caption{\nameref{subsec:006F-parD}:平行线等比例分线段。}
  \label{fig:006F-parD}
\end{figure}

如图~\ref{fig:006F-parD},作$DG \parallel AF$交$BC$于点$G$。

\begin{align*}
  \because  {}& AF \parallel DG \\
  \therefore{}& AD:BD = FG:BG \\
  \because  {}& D\text{是}AB\text{的中点} \\
  \therefore{}& AD = BD \\
  \therefore{}& BG = FG \\
  \text{又}\because{}& EF \parallel DG \\
  \therefore{}& CE:DE = CF:FG \\
  \because  {}& CE:DE = 1:3 \\
  \therefore{}& CF:FG = 1:3 \\
  \therefore{}& CF:FG:BG = 1:3:3 \\
  \therefore{}& CF:BF = 1:6 \\
\end{align*}

综上,$CF:BF = 1:6$。

\subsection{作过$B$的平行线} \label{subsec:006F-parB}

基本思路:作过$B$的平行线,然后通过平行线等比例分线段的性质求解。

\begin{figure}[htbp]
  \centering
  \image{006F-parB}
  \caption{\nameref{subsec:006F-parB}:平行线等比例分线段。}
  \label{fig:006F-parB}
\end{figure}

如图~\ref{fig:006F-parB},作$BG \parallel AF$交$CD$的延长线于点$G$。

\begin{align*}
  \because  {}& BG \parallel AE \\
  \therefore{}& \angle AED = \angle G \\
  \because  {}& D\text{是}AB\text{的中点} \\
  \therefore{}& AD = BD \\
  \because  {}& \text{在}\triangle ADE\text{与}\triangle BDG\text{中,} \\
  & \begin{cases}
    \angle AED = \angle G \\
    \angle ADE = \angle BDG \\
    AD = BD \\
  \end{cases} \\
  \therefore{}& \triangle ADE \cong \triangle BDG \\
  \therefore{}& GE = 2DE \\
  \because  {}& CE:DE = 1:3 \\
  \therefore{}& CE:GE = 1:6 \\
  \text{又}\because{}& BG \parallel EF \\
  \therefore{}& CF:BF = CE:GE = 1:6 \\
\end{align*}

综上,$CF:BF = 1:6$。

\subsection{等面积法} \label{subsec:006F-S}

基本思路:运用等面积法得出$S_{\triangle ABE}:S_{\triangle ACE}$,进而求出$CF:BF$。

\begin{figure}[htbp]
  \centering
  \image{006F-S}
  \caption{\nameref{subsec:006F-S}:求$S_{\triangle ABE}:S_{\triangle ACE}$。}
  \label{fig:006F-S}
\end{figure}

如图~\ref{fig:006F-S},连接$BE$。设$S = S_{\triangle ACE}$。

\begin{align*}
  \because  {}& CE:DE = 1:3 \\
  \therefore{}& DE = 3CE \\
  \therefore{}& S_{\triangle ADE} = 3S \\
  \because  {}& D\text{是}AB\text{的中点} \\
  \therefore{}& S_{\triangle ADE} = S_{\triangle BDE} \\
  \therefore{}& S_{\triangle ABE} = 6S \\
  \therefore{}& S_{\triangle ABE}:S_{\triangle ACE} = 1:6 \\
  \therefore{}& CF:BF = 1:6 \\
\end{align*}

综上,$CF:BF = 1:6$。


\prob{00B6}{平行线双圆}

\begin{figure}[htbp]
  \centering \image{00B6}
  \caption{总第~\ref{sec:00B6} 题图} \label{fig:00B6}
\end{figure}

如图~\ref{fig:00B6},在$\triangle ABC$中,作$CH \perp AB$于$H$;$D$是$CH$上一点,过$D$作$DM \parallel BC$交$AB$于$M$,$DN \parallel AC$交$AB$于$N$,连接$CM, CN$;在$\triangle ACM$的外接圆上找一个不与$C$重合的点$P$使得$MC = MP$,在$\triangle BCN$的外接圆上找一个不与$C$重合的点$Q$使得$NC = NQ$。求证:$A, B, P, Q$四点共圆。
\problabels{yellow/平面几何, green/证明题}

\emph{LTY提供的题目。}

\subsection{三角函数} \label{subsec:00B6-sin}

\begin{figure}[htbp]
  \centering \image{00B6-sin}
  \caption{方法~\ref{subsec:00B6-sin} 图} \label{fig:00B6-sin}
\end{figure}

如图~\ref{fig:00B6-sin},作$P, Q$关于$AB$的对称点$P', Q'$;作$MH_1 \perp AC$于$H_1$,$NH_2 \perp BC$于$H_2$;连接$AP, BQ, P'Q'$。设
\[ \angle BAC = \alpha, \angle ABC = \beta, \angle MCH = \gamma, \angle NCH = \delta \]
由平行线的性质可设
\begin{align*}
  NH &= a, MH = b, DH = h, \\
  AH &= ka, BH = kb, CH = kh
\end{align*}

由于$MC = MP, NC = NQ$,由同弦所对圆周角相等以及对称的性质知
\begin{align*}
  \angle BAC &= \angle BAP = \angle BAP', \\
  \angle ABC &= \angle ABQ = \angle ABQ', \\
  MC &= MP', NC = NQ'
\end{align*}
故$P', Q'$在$AC, BC$上,由三线合一知$CP' = 2CH_1, CQ' = 2CH_2$。由对称的性质,证明$A, B, P', Q'$四点共圆亦可证原命题。

显然
\begin{align*}
  & \angle ACM = 90^\circ - \alpha + \gamma \Rightarrow \angle CMH_1 = \alpha - \gamma \\
  &\Rightarrow CH_1 = MC\sin(\alpha - \gamma)
\end{align*}
而
\begin{align*}
  \sin\alpha &= \frac h{\sqrt{a^2 + h^2}}, \cos\alpha = \frac a{\sqrt{a^2 + h^2}} \\
  \sin\gamma &= \frac b{MC}, \cos\gamma = \frac{kh}{MC}
\end{align*}
故
\begin{align*}
  CH_1 &= MC(\sin\alpha\cos\gamma - \cos\alpha\sin\gamma) \\
  &= \frac{h\cdot kh}{\sqrt{a^2 + h^2}} - \frac{a\cdot b}{\sqrt{a^2 + h^2}} = \frac{kh^2 - ab}{\sqrt{a^2 + h^2}}
\end{align*}
同理得
\[ CH_2 = \frac{kh^2 - ab}{\sqrt{b^2 + h^2}} \]
故
\[ \frac{CP'}{CQ'} = \frac{CH_1}{CH_2} = \frac{\sqrt{b^2 + h^2}}{\sqrt{a^2 + h^2}} = \frac{CB}{CA} \]
于是$\triangle ACB \sim \triangle Q'CP'$,故
\[ \angle CP'Q' = \angle ABC \Rightarrow \angle ABQ' + \angle AP'Q' = 180^\circ \]
故$A, B, P', Q'$四点共圆,由对称的性质知$A, B, P, Q$四点共圆。证毕。

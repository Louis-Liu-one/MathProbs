
\prob{0087}{广场铺砖}

\begin{figure}[htbp]
  \centering
  \image{0087}
  \caption{总第~\ref{sec:0087} 题图} \label{fig:0087}
\end{figure}

如图~\ref{fig:0087},在$2^n\times2^n$的广场上铺如图的L形砖。证明:存在一种铺砖方式,使得广场中心的$2\times2$正方形中恰留下一个未铺砖的空格,该空格恰能放一个$1\times1$的雕像$S$。
\problabels{yellow/数学谜题, green/证明题}

\subsection{归纳法} \label{subsec:0087-sep}

若要证明中心的$2\times2$正方形中恰有一个空格放雕像,不如证明其一般化命题:不论雕像在何处,都存在一种铺砖方法铺满剩余广场。对$n$从1开始进行归纳法。

当$n = 1$时,命题显然成立,用1个L形砖即可。

当$n > 1$时,假设对于$2^{n - 1}$为边长的广场,不论雕像在何处,都存在一种铺砖方法铺满剩余广场,即要证明对于$2^n\times2^n$的广场,不论雕像在何处,都存在一种铺砖方法铺满剩余广场。

\begin{figure}[htbp]
  \centering
  \image{0087-sep}
  \caption{解法~\ref{subsec:0087-sep} 图} \label{fig:0087-sep}
\end{figure}

将大广场分为4个边长为$2^{n - 1}$的小广场,如图~\ref{fig:0087-sep}。令4个小广场的其中3个各临时放一个雕像在边角处,即$S_1, S_2, S_3$,真正的雕像放在另外一个广场的任意一处,由归纳假设可知这样可行。

然后撤走临时的$S_1, S_2, S_3$,换成一个L形砖,就从$2^{n - 1}$为边长的广场推到了$2^n$为边长的广场,于是这一一般化的命题成立。

而题目中的命题只是该一般命题的特殊情况,于是命题得证。

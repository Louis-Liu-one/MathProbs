
\prob{0084}{Eisenstein判别法}

设$f(x) = a_nx^n + a_{n - 1}x^{n - 1} + \dots + a_1x + a_0$是整系数多项式,若存在某质数$p$,使其满足
\begin{enumerate}
  \item $p$不整除$a_n$;
  \item $p$整除其它的系数;
  \item $p^2$不整除$a_0$,
\end{enumerate}
求证:$f(x)$在有理数集上不可约。
\problabels{yellow/数论, green/证明题}

\subsection{反证法}

\begin{lemma}[Gauss引理] \label{lemma:0084-Gauss}
  若$f(x)$在有理数集上可约,则其在整数集上可约。
\end{lemma}

\begin{proof}
  参见总第~\ref{sec:0083} 题。
\end{proof}

不妨假设$f(x)$在有理数集上可约,则由引理~\ref{lemma:0084-Gauss},其在整数集上可约,故存在
\begin{align*}
  g(x) &= b_rx^r + b_{r - 1}x^{r - 1} + \dots + b_1x + b_0 \\
  h(x) &= c_sx^s + c_{s - 1}x^{s - 1} + \dots + c_1x + c_0
\end{align*}
其中每个$b_i, c_i$都是整数,$r + s = n$,使得$f(x) = g(x)h(x)$。由$p \mid a_0$知$p \mid b_0c_0$,不妨设$p \mid b_0$,则由$p^2 \nmid a_0$知$p \nmid c_0$。由$f(x) = g(x)h(x)$知
\[ a_n = b_nc_0 + b_{n - 1}c_1 + \dots + b_1c_{n - 1} + b_0c_n \]
约定当$k > r$时,$b_k = 0$;当$k > s$时,$c_k = 0$。由$p \nmid a_n$知$b$至少有一个不被$p$整除,设$b$中第一个不能被$p$整除的系数为$b_i$,则$b_0, b_1, \dots, b_{i - 1}$均被$p$整除。于是
\[ a_i = b_0c_i + b_1c_{i - 1} + \dots + b_{i - 1}c_1 + b_ic_0 \]
其中$b_ic_0$不被$p$整除,其余项均被$p$整除,故$p \nmid a_i$,与题设矛盾,故原假设错误,故$f(x)$在有理数集上不可约。证毕。

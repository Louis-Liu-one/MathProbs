
\prob{0047}{实根存在性}

若关于$x$的方程

\[ ax^2 + (c - b)x + (e - d) = 0 \]

有一个比1大的实根,求证:关于$x$的方程

\[ ax^4 + bx^3 + cx^2 + dx + e = 0 \]

至少有一个实根。
\problabels{yellow/代数, green/证明题}

\subsection{零值定理}

基本思路:利用零值定理证明方程有实数根。

令$f(x) = ax^4 + bx^3 + cx^2 + dx + e$,即要证明:存在一个实数$x$,使得$f(x) = 0$。

令方程$ax^2 + (c - b)x + (e - d) = 0$的这个比1大的实根为$m^2$,其中$m > 1$。易知

\begin{align*}
  am^4 + (c - b)m^2 + (e - d) &= 0 \\
  am^4 + cm^2 + e &= bm^2 + d \\
\end{align*}

由此易知

\begin{align*}
  f(m) &= am^4 + bm^3 + cm^2 + dm + e \\
  &= (am^4 + cm^2 + e) + m(bm^2 + d) \\
  &= (1 + m)(bm^2 + d) \\
  f(-m) &= am^4 - bm^3 + cm^2 - dm + e \\
  &= (am^4 + cm^2 + e) - m(bm^2 + d) \\
  &= (1 - m)(bm^2 + d) \\
\end{align*}

由$m > 1$易知$1 + m$与$1 - m$符号不同,因此$f(m)$与$f(-m)$符号不同。

由零值定理知在$[-m, m]$之间,必然存在一实数$x$,使得$f(x) = 0$,故$f$有零值点,原方程存在实数根。证毕。

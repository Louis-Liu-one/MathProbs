
\prob{00BF}{Sylvester-Gallai 定理}

求证:对于平面上任意有限点集,若此点集中所有点不共线,则必然存在至少一条直线,使得其通过此点集中两点,且不通过此点集中其它点。
\problabels{yellow/平面几何, green/证明题}

\subsection{反证法} \label{subsec:00BF-cont}

\begin{figure}[htbp]
  \centering \image{00BF-cont}
  \caption{方法~\ref{subsec:00BF-cont} 图}
  \label{fig:00BF-cont}
\end{figure}

对于一有限点集$S_P$,此点集中所有点不共线,假设过此点集中过任意两点的直线都过其中第三点。

过$S_P$中的任意两点作一条直线,这些直线组成一个直线集$S_L$;$\forall l \in S_L$,从属于$S_P$且在$l$外的每一点作$l$的垂线,所有垂线组成一个垂线集$S_H$;显然存在一条垂线$P_0H$在$S_H$中最短。如图~\ref{fig:00BF-cont},设这条垂线是从$P_0 \in S_P$向直线$l_0 \in S_L$所作的垂线,垂足为$H$。

由假设,$l_0$上必然存在至少三个$S_P$中的点,故在$H$的两侧必然有一侧存在至少两个$S_P$中的点,将此两点记作$A, B$,且$AH > BH$。显然,直线$AP_0 \in S_L$。过$B$作$BH' \perp AP_0$于$H'$,显然$BH' \in S_H$且$BH' < P_0H$。然而,$P_0H$是$S_H$中最短的垂线,故矛盾,假设不成立,原命题成立。证毕。

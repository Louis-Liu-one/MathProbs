
\prob{0016}{两垂直求面积}

\begin{figure}[htbp]
  \centering
  \image{0016}
  \caption{0016:两垂直求面积} \label{fig:0016}
\end{figure}

如图~\ref{fig:0016},在$\triangle ABC$中,$AD \perp BC$于$D$,$CE \perp AB$于$E$,$EM$平分$\angle BEC$交$AD$的延长线于$M$,连接$BM$、$CM$、$DM$,若$\angle CFD + \angle ABM = 180^\circ$,$2AE = 5BE$,$S_{\triangle AEF} = 5$,求$\triangle CEM$的面积。
\problabels{yellow/平面几何, green/面积问题}

\ans{$S_{\triangle CEM} = \sfrac{25}3$}

\subsection{蝴蝶模型} \label{subsec:0016-butf}

基本思路:通过证明两个全等,得出$AC \parallel EM$以及$EF:CF$,然后运用蝴蝶模型求解。

\begin{align*}
  &\because   \angle CFD + \angle ABM = 180^\circ, \\
  &\mathalignsep \angle CFD + \angle EFM = 180^\circ \\
  &\therefore \angle EBM = \angle EFM, \angle CEM = \frac12\angle BEC \\
  &\because   EM\text{平分}\angle BEC \\
  &\therefore \angle BEM = \angle FEM \\
  &\therefore \triangle BEM \cong \triangle FEM\ \text{(证明省略)} \\
  &\therefore EB = EF \\
  &\because   2AE = 5BE \\
  &\therefore 2AE = 5EF \\
  &\because   S_{\triangle AEF} = 5 \\
  &\therefore \frac12AE\cdot EF = 5 \\
  &\therefore AE = 5, EF = 2 \\
  \text{又}&\because CE \perp AB \\
  &\therefore \angle AEF = 90^\circ \\
  &\therefore \angle EAF + \angle AFE = 90^\circ \\
  &\because   \angle AEF + \angle CEB = 180^\circ \\
  &\therefore \angle CEB = 90^\circ \\
  &\therefore \angle AEF = \angle CEB, \angle CEM = 45^\circ \\
  \text{又} &\because AD \perp BC \\
  &\therefore \angle ADB = 90^\circ \\
  &\therefore \angle EAF + \angle CBE = 90^\circ \\
  &\therefore \angle AFE = \angle CBE \\
  &\because   \text{在}\triangle AEF\text{与}\triangle CEB\text{中} \\
  &\mathalignsep \begin{cases}
    \angle AEF = \angle CEB \\
    EF = EB \\
    \angle AFE = \angle CBE \\
  \end{cases} \\
  &\therefore \triangle AEF \cong \triangle CEB \\
  &\therefore AE = CE \\
  &\therefore \angle ACE = 90^\circ - \frac12\angle AEC = 45^\circ \\
  &\because   \angle CEM = 45^\circ \\
  &\therefore \angle ACE = \angle CEM \\
  &\therefore AC \parallel EM \\
  &\therefore S_{\triangle AEM} = S_{\triangle CEM} \\
  &\therefore S_{\triangle AEF} = S_{\triangle CMF} = 5 \\
  &\because   AE = 5 \\
  &\therefore CE = 5 \\
  &\because   EF = 2 \\
  &\therefore CF = 3 \\
  &\therefore EF:CF = 2:3 \\
  &\because   S_{\triangle EMF}:S_{\triangle CMF} = EF:CF \\
  &\therefore S_{\triangle EMF}:5 = 2:3 \\
  &\therefore S_{\triangle EMF} = \frac{10}3 \\
  &\because   S_{\triangle CEM} = S_{\triangle EMF} + S_{\triangle CMF} \\
  &\therefore S_{\triangle CEM} = \frac{10}3 + 5 = \frac{25}3 \\
\end{align*}

综上,$\triangle CEM$的面积为$\sfrac{25}3$。

\subsection{作垂线} \label{subsec:0016-vert}

\begin{figure}[htbp]
  \centering
  \image{0016-vert}
  \caption{\nameref{subsec:0016-vert}:通过作垂线构造等腰直角三角形,然后利用正弦定理和相似三角形求解。}
  \label{fig:0016-vert}
\end{figure}

基本思路:通过作垂线作出等腰直角三角形,然后根据相似三角形的比例关系结合等腰直角三角形的性质求解。

如图~\ref{fig:0016-vert},作$MG \perp AB$于$G$。设$x = BG$。

\begin{align*}
  &\because   S_{\triangle BEM} = \frac12BE\cdot EM\sin\angle BEM, \\
  &\mathalignsep S_{\triangle CEM} = \frac12CE\cdot EM\sin\angle CEM \\
  &\therefore S_{\triangle BEM}:S_{\triangle CEM} \\
  &\mathalignsep = BE\sin\angle BEM:CE\sin\angle CEM \\
  &\because   EM\text{平分}\angle BEC \\
  &\therefore \angle BEM = \angle CEM \\
  &\therefore \sin\angle BEM = \sin\angle CEM \\
  &\therefore S_{\triangle BEM}:S_{\triangle CEM} = BE:CE \\
  &\because   \angle EAF = \angle GAM, \angle AEF = \angle G \\
  &\therefore \triangle EAF \sim \triangle GAM \\
  &\therefore AE:AG = EF:GM \\
  &\because   \triangle BEM \cong \triangle FEM \\
  &\mathalignsep \text{(证法参见\nameref{subsec:0016-butf})}, \\
  &\mathalignsep S_{\triangle AEF} = 5 \\
  &\therefore AE = 5, EB = EF = 2 \\
  &\mathalignsep \text{(证法参见\nameref{subsec:0016-butf})} \\
  &\because   \angle GEM = \angle CEM, \angle FEG = 90^\circ \\
  &\therefore \angle GEM = 45^\circ \\
  &\because   \angle G = 90^\circ \\
  &\therefore \angle EMG = 45^\circ \\
  &\therefore \angle GEM = \angle EMG \\
  &\therefore GE = GM \\
  &\because   GE = x + 2 \\
  &\therefore GM = x + 2, AG = x + 7 \\
  &\therefore \frac5{x + 7} = \frac2{x + 2} \\
  &\therefore x = \frac{14}3 \\
  &\therefore GM = x + 2 = \frac{20}3 \\
  &\therefore S_{\triangle BEM} = \frac12EB\cdot GM = \frac{10}3 \\
  &\because   \triangle AEF \cong \triangle BEC \\
  &\mathalignsep \text{(证法参见\nameref{subsec:0016-butf})} \\
  &\therefore AE = CE = 5 \\
  &\therefore \frac{10}3:S_{\triangle CEM} = 2:5 \\
  &\therefore S_{\triangle CEM} = \frac{25}3 \\
\end{align*}

综上,$\triangle CEM$的面积为$\sfrac{25}3$。

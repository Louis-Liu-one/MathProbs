
\prob{006A}{连续合数}

求证:可以找到连续$n$个正整数,使得其中恰有$1$个质数。
\problabels{yellow/数论, green/证明题}

\subsection{阶乘}

基本思路:考虑$n!$的后面几个数,可见其都是合数;然后一直将该数列右移,直到其右端出现一个质数为止。

考虑$n! + 2, n! + 3, n! + 4, \dots, n! + n$,这$n - 1$个数显然都是合数。又考虑$n! + n + 1$,若此数为质数,则找到了满足条件的正整数。

若此数为合数,则这$n$个数都是合数。故将这$n$个数每个加上$1$,则出现一个新数$n! + n + 2$;若此数为质数,则找到了满足条件的正整数;若此数为合数,则重复前操作,于是又出现一个新数$n! + n + 3$……

由质数的无穷性可知经过有限次操作后,产生的新数必会是质数,故总能找到连续$n$个正整数,使得其中恰有$1$个质数。

证毕。


\prob{000B}{中线求角}

\begin{figure}[htbp]
  \centering
  \image{000B}
  \caption{000B:中线求角} \label{fig:000B}
\end{figure}

如图~\ref{fig:000B},在$\triangle ABC$中,$AD = CD$,$\angle A = 30^\circ$,$\angle C = 15^\circ$,求$\angle CBD$。
\problabels{yellow/平面几何, green/角度问题}

\ans{$\angle CBD = 30^\circ$}

\subsection{构造等腰直角三角形} \label{subsec:000B-isrtri}

\begin{figure}[htbp]
  \centering
  \image{000B-isrtri}
  \caption{\nameref{subsec:000B-isrtri}:通过构造等腰直角三角形找到等边三角形,最后利用一系列等腰三角形求解。}
  \label{fig:000B-isrtri}
\end{figure}

基本思路:通过构造一个等腰直角三角形和一个等边三角形,得出一个包含所求角的等腰三角形,最后通过等腰三角形的性质求解。

如图~\ref{fig:000B-isrtri},作$CE \perp AB$于$E$,连接$DE$。

\begin{align*}
  &\because   CE \perp AB \\
  &\therefore \angle BEC = 90^\circ \\
  &\therefore \angle A + \angle ACE = 90^\circ, \angle BCE + \angle CBE = 90^\circ \\
  &\because   \angle A = 30^\circ \\
  &\therefore \angle ACE = 90^\circ - 30^\circ = 60^\circ, AC = 2CE \\
  &\because   AD = CD \\
  &\therefore CD = CE \\
  &\therefore \triangle CDE\text{是等边三角形} \\
  &\therefore \angle DEC = 60^\circ, CE = DE \\
  &\therefore \angle BED = 90^\circ - 60^\circ = 30^\circ \\
  \text{又}&\because \angle ACB = 15^\circ \\
  &\therefore \angle BCE = 60^\circ - 15^\circ = 45^\circ \\
  &\therefore \angle CBE = 90^\circ - 45^\circ = 45^\circ \\
  &\therefore \angle BCE = \angle CBE \\
  &\therefore BE = CE \\
  &\therefore BE = DE \\
  &\therefore \angle BDE = \angle DBE \\
  &\because   \angle BDE + \angle DBE + \angle BED = 180^\circ \\
  &\therefore \angle DBE = \frac12(180^\circ - 30^\circ) = 75^\circ \\
  &\therefore \angle CBD = 75^\circ - 45^\circ = 30^\circ \\
\end{align*}

综上,$\angle CBD = 30^\circ$。

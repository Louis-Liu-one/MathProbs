
\prob{0004}{线段比例I}

\begin{figure}[htbp]
  \centering
  \image{0004}
  \caption{0004:线段比例I} \label{fig:0004}
\end{figure}

如图~\ref{fig:0004},在$\triangle ABC$中,$AF:EF:EB:EC = 5:2:2:3$,求证:$DA = DC$。
\problabels{yellow/平面几何, green/证明题}

\subsection{面积法} \label{subsec:0004-S}

\begin{figure}[htbp]
  \centering
  \image{0004-S}
  \caption{\nameref{subsec:0004-S}:连接$CF$,再根据面积比例求解。} \label{fig:0004-S}
\end{figure}

基本思路:寻找两个面积相等,底也相等的三角形,根据它们的高相等这一结论推出一个8形全等。

如图~\ref{fig:0004-S},连接$CF$,作$AA' \perp BF$、$CC' \perp BF$,垂足分别为$A'$、$C'$。

\begin{align*}
  &\because   AF:EF = 5:2 \\
  &\therefore S_{\triangle AFB}:S_{\triangle EFB} = 5:2 \\
  &\because   BE:CE = 2:3 \\
  &\therefore S_{\triangle EFB}:S_{\triangle EFC} = 2:3 \\
  &\therefore S_{\triangle AFB}:S_{\triangle CFB} = 1:1 \\
  &\therefore S_{\triangle AFB} = S_{\triangle CFB} \\
  &\therefore \frac12BF \cdot AA' = \frac12BF \cdot CC' \\
  &\therefore AA' = CC' \\
  &\therefore \triangle AA'D \cong \triangle CC'D \ \text{(证明省略)} \\
  &\therefore DA = DC \\
  &\text{证毕。} \\
\end{align*}

\subsection{倍长“中线”法} \label{subsec:0004-mid}

\begin{figure}[htbp]
  \centering
  \image{0004-mid}
  \caption{\nameref{subsec:0004-mid}:延长$BD$,然后构造8形全等。} \label{fig:0004-mid}
\end{figure}

\emph{ZYQ提供的方法。}

基本思路:构造等腰三角形,然后找出一个8形全等,进而证明。

如图~\ref{fig:0004-mid},延长$BD$到点$B'$,使得$AB' = CB = 5$。连接$AB'$、$DB'$。

\begin{align*}
  &\because   AB' = CB, CB = AD \\
  &\therefore AB' = AD \\
  &\therefore \angle B' = \angle AFD \\
  &\because   \angle AFD = \angle BFE = \angle EBF \\
  &\therefore \angle B' = \angle EBF \\
  \text{又}&\because \text{在}\triangle ADB'\text{与}\triangle CDB\text{中} \\
  &\mathalignsep \begin{cases}
    AB' = CB \\
    \angle B' = \angle EBF \\
    \angle ADB' = \angle CDB \\
  \end{cases} \\
  &\therefore \triangle ADB' \cong \triangle CDB \\
  &\therefore DA = DC \\
  &\text{证毕。} \\
\end{align*}

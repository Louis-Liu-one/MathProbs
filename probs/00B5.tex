
\prob{00B5}{双外心等腰}

\begin{figure}[htbp]
  \centering \image{00B5}
  \caption{总第~\ref{sec:00B5} 题} \label{fig:00B5}
\end{figure}

如图~\ref{fig:00B5},有$\triangle ABC$;延长$AB$至$D$、$BA$至$E$,使得
\[ \frac{AC}{BC} = \frac{AE}{BD} \]
连接$CD, CE$;作$\triangle ACD$与$\triangle BCE$的外心$O_1, O_2$;作直线$O_1O_2$分别与$AC, BC$交于$P, Q$。求证:$CP = CQ$。
\problabels{yellow/平面几何, green/证明题}

\emph{LTY提供的题目。}

\subsection{相交弦定理} \label{subsec:00B5-cc}

\begin{figure}[htbp]
  \centering \image{00B5-cc}
  \caption{方法~\ref{subsec:00B5-cc} 图} \label{fig:00B5-cc}
\end{figure}

如图~\ref{fig:00B5-cc},作$\triangle ACD$与$\triangle BCE$的外接圆$\odot O_1, \odot O_2$,两圆相交于$C, C'$;连接$C, C'$,交$AB$于点$K$。显然$CC' \perp PQ$。

由相交弦定理有
\begin{align*}
  & BK\cdot EK = CK\cdot C'K = AK \cdot DK \\
  \Rightarrow{}& BK\cdot(AE + AK) = AK\cdot(BD + BK) \\
  \Rightarrow{}& BK\cdot AE = AK\cdot BD \Rightarrow \frac{AK}{BK} = \frac{AE}{BD} = \frac{AC}{BC}
\end{align*}
由角平分线定理逆定理知$CK$平分$\angle PCQ$,而$CK \perp PQ$,由三线合一知$CP = CQ$。

证毕。


\prob{0078}{三角形垂心II}

\begin{figure}[htbp]
  \centering
  \image{0078}
  \caption{总第~\ref{sec:0078} 题图} \label{fig:0078}
\end{figure}

证明:三角形的垂心是以三角形的垂足为顶点的三角形的内心。
\problabels{yellow/平面几何, green/证明题}

\subsection{四点共圆} \label{subsec:0078-circ}

\begin{figure}[htbp]
  \centering
  \image{0078-circ}
  \caption{总第~\ref{sec:0078} 题解法“\nameref{subsec:0078-circ}”图}
  \label{fig:0078-circ}
\end{figure}

如图~\ref{fig:0078-circ},令$\triangle ABC$的垂心为$O$。

由$\angle OPB = \angle ORB = 90^\circ$知$B, P, O, R$四点共圆;由$\angle APB = \angle AQB = 90^\circ$知$A, B, P, Q$四点共圆。

将圆周角定理在两圆中应用,知$\angle APR = \angle ABQ, \angle APQ = \angle ABQ$,故$\angle APQ = \angle APR$。因此,$PO$平分$\angle QPR$。同理可证$QO$平分$\angle RQP$,$RO$平分$\angle PRQ$。

因此$O$是$\triangle PQR$的内心。原命题得证。

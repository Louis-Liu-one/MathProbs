
\prob{0042}{直线上的内心}

\begin{figure}[htbp]
  \centering
  \image{0042}
  \caption{0042:直线上的内心} \label{fig:0042}
\end{figure}

如图~\ref{fig:0042},在$\mathrm{Rt}\triangle ABC$中,$\angle ACB = 90^\circ$,$CD \perp AB$于$D$,$I$、$J$分别为$\triangle ACD$与$\triangle BCD$的内心。延长$IJ$,与$BC$交于点$K$,求$\angle IKC$的度数。
\problabels{yellow/平面几何, green/角度问题}

\ans{$\angle IKC = 45^\circ$}

\subsection{垂心} \label{subsec:0042-vert}

\begin{figure}[htbp]
  \centering
  \image{0042-vert}
  \caption{\nameref{subsec:0042-vert}:运用内心与垂心的性质求解。}
  \label{fig:0042-vert}
\end{figure}

如图~\ref{fig:0042-vert},连接$CI$、$CJ$;作射线$AI$,与$CJ$交于点$P$;作射线$BJ$,与$CI$交于点$Q$,与$AP$交于点$O$;作射线$CO$,与$IJ$交于点$R$。

由于$I$、$J$分别为$\triangle ACD$与$\triangle BCD$的内心,易知$AP$平分$\angle BAC$,$BQ$平分$\angle ABC$,$CI$平分$\angle ACD$,$CJ$平分$\angle BCD$。

又由内心的定义知,$O$是$\triangle ABC$的内心,故$CO$平分$\angle ACB$,即$\angle RCK = 45^\circ$。

\begin{align*}
  &\because   BQ\text{平分}\angle ABC \\
  &\therefore \angle CBD = 2\angle CBQ \\
  &\because   \angle CDB = 90^\circ \\
  &\therefore \angle BCD + \angle CBD = 90^\circ \\
  &\therefore \angle BCD = 90^\circ - 2\angle CBQ \\
  &\because   \angle ACB = 90^\circ \\
  &\therefore \angle ACD + \angle BCD = 90^\circ \\
  &\therefore \angle ACD = 2\angle CBQ \\
  &\because   CI\text{平分}\angle ACD \\
  &\therefore \angle ICD = \angle CBQ \\
  &\therefore \angle BCQ = 90^\circ - \angle CBQ \\
  &\therefore \angle BCQ + \angle CBQ = 90^\circ \\
  &\therefore BQ \perp CI, \text{同理可知}AP \perp CJ \\
  &\therefore O\text{是}\triangle ICJ\text{的垂心} \\
  &\therefore CR \perp IJ \\
  &\therefore \angle RCK + \angle IKC = 90^\circ \\
  &\because   \angle RCK = 45^\circ \\
  &\therefore \angle IKC = 45^\circ \\
\end{align*}

综上,$\angle IKC = 45^\circ$。

\subsection{解析几何} \label{subsec:0042-dec}

\begin{figure}[htbp]
  \centering
  \image{0042-dec}
  \caption{总第~\ref{sec:0042} 题解法“\nameref{subsec:0042-dec}”图}
  \label{fig:0042-dec}
\end{figure}

如图~\ref{fig:0042-dec},以$C$为原点,$CB$为$x$轴,$CA$为$y$轴建立平面直角坐标系$xCy$;连接$AI, BJ, CI, CJ$。不妨设$BC = a, AC = b, AB = c, \angle IAC = \alpha, \angle JBC = \beta$,则易知$\angle BAC = 2\alpha, \angle ABC = 2\beta, \angle ICA = 45^\circ - \alpha, \angle JCB = 45^\circ - \beta$。于是有
\begin{align}
  CI&: y = \frac{\cos(45^\circ - \alpha)}{\sin(45^\circ - \alpha)}x = \frac{\cos\alpha + \sin\alpha}{\cos\alpha - \sin\alpha}x \label{eq:0042-dec-CI} \\
  CJ&: y = \frac{\sin(45^\circ - \beta)}{\cos(45^\circ - \beta)}x = \frac{\cos\beta - \sin\beta}{\cos\beta + \sin\beta}x \label{eq:0042-dec-CJ} \\
  AI&: y = -\frac{\cos\alpha}{\sin\alpha}x + b \label{eq:0042-dec-AI} \\
  BJ&: y = -\frac{\sin\beta}{\cos\beta}(x - a) = -\frac{\sin\beta}{\cos\beta}x + \frac{\sin\beta}{\cos\beta}a \label{eq:0042-dec-BJ}
\end{align}

联立式~\ref{eq:0042-dec-AI}与式~\ref{eq:0042-dec-CI},解得
\begin{align}
  \left\{\begin{aligned}
    x_I &= b(\cos\alpha\sin\alpha - \sin^2\alpha) \\
    y_I &= b(\cos\alpha\sin\alpha + \sin^2\alpha) \\
  \end{aligned}\right. \label{eq:0042-dec-I}
\end{align}
同理,联立式~\ref{eq:0042-dec-BJ}与式~\ref{eq:0042-dec-CJ},解得
\begin{align}
  \left\{\begin{aligned}
    x_J &= a(\cos\beta\sin\beta + \sin^2\beta) \\
    y_J &= a(\cos\beta\sin\beta - \sin^2\beta) \\
  \end{aligned}\right. \label{eq:0042-dec-J}
\end{align}
于是有
\begin{align*}
  x_I - x_J ={}& b\cos\alpha\sin\alpha - b\sin^2\alpha \\
  &- a\cos\beta\sin\beta - a\sin^2\beta \\
  ={}& \frac12(b\sin2\alpha - a\sin2\beta) \\
  &- (b\sin^2\alpha + a\sin^2\beta) \\
\end{align*}
易知$b\sin2\alpha = a\sin2\beta = CD$,故
\[ x_I - x_J = -(b\sin^2\alpha + a\sin^2\beta) \]
同理得
\[ y_I - y_J = b\sin^2\alpha + a\sin^2\beta = -(x_I - x_J) \]

因此,直线$IJ$的斜率为
\[ k = \frac{y_I - y_J}{x_I - x_J} = \frac{-(x_I - x_J)}{x_I - x_J} = -1 \]
故知$\angle IKC = 45^\circ$。

\paragraph{或} 由
\begin{align*}
  a &= c\sin2\alpha = 2c\sin\alpha\cos\alpha \\
  b &= c\sin2\beta = 2c\sin\beta\cos\beta
\end{align*}
将式~\ref{eq:0042-dec-I}、式~\ref{eq:0042-dec-J}改写为
\begin{align*}
  \left\{\begin{aligned}
    x_I &= 2c\sin\beta\cos\beta(\cos\alpha\sin\alpha - \sin^2\alpha) \\
    y_I &= 2c\sin\beta\cos\beta(\cos\alpha\sin\alpha + \sin^2\alpha) \\
  \end{aligned}\right. \\
  \left\{\begin{aligned}
    x_J &= 2c\sin\alpha\cos\alpha(\cos\beta\sin\beta + \sin^2\beta) \\
    y_J &= 2c\sin\alpha\cos\alpha(\cos\beta\sin\beta - \sin^2\beta) \\
  \end{aligned}\right.
\end{align*}
于是有
\begin{align*}
  x_I - x_J &= -2c\sin\alpha\sin\beta(\sin\alpha\cos\beta + \cos\alpha\sin\beta) \\
  y_I - y_J &= 2c\sin\alpha\sin\beta(\sin\alpha\cos\beta + \cos\alpha\sin\beta)
\end{align*}
后易知$\angle IKC = 45^\circ$。

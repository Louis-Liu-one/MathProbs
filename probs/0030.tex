
\prob{0030}{等差数列不等式}

若整数$a_1, a_2, a_3, \dots, a_{15}$可以组成等差数列,且满足

\begin{align*}
  a_1 < a_2 < \dots < a_{15} \\
  1 \le a_1 \le 10 \\
  13 \le a_2 \le 20 \\
  241 \le a_{15} \le 250 \\
\end{align*}

求$a_{14}$。
\problabels{yellow/代数, green/代数求值问题}

\ans{$a_{14} = 224$}

\subsection{数论分析}

基本思路:先求出$a_1$,然后求出等差数列的公差$d$,即可求出$a_{14}$。

设等差数列$a_1, a_2, a_3, \dots, a_{15}$的公差为$d$,则可知

\begin{align*}
  a_1 &= a_1 \\
  a_2 &= a_1 + d \\
  \dots &= \dots \\
  a_{14} &= a_1 + 13d \\
  a_{15} &= a_1 + 14d \\
\end{align*}

由$a_{15} = a_1 + 14d$知

\[ 241 \le a_1 + 14d \le 250 \]

又由$1 \le a_1 \le 10$可知当$a_1 = 10$时$14d$取最小值$241 - 10 = 231$,而当$a_1 = 1$时$14d$取最大值$250 - 1 = 249$。因此,

\[ 231 \le 14d \le 249 \]

由于$d$是整数,而在区间$[231, 249]$中唯一的$14$的倍数是$238 = 14\times17$,因此

\begin{align*}
  d &= 17 \\
  a_2 &= a_1 + 17 \\
  a_{14} &= a_1 + 221 \\
  a_{15} &= a_1 + 238 \\
\end{align*}

由$a_{15} \ge 241$知

\begin{align*}
  a_1 + 238 &\ge 241 \\
  a_1 &\ge 3 \\
\end{align*}

又由$a_2 \le 20$知

\begin{align*}
  a_1 + 17 &\le 20 \\
  a_1 &\le 3 \\
\end{align*}

由

\begin{align*}
  \begin{cases}
    a_1 \ge 3 \\
    a_1 \le 3 \\
  \end{cases}
\end{align*}

可知$a_1 = 3$。因此$a_{14} = 3 + 221 = 224$。

综上,$a_{14} = 224$。

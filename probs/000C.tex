
\prob{000C}{单位等边三角形}

\begin{figure}[htbp]
  \centering
  \image{000C}
  \caption{000C:单位等边三角形} \label{fig:000C}
\end{figure}

如图~\ref{fig:000C},$\triangle ABC$是等边三角形,$AB = AC = BC = 1$,$P$在线段$AB$上,$Q$在$BC$的延长线上且$AP = CQ$,$PE \perp AC$于$E$,求$DE$的长。
\problabels{yellow/平面几何, green/长度问题}

\ans{$DE = \sfrac12$}

\subsection{构造8形全等} \label{subsec:000C-8eq}

\begin{figure}[htbp]
  \centering
  \image{000C-8eq}
  \caption{\nameref{subsec:000C-8eq}:通过构造8形全等和一个等边三角形求解。}
  \label{fig:000C-8eq}
\end{figure}

基本思路:通过作一个等边三角形构造8形全等,进而求出$DE$的长。

如图~\ref{fig:000C-8eq},作$PC' \parallel BQ$,交$AC$于点$C'$。

\begin{align*}
  &\because   PC' \parallel BQ \\
  &\therefore \angle APC' = \angle B, \angle AC'P = \angle ACB, \\
  &\mathalignsep \angle C'PD = \angle Q \\
  &\because   \triangle ABC\text{是等边三角形} \\
  &\therefore \angle B = \angle ACB = 60^\circ \\
  &\therefore \angle APC' = \angle AC'P = 60^\circ \\
  &\therefore \triangle APC'\text{是等边三角形} \\
  &\therefore \angle A = \angle AC'P, AP = C'P \\
  &\because   PE \perp AC \\
  &\therefore \angle AEP = \angle C'EP = 90^\circ \\
  &\because   \text{在}\triangle AEP\text{与}\triangle C'EP\text{中} \\
  &\mathalignsep \begin{cases}
    \angle A = \angle PC'E \\
    \angle AEP = \angle C'EP \\
    EP = EP \\
  \end{cases} \\
  &\therefore \triangle AEP \cong \triangle C'EP \\
  &\therefore AE = C'E \\
  \text{又}&\because AP = CQ \\
  &\therefore C'P = CQ \\
  &\because   \text{在}\triangle C'DP\text{与}\triangle CDQ\text{中} \\
  &\mathalignsep \begin{cases}
    \angle C'DP = \angle CDQ \\
    \angle C'PD = \angle Q \\
    C'P = CQ \\
  \end{cases} \\
  &\therefore \triangle C'DP \cong \triangle CDQ \\
  &\therefore CD = C'D \\
  &\because   DE = C'D + C'E \\
  &\therefore DE = CD + AE \\
  &\because   DE + CD + AE = AC = 1 \\
  &\therefore 2DE = 1 \\
  &\therefore DE = \frac12 \\
\end{align*}

综上,$DE$的长度为$\sfrac12$。

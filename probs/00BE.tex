
\prob{00BE}{三角函数互比}

对于角$\theta$,求
\[ f(\theta) = \frac{\sin\theta + 2}{\cos\theta + 3} \]
的取值范围。
\problabels{yellow/代数, green/取值范围问题}

\ans{\[
  \frac{3 - \sqrt3}4 \le f(\theta) \le \frac{3 + \sqrt3}4
\]}

\emph{JYH提供的题目。}

\subsection{数形结合} \label{subsec:00BE-d}

\begin{figure}[htbp]
  \centering \image{00BE-d}
  \caption{方法~\ref{subsec:00BE-d} 图}
  \label{fig:00BE-d}
\end{figure}

如图~\ref{fig:00BE-d},在平面直角坐标系中,定义点$T(\cos\theta, \sin\theta)$,其恒在单位$\odot O$上;令$P(-3, -2)$,则$f(\theta)$是直线$PT$的斜率。

显然直线$PT$的斜率最大或最小时,$PT$切$\odot O$于$T$,$T$在图中$T_1$或$T_2$处。设此时斜率$f(\theta) = k$,则
\[ PT: y = k(x + 3) - 2 = kx + 3k - 2 \]
而单位$\odot O$之方程为$x^2 + y^2 = 1$,联立得
\[ \left\{\begin{aligned}
  & y = kx + 3k - 2 \\ & x^2 + y^2 = 1
\end{aligned}\right. \Rightarrow x^2 + (kx + 3k - 2)^2 = 1 \]
展开得
\[ \left(k^2 + 1\right)x^2 + 2k(3k - 2)x + \left(9k^2 - 12k + 3\right) = 0 \]
由于$PT$切于$\odot O$,故此二次方程有两相同根,故
\begin{align*}
  \Delta = 0 \Rightarrow{}& k^2(3k - 2)^2 = \left(k^2 + 1\right)\left(9k^2 - 12k + 3\right) \\
  \Rightarrow{}& k^2\left(9k^2 - 12k + 4\right) \\
    &= \left(k^2 + 1\right)\left(9k^2 - 12k + 3\right) \\
  \Rightarrow{}& k^2 = 9k^2 - 12k + 3 \Rightarrow k_{1, 2} = \frac{3 \pm\sqrt3}4
\end{align*}
故$f(\theta)$最大值为$\left(3 + \sqrt3\right)/4$,$f(\theta)$最小值为$\left(3 - \sqrt3\right)/4$。因此,
\[ \frac{3 - \sqrt3}4 \le f(\theta) \le \frac{3 + \sqrt3}4 \]

\subsection{求导}

当$f(\theta)$取得极值时,
\begin{align*}
  \frac{\dif f}{\dif\theta} ={}& \cos\theta\cdot\frac1{\cos\theta + 3} \\
    &+ (\sin\theta + 2)\cdot(\cos\theta + 3)^{-2}\cdot\sin\theta \\
  ={}& \frac{\cos\theta}{\cos\theta + 3} + \frac{\sin\theta(\sin\theta + 2)}{(\cos\theta + 3)^2} \\
  ={}& \frac{\cos\theta(\cos\theta + 3) + \sin\theta(\sin\theta + 2)}{(\cos\theta + 3)^2} \\
  ={}& \frac{3\cos\theta + 2\sin\theta + 1}{(\cos\theta + 3)^2} = 0
\end{align*}
令$\cos\theta = x, \sin\theta = y$,则
\[ \left\{\begin{aligned}
  x^2 + y^2 &= 1 \\ 3x + 2y &= -1
\end{aligned}\right. \Rightarrow
\left\{\begin{aligned}
  & 4x^2 + 4y^2 = 4 \\ & 4y^2 = (3x + 1)^2
\end{aligned}\right. \]
即$4x^2 + (3x + 1)^2 = 4$,解得
\[ x = \frac{-3 \pm4\sqrt3}{13}, y = \frac{-2 \mp6\sqrt3}{13} \]
此时
\begin{align*}
  f(\theta) &= \frac{y + 2}{x + 3} = \frac{24 \mp6\sqrt3}{36 \pm4\sqrt3} = \frac{12 \mp3\sqrt3}{2\sqrt3\left(3\sqrt3 \pm1\right)} \\
  &= \frac{\sqrt3\left(12 \mp3\sqrt3\right)\left(3\sqrt3 \mp1\right)}{6\left(3\sqrt3 \pm1\right)\left(3\sqrt3 \mp1\right)} \\
  &= \frac{\sqrt3\left(36\sqrt3 \mp27 \mp12 + 3\sqrt3\right)}{6\cdot26} \\
  &= \frac{\sqrt3\left(39\sqrt3 \mp39\right)}{12\cdot13} = \frac{3 \mp\sqrt3}4
\end{align*}
因此,
\[ \frac{3 - \sqrt3}4 \le f(\theta) \le \frac{3 + \sqrt3}4 \]

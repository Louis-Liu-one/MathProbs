
\prob{000F}{积和最大值}

已知$p, r, s$是常数,$m, n$为正数,$m^2 + n^2 = p$,求$rm + sn$的最大值。
\problabels{yellow/代数, green/最值问题}

\ans{$rm + sn$的最大值为$\sqrt{p(r^2 + s^2)}$。}

\subsection{对偶式}

基本思路:运用对偶式拼出$m^2 + n^2$。

求$rm + sn$与其对偶式$sm - rn$的平方和,可得

\begin{align*}
  &\phantom{=} (rm + sn)^2 + (sm - rn)^2 \\
  &= r^2m^2 + 2rsmn + s^2n^2 + s^2m^2 - 2rsmn + r^2n^2 \\
  &= (r^2 + s^2)(m^2 + n^2) \\
  &= p(r^2 + s^2) \\
\end{align*}

当且仅当$sm = rn$时,$rm + sn$取最大值,即

\begin{align*}
  rm + sn &= \sqrt{p(r^2 + s^2) - (sm - rn)^2} \\
  &= \sqrt{p(r^2 + s^2)}
\end{align*}

综上,$rm + sn$的最大值为$\sqrt{p(r^2 + s^2)}$。

\subsection{求导}

基本思路:通过将函数求导解出最大值。

由$m^2 + n^2 = p$与$m > 0, n > 0$可得$n = \sqrt{p - m^2}$,代入$rm + sn$得$rm + s\sqrt{p - m^2}$。

令$y = rm + s\sqrt{p - m^2}$,则

\begin{align*}
  \frac{\dif y}{\dif m} &= r + s\cdot(p - m^2)^{-\frac12}\cdot(-2m) \\
  &= r - \frac{sm}{\sqrt{1 - m^2}} \\
\end{align*}

当$\sfrac{\dif y}{\dif m} = 0$时$rm + sn$取最值,求解方程并取正值得

\[ m = \frac{r\sqrt p}{\sqrt{r^2 + s^2}} \]

代入$n = \sqrt{p - m^2}$得

\[ n = \frac{s\sqrt p}{\sqrt{r^2 + s^2}} \]

由此可得

\begin{align*}
  rm + sn &= r\frac{r\sqrt p}{\sqrt{r^2 + s^2}} + s\frac{s\sqrt p}{\sqrt{r^2 + s^2}} \\
  &= \frac{\sqrt p(r^2 + s^2)}{\sqrt{r^2 + s^2}} \\
  &= \sqrt{p(r^2 + s^2)} \\
\end{align*}

综上,$rm + sn$的最大值为$\sqrt{p(r^2 + s^2)}$。


\prob{0028}{相交弦定理}

\begin{figure}[htbp]
  \centering
  \image{0028}
  \caption{0028:相交弦定理} \label{fig:0028}
\end{figure}

证明相交弦定理:经过圆内一点引两条弦,各弦被这点所分成的两线段的积相等。
\problabels{yellow/平面几何, green/证明题}

\subsection{相似三角形} \label{subsec:0028-sim}

\begin{figure}[htbp]
  \centering
  \image{0028-sim}
  \caption{\nameref{subsec:0028-sim}:通过圆周角定理找出相似三角形。}
  \label{fig:0028-sim}
\end{figure}

基本思路:运用圆周角定理找出相似三角形,证明命题。

如图~\ref{fig:0028-sim},连接$AD$、$BC$。

\begin{align*}
  &\because   \text{据圆周角定理\footnotemark 得,}\angle PAD = \angle PCB, \\
  &\mathalignsep \angle APD = \angle CPB \\
  &\therefore \triangle APD \sim \triangle CPB \\
  &\therefore PA:PD = PC:PB \\
  &\therefore PA\cdot PB = PC\cdot PD \\
\end{align*}

\footnotetext{圆周角定理的证明参见第~\ref{sec:0025} 题。}

证毕。


\prob{003F}{矩形翻折}

\begin{figure}[htbp]
  \centering
  \image{003F}
  \caption{003F:矩形翻折} \label{fig:003F}
\end{figure}

如图~\ref{fig:003F},在矩形$ABCD$中,$AB = 4$,$BC = 3$,$P$为边$BC$上一点,将$\triangle CDP$沿$DP$折叠,点$C$落在点$E$处,$DE, PE$分别交$AB$于点$F, G$。若$EG = BG$,求$BF$的长。
\problabels{yellow/平面几何, green/长度问题}

\ans{$BF = \sfrac{12}5$}

\subsection{勾股定理} \label{subsec:003F-pyth}

基本思路:设$BF = x$,然后在$\triangle ADF$中应用勾股定理求解。

设$BF = x$。

\begin{align*}
  &\because   \triangle EDP\text{由}\triangle CDP\text{折叠而来} \\
  &\therefore \angle C = \angle E, PC = PE, CD = DE \\
  &\because   \text{四边形}ABCD\text{为矩形} \\
  &\therefore \angle A = \angle B = \angle C = 90^\circ, \\
  &\mathalignsep AB = CD, BC = AD \\
  &\therefore \angle B = \angle E \\
  &\because   \text{在}\triangle EFG\text{与}\triangle BPG\text{中} \\
  &\mathalignsep \begin{cases}
    \angle E = \angle B \\
    EG = BG \\
    \angle EGF = \angle BGP \\
  \end{cases} \\
  &\therefore \triangle EFG \cong \triangle BPG \\
  &\therefore FG = PG, EF = BP \\
  &\because   BF = BG + FG \\
  &\therefore BF = EG + PG = PE \\
  &\because   BF = x \\
  &\therefore BF = PE = PC = x \\
  &\because   BC = 3 \\
  &\therefore AD = 3, BP = 3 - x \\
  &\therefore EF = 3 - x \\
  &\because   AB = 4 \\
  &\therefore CD = 4, AF = AB - BF = 4 - x \\
  &\therefore DE = 4 \\
  &\therefore DF = DE - EF = x + 1 \\
  \text{又}&\because \angle A = 90^\circ \\
  &\therefore AD^2 + AF^2 = DF^2 \\
  &\therefore 3^2 + (4 - x)^2 = (x + 1)^2 \\
  &\therefore x = \frac{12}5 \\
  &\therefore BF = \frac{12}5 \\
\end{align*}

综上,$BF$的长为$\sfrac{12}5$。

\subsection{面积法}

\emph{YCY提供的方法。}

基本思路:通过比较梯形$BCDF$与四边形$CDEP$的面积求解。

根据解法\nameref{subsec:003F-pyth}知,$BF = PC$,设$BF = PC = x$,则易知$S_{\triangle CDP} = 2x$,$S_{BCDF} = \sfrac32(x + 4) = \sfrac32x + 6$。

又根据解法\nameref{subsec:003F-pyth}知

\[ \triangle EFG \cong \triangle BPG \]

因此

\begin{align*}
  S_{\triangle EFG} &= S_{\triangle BPG} \\
  S_{\triangle EFG} + S_{CDFP} &= S_{\triangle BPG} + S_{CDFP} \\
  S_{CDEP} &= S_{BCDF} \\
  2S_{\triangle CDP} &= S_{BCDF} \\
  4x &= \frac32x + 6 \\
  x &= \frac{12}5 \\
  BF &= \frac{12}5 \\
\end{align*}

综上,$BF$的长为$\sfrac{12}5$。

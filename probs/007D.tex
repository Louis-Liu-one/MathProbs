
\prob{007D}{主定理}

证明主定理:若对常数$a > 0, b > 1, d \ge0$,有
\[ T(n) = aT\left(\frac nb\right) + O\left(n^d\right) \]
成立,则
\[ T(n) = \begin{cases}
  O\left(n^d\right), & a < b^d \\
  O\left(n^d\log n\right), & a = b^d \\
  O\left(n^{\log_ba}\right), & a > b^d \\
\end{cases} \]
\problabels{yellow/算法理论, green/证明题}

\subsection{几何级数}

由条件知,每个问题被分成$a$个子问题,每个子问题的规模是原问题的$\sfrac1b$。不妨考虑分支因子为$a$的递归树,在第$k$层上共$a^k$个子问题,每个子问题规模为$\sfrac n{b^k}$。故规模为1的子问题(即树叶)在第$\log_bn$层上,即递归树高度为$\log_bn$。

考虑在第$k$层上的一个子问题的耗时为
\[ O\left(\left(\frac n{b^k}\right)^d\right) = \frac1{b^{kd}}O\left(n^d\right) \]
共$a^k$个子问题,故总耗时为
\[ \frac{a^k}{b^{kd}}O\left(n^d\right) = \left(\frac a{b^d}\right)^kO\left(n^d\right) \]
由此可知总耗时为
\[ T(n) = O\left(n^d\right)\sum_{k = 0}^{\log_bn}\left(\frac a{b^d}\right)^k \]
可见等号右侧是一个几何级数,公比为$q = \sfrac a{b^d}$。对$q$进行分情况讨论。

若$q < 1$,即$a < b^d$,此时几何级数是递减的,从而可用其首项表示。而其首项为1,故
\[ T(n) = O\left(n^d\right) \]

若$q = 1$,即$a = b^d$,此时级数的每项都是1,共$\log_bn$项,故
\[ T(n) = O\left(n^d\right)\log_bn = O\left(n^d\log n\right) \]

若$q > 1$,即$a > b^d$,此时几何级数是递增的,从而可用其末项表示。而其末项为
\[ \left(\frac a{b^d}\right)^{\log_bn} = \frac{a^{\log_bn}}{\left(b^{\log_bn}\right)^d} = \frac{\left(b^{\log_bn}\right)^{\log_ba}}{n^d} = \frac{n^{\log_ba}}{n^d} \]
故
\[ T(n) = \frac{n^{\log_ba}}{n^d}O\left(n^d\right) = O\left(n^{\log_ba}\right) \]

即为
\[ T(n) = \begin{cases}
  O\left(n^d\right), & a < b^d \\
  O\left(n^d\log n\right), & a = b^d \\
  O\left(n^{\log_ba}\right), & a > b^d \\
\end{cases} \]
命题得证。

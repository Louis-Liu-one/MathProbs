
\prob{008D}{两类五位数}

存在如下两类五位数:

一类数:各位数字之和为36的偶数;

二类数:各位数字之和为38的奇数;

求哪类数更多。
\problabels{yellow/数论}

\ans{一类数更多。}

\subsection{唯一对应}

\begin{lemma} \label{lemma:008D-20}
  二类数的数位中没有0。
\end{lemma}

\begin{proof}
  若二类数的数位中存在1个0,则另外4个数位和的最大值为$4\cdot9 = 36 < 38$,故二类数的数位中没有0。
\end{proof}

对于任意一个二类数,由引理~\ref{lemma:008D-20} 知,可以将其个位减去1,另外4位中任选一位减去1,得到一个新数,显然该数为一类数。于是可知,每一个二类数都可以唯一对应一个一类数,故一类数不会比二类数少。

在如上的对应关系中,某些一类数可以通过将个位加1,另外4位中选一位加1对应二类数。由上段可知,部分一类数可以通过这种对应方式对应全部二类数。然而,一类数99990显然不能通过此方式对应某个二类数,故99990不在上述对应关系中的“部分一类数”中,于是可知一类数更多。

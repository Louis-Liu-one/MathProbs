
\prob{007E}{从不等式到函数}

若$f$是某定义域和值域均为正整数的函数,且对于任意正整数$n$,满足
\[ f(n + 1) > f(f(n)) \]
求$f(n)$。
\problabels{yellow/数论}

\ans{$f(n) = n$}

\subsection{利用正整数的性质}

\begin{lemma} \label{lemma:007E-mnfmn}
  对于任意正整数$m \ge n$,$f(m) \ge n$。
\end{lemma}

\begin{proof}
  对$n$使用归纳法。

  对于$n = 1$,由于$f(m)$是正整数,显然有$f(m) \ge n$;

  对于$n > 1$,假设对于所有$m \ge n$,有$f(m) \ge n$,即要证明对于所有$m \ge n + 1$,有$f(m) \ge n + 1$。由题目条件知$f(m) > f(f(m - 1))$。而$m - 1 \ge n$,故由归纳假设有$f(m - 1) \ge n$,又由归纳假设有$f(f(m - 1)) \ge n$,于是$f(m) > n$,即$f(m) \ge n + 1$。
\end{proof}

考虑引理~\ref{lemma:007E-mnfmn} 中$m = n$时的特殊情形,有$f(n) \ge n$,而由此又有$f(f(n)) \ge f(n)$,故
\[ f(n + 1) > f(f(n)) \ge f(n) \]
即$f(n + 1) > f(n)$,于是$f$是单调增的。因此当对于正整数$a, b$,有$f(a) > f(b)$,则必然有$a > b$,于是式$f(n + 1) > f(f(n))$可改写为$n + 1 > f(n)$,即$f(n) \le n$。而$f(n) \ge n$,故$f(n) = n$。


\prob{0049}{三角形与半角}

\begin{figure}[htbp]
  \centering
  \image{0049}
  \caption{0049:三角形与半角} \label{fig:0049}
\end{figure}

\emph{LHQ提供的题目。}

如图~\ref{fig:0049},在$\triangle ABC$中,$\angle BAC = 60^\circ$,$BC = \sqrt7$。若$D$为线段$BC$上一点且满足$BD = 1$且$\angle ADC = 30^\circ$,求$AD$的长。
\problabels{yellow/平面几何, green/长度问题}

\ans{$AD = \sfrac13(\sqrt3 + \sqrt{21})$}

\subsection{全等三角形} \label{subsec:0049-eqtri}

\begin{figure}[htbp]
  \centering
  \image{0049-eqtri}
  \caption{\nameref{subsec:0049-eqtri}:利用两组全等三角形求解。}
  \label{fig:0049-eqtri}
\end{figure}

\emph{LHQ提供的方法。}

基本思路:利用两组全等三角形,然后通过一系列作垂线和勾股定理求解。

如图~\ref{fig:0049-eqtri},延长射线$BC$,将$\triangle BAD$旋转至$\triangle B'AD'$处,使得$D'$在$BC$上,连接$B'C$,作$B'P \perp BC$于$P$,$AQ \perp BC$于$Q$。由旋转的性质易知$\triangle BAD \cong \triangle B'AD'$。

\begin{align*}
  &\because   \triangle BAD \cong \triangle B'AD' \\
  &\therefore BD = B'D', AB = AB', AD = AD', \\
  &\phantom{\because} \angle ADB = \angle AD'B', \angle BAD = \angle B'AD' \\
  &\therefore \angle ADD' = \angle AD'D \\
  &\because   \angle ADD' = 30^\circ \\
  &\therefore \angle AD'D = 30^\circ, \angle ADB = 150^\circ \\
  &\therefore \angle DAD' = 120^\circ, \angle AD'B' = 150^\circ \\
  &\therefore \angle B'AD = 120^\circ - \angle B'AD' \\
  &\because   \angle BAC = 60^\circ \\
  &\therefore \angle CAD = 60^\circ - \angle BAD = 60^\circ - \angle B'AD' \\
  &\therefore \angle B'AC = \angle B'AD - \angle CAD = 60^\circ \\
  &\therefore \angle BAC = \angle B'AC \\
  &\because   \text{在}\triangle BAC\text{与}\triangle B'AC\text{中} \\
  &\mathalignsep \begin{cases}
    AB = AB' \\
    \angle BAC = \angle B'AC \\
    AC = AC \\
  \end{cases} \\
  &\therefore \triangle BAC \cong \triangle B'AC \\
  &\therefore BC = B'C \\
  &\because   BC = \sqrt7 \\
  &\therefore B'C = \sqrt7 \\
  \text{又}&\because \angle AD'B' = 150^\circ, \angle AD'C = 30^\circ \\
  &\therefore \angle CD'B' = 120^\circ \\
  &\therefore \angle B'D'P = 60^\circ \\
  &\because   B'P \perp BC \\
  &\therefore \angle P = 90^\circ \\
  &\therefore PD' = \frac12B'D', PB' = \frac{\sqrt3}2B'D', \\
  &\phantom{\because} PB'^2 + PC^2 = B'C^2 \\
  &\because   BD = 1 \\
  &\therefore B'D' = 1, PD' = \frac12, PB' = \frac{\sqrt3}2 \\
  &\therefore PC = \sqrt{(\sqrt7)^2 - \left(\frac{\sqrt3}2\right)^2} = \frac52 \\
  &\therefore CD' = PC - PD' = 2 \\
  &\because   CD = BC - BD = \sqrt7 - 1 \\
  &\therefore DD' = 1 + \sqrt7 \\
  \text{又}&\because \triangle DAD'\text{是等腰三角形且}AQ \perp DD' \\
  &\therefore DQ = \frac12DD' = \frac12(1 + \sqrt7) \\
  &\because   \angle ADC = 30^\circ \\
  &\therefore AD = \frac2{\sqrt3}DQ \\
  &\therefore AD = \frac{\sqrt3}3(1 + \sqrt7) = \frac13(\sqrt3 + \sqrt{21}) \\
\end{align*}

综上,$AD$的长为$\sfrac13(\sqrt3 + \sqrt{21})$。\footnotemark

\subsection{解析几何} \label{subsec:0049-dec}

\begin{figure}[htbp]
  \centering
  \image{0049-dec}
  \caption{\nameref{subsec:0049-dec}:通过建立平面直角坐标系,然后通过正切求直线夹角,最后列方程求解。}
  \label{fig:0049-dec}
\end{figure}

基本思路:建立平面直角坐标系,然后将$\angle BAC$用$AB$与$AC$的斜率表示,最后由$\angle BAC = 60^\circ$即可列方程求解。

如图~\ref{fig:0049-dec},以$D$为原点,$DC$为$x$轴建立平面直角坐标系$xDy$,则易知$D(0, 0)$、$B(-1, 0)$、$C(\sqrt7 - 1, 0)$。

由$\angle ADC = 30^\circ$易知,$A$在直线$y = \sfrac{\sqrt3}3x$上,不妨设$A(\sqrt3x, x)$。设直线$AB$的斜率为$k_1$,直线$AC$的斜率为$k_2$,$\alpha = \angle ABC$,$\beta$为如图所示$\angle ACB$的补角。

由于$A(\sqrt3x, x)$、$B(-1, 0)$、$C(\sqrt7 - 1, 0)$,根据斜率公式知

\begin{align*}
  \tan\alpha = k_1 &= \frac{x - 0}{\sqrt3x - (-1)} \\
  &= \frac x{\sqrt3x + 1} \\
  \tan\beta = k_2 &= \frac{x - 0}{\sqrt3x - (\sqrt7 - 1)} \\
  &= \frac x{\sqrt3x + 1 - \sqrt7} \\
\end{align*}

由正切的和差化积公式知

\begin{align*}
  \tan(\beta - \alpha) &= \frac{\tan\beta - \tan\alpha}{1 + \tan\alpha\tan\beta} \\
  &= \frac{k_2 - k_1}{1 + k_1k_2} \\
  &= \frac{x\left((\sqrt3x + 1) - (\sqrt3x + 1 - \sqrt7)\right)}{\left(\sqrt3x + 1\right)\left(\sqrt3x + 1 - \sqrt7\right) + x^2} \\
  &= \frac{\sqrt7x}{4x^2 + \sqrt3(2 - \sqrt7)x + (1 - \sqrt7)} \\
\end{align*}

又由于$\beta - \alpha$显然等于$60^\circ$,故$\tan(\beta - \alpha) = \tan 60^\circ = \sqrt3$,因此

\begin{align*}
  \frac{\sqrt7x}{4x^2 + \sqrt3(2 - \sqrt7)x + (1 - \sqrt7)} &= \sqrt3 \\
  4\sqrt3x^2 + 3(2 - \sqrt7)x + \sqrt3(1 - \sqrt7) &= \sqrt7x \\
\end{align*}

得到一个关于$x$的一元二次方程。解方程并舍去负值得

\[ x = \frac16(\sqrt3 + \sqrt{21}) \]

由此可知$y_A = \sfrac16(\sqrt3 + \sqrt{21})$。又由$\angle ADC = 30^\circ$知,$AD$的长为$2y_A = \sfrac13(\sqrt3 + \sqrt{21})$。

综上,$AD$的长为$\sfrac13(\sqrt3 + \sqrt{21})$。

\footnotetext{步骤中作垂线应用勾股定理的部分若使用余弦定理等则可省略。}

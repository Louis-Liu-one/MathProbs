
\prob{0018}{Mersenne完全数}

对于任意Mersenne素数$M_p$,试证明:$2^{p - 1}M_p$都是完全数。\footnotemark
\problabels{yellow/数论, green/证明题}

\subsection{枚举真因子}

基本思路:枚举$2^{p - 1}M_p$的真因子,再证明这些真因子之和就是$2^{p - 1}M_p$。

$2^{p - 1}M_p$的真因子中无法被$M_p$整除的有$2^0, 2^1, 2^2, \dots, 2^{p - 1}$,可以被$M_p$整除的有$2^0M_p, 2^1M_p, \dots, 2^{p - 2}M_p$。因为某数的真因子不包括其本身,所以这里没有$2^{p - 1}M_p$。

将这些数求和后可得

\begin{align*}
  & 2^0 + 2^1 + \dots + 2^{p - 1} \\
  & + M_p(2^0 + 2^1 + \dots + 2^{p - 2}) \\
  ={}& 2^p - 1 + (2^p - 1)(2^{p - 1} - 1) \\
  ={}& (2^p - 1)(2^{p - 1} - 1 + 1) \\
  ={}& 2^{p - 1}M_p \\
\end{align*}

就是$2^{p - 1}M_p$本身。因此,$2^{p - 1}M_p$是完全数。

证毕。

\footnotetext{Mersenne素数是指形如$M_p = 2^p - 1$的素数,其中$p$是素数;完全数是指等于自身真因子之和的整数,例如28的真因子包括14、7、4、2、1,其和正好是28。}


\prob{002F}{数字推盘游戏}

\begin{figure}[htbp]
  \centering
  \image{002F}
  \caption{002F:数字推盘游戏} \label{fig:002F}
\end{figure}

图~\ref{fig:002F} 是一种数字推盘游戏。推板上有15个滑块,分别写有1~15,还有一个空格在右下角,滑块间没有空隙。你可以把空格四个方向的滑块滑到空格上,即将空格四个方向的滑块之一与空格交换位置,但是不能从推板上拿出滑块。游戏开始时,14与15不在正确的位置(图~\ref{fig:002F}(上)),你的目标是把14和15交换,这样它们就都在正确的位置上了(图~\ref{fig:002F}(下))。

在遵守游戏规则的前提下,交换14和15可能吗?若可能,说明一种滑动的方式;若不可能,说明理由。
\problabels{green/数学谜题}

\ans{不可能;理由略。}

\subsection{奇偶性分析}

基本思路:通过奇偶性分析得知某量的奇偶性不变,又通过游戏开始与游戏结束时该量的奇偶性变化得出结论。

不可能。理由如下:

设空格所在行数为$p$;排列的反序数为$q$。若在一个排列中,一个较大的数排在一个较小的数前面,则构成一个反序;一个排列的反序数就是这个排列中反序的数量,例如$2,5,3,4,1$的反序有$(2,1)$、$(3,1)$、$(4,1)$、$(5,3)$、$(5,4)$、$(5,1)$,因此该排列的反序数为6。将推盘上的滑块从上到下、从左到右排成一个数列,就是一个排列,$q$为该排列的反序数,例如游戏开始时的排列为$1,2,\dots,13,15,14$,其反序数为1。

若将某一滑块左右移动,与空格交换,则既没有改变空格的行数$p$,也没有改变排列,因此$q$亦不变,所以$p + q$不变,奇偶性亦不变。

若将某一滑块上下移动,则既改变了$p$,也改变了$q$。由于空格因为移动向上或向下移动了一格,所以$p$变成了$p \pm1$,因此其奇偶性改变。由于该滑块上下移动,因此在排列中跨越了3个滑块。若该滑块与这3个滑块构成3个反序或没有构成反序,则移动后该滑块与这3个滑块没有构成反序或构成3个反序,因此$q$变为$q \pm3$;若该滑块与这3个滑块构成1个或2个反序,则移动后该滑块与另外2个滑块构成2个反序或与另外一个滑块构成1个反序,即$q$变为$q \pm1$。因此上下移动后,$q$的奇偶性亦改变。由于$p$的奇偶性亦改变,所以$p + q$的奇偶性仍不变。

因此不论滑块如何移动,$p + q$的奇偶性不变。游戏开始时,$p = 4$、$q = 1$,$p + q = 5$,是奇数;而游戏目标中,$p = 4$、$q = 0$,$p + q = 4$是偶数,所以从游戏开始的情况起,不论如何移动滑块,都无法达到游戏目标的情况。

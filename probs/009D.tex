
\prob{009D}{求等腰部分底}

\begin{figure}[htbp]
  \centering
  \image{009D}
  \caption{总第~\ref{sec:009D} 题图} \label{fig:009D}
\end{figure}

如图~\ref{fig:009D},在等腰$\triangle ABC$中,$AB = AC$,$D, E$分别是边$BC, AC$上的点,连接$AD, BE$交于点$P$,满足$\angle ADB = \angle AEB = 60^\circ$。若$AP = 3, CD = 5$,求$BD$。
\problabels{yellow/平面几何, green/长度问题}

\ans{$BD = 13$}

\subsection{向外证全等} \label{subsec:009D-oeq}

\begin{figure}[htbp]
  \centering
  \image{009D-oeq}
  \caption{解法~\ref{subsec:009D-oeq} 图} \label{fig:009D-oeq}
\end{figure}

如图~\ref{fig:009D-oeq},以$CD$为边向下作等边$\triangle CDQ$,易知$A, D, Q$共线。

由$\angle ADB = \angle AEB, \angle APE = \angle BPD$知$\angle CAQ = \angle PBD$。同时有$\angle BDP = \angle Q = 60^\circ$。

设$\angle APB = \alpha$,则易知$\angle ABC = \angle ACB = 120^\circ - \alpha$。由$\angle ADB = 60^\circ$知$\angle PAB = \alpha = \angle APB$,于是$AC = AB = BP$。

\begin{align*}
  \because  {}& \text{在$\triangle CAQ$与$\triangle PBD$中,} \\
  & \left\{ \begin{aligned}
    \angle Q &= \angle PDB \\
    \angle CAQ &= \angle PBD \\
    CA &= PB
  \end{aligned} \right. \\
  \therefore{}& \triangle CAQ \cong \triangle PBD \\
  \therefore{}& CQ = PD, AP + PD + DQ = BD \\
  \because  {}& CD = CQ = DQ = 5 \\
  \therefore{}& PD = 5 \\
  \because  {}& AP = 3 \\
  \therefore{}& BD = AP + PD + DQ = 3 + 5 + 5 = 13
\end{align*}

故$BD = 13$。

\subsection{向内证全等} \label{subsec:009D-ieq}

\textit{XWT提供的方法,有改动。}

\begin{figure}[htbp]
  \centering
  \image{009D-ieq}
  \caption{解法~\ref{subsec:009D-ieq} 图} \label{fig:009D-ieq}
\end{figure}

如图~\ref{fig:009D-ieq},在$BD$上找一点$Q$,连接$PQ$,使得$\triangle DPQ$为等边三角形。易知$\angle BQP = \angle ADC = 120^\circ$,由解法~\ref{subsec:009D-oeq} 知$BP = AC, \angle PBQ = \angle CAD$。

\begin{align*}
  \because  {}& \text{在$\triangle CAD$与$\triangle PBQ$中,} \\
  & \left\{ \begin{aligned}
    \angle ADC &= \angle BQP \\
    \angle CAD &= \angle PBQ \\
    CA &= PB
  \end{aligned} \right. \\
  \therefore{}& \triangle CAD \cong \triangle PBQ \\
  \therefore{}& BQ = AD, PQ = CD \\
  \because  {}& CD = 5, DP = DQ = PQ \\
  \therefore{}& DP = DQ = 5 \\
  \because  {}& AP = 3 \\
  \therefore{}& BD = BQ + DQ = 3 + 5 + 5 = 13
\end{align*}

故$BD = 13$。

\subsection{四点共圆} \label{subsec:009D-circ}

\begin{figure}[htbp]
  \centering
  \image{009D-circ}
  \caption{解法~\ref{subsec:009D-circ} 图} \label{fig:009D-circ}
\end{figure}

如图~\ref{fig:009D-circ},连接$DE$,作$AQ \perp BC$于$Q$。由$\angle ADB = \angle AEB$知$A, B, D, E$四点共圆。

\begin{align*}
  \because  {}& \text{$A, B, D, E$四点共圆} \\
  \therefore{}& \angle ABC + \angle AED = 180^\circ \\
  \because  {}& \angle CED + \angle AED = 180^\circ \\
  \therefore{}& \angle ABC = \angle CED \\
  \because  {}& AB = AC \\
  \therefore{}& \angle ABC = \angle ECD \\
  \therefore{}& \angle CED = \angle ECD \\
  \therefore{}& DC = DE
\end{align*}

不妨设$\angle BAC = 2\alpha$,则易知$\angle ABC = 90^\circ - \alpha, \angle ABP = 120^\circ - 2\alpha$。同时,由$\angle ADB = 60^\circ$知$\angle BAP = 30^\circ + \alpha$,于是有$\angle BPA = 30^\circ + \alpha = \angle BAP$。由于$A, B, D, E$四点共圆,由圆周角定理知$\angle BAD = \angle BED$,于是有$\angle DEP = \angle DPE \Rightarrow DE = DP$。而$DC = 5, AP = 3$,故$DP = 5 \Rightarrow AD = 5 + 3 = 8$。

由于$AQ \perp BC, \angle ADQ = 60^\circ$,故$DQ = \sfrac12AD = 4 \Rightarrow CQ = 4 + 5 = 9$。而$\triangle ABC$是等腰三角形,由三线合一知$BQ = CQ = 9 \Rightarrow BD = 4 + 9 = 13$。

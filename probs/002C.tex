
\prob{002C}{余弦定理}

\begin{figure}[htbp]
  \centering
  \image{002C}
  \caption{002C:余弦定理} \label{fig:002C}
\end{figure}

如图~\ref{fig:002C},在任意$\triangle ABC$中,设$\alpha = \angle A$,证明余弦定理:

\[ \cos\alpha = \frac{b^2 + c^2 - a^2}{2bc} \]
\problabels{yellow/平面几何, green/证明题}

\subsection{勾股定理} \label{subsec:002C-pyth}

\begin{figure}[htbp]
  \centering
  \image{002C-pyth}
  \caption{\nameref{subsec:002C-pyth}:作垂线,然后利用勾股定理证明。} \label{fig:002C-pyth}
\end{figure}

基本思路:通过作垂线构造直角三角形,然后运用勾股定理并通过一系列代数变换证明命题。

如图~\ref{fig:002C-pyth},作$CD \perp AB$于$D$。

由正余弦函数的定义易知

\begin{align*}
  CD &= b\sin\alpha \\
  AD &= b\cos\alpha \\
\end{align*}

又由于$AB = c$,可知

\[ BD = c - b\cos\alpha \]

又由于$CD \perp AB$,即$\angle BDC = 90^\circ$,因此$\triangle BDC$是直角三角形。由勾股定理得

\begin{align*}
  BD^2 + CD^2 &= BC^2 \\
  (c - b\cos\alpha)^2 + (b\sin\alpha)^2 &= a^2 \\
  c^2 - 2bc\cos\alpha + b^2\cos^2\alpha & \\
  + b^2\sin^2\alpha &= a^2 \\
  -2bc\cos\alpha & \\
  + b^2(\cos^2\alpha + \sin^2\alpha) &= -c^2 + a^2 \\
  -2bc\cos\alpha + b^2 &= -c^2 + a^2 \\
  -2bc\cos\alpha &= -b^2 - c^2 + a^2 \\
  2bc\cos\alpha &= b^2 + c^2 - a^2 \\
  \cos\alpha &= \frac{b^2 + c^2 - a^2}{2bc} \\
\end{align*}

即为余弦定理。证毕。

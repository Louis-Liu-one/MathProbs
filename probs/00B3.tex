
\prob{00B3}{直角三角形二倍角}

\begin{figure}[htbp]
  \centering \image{00B3}
  \caption{总第~\ref{sec:00B3} 题} \label{fig:00B3}
\end{figure}

如图~\ref{fig:00B3},在$\rttri ABC$中,$\angle C = 90^\circ$;$D, E$是边$AC$上的点,连接$BD, BE$,有$AD = BD, \angle ABD = 2\angle CBE$。若$CE = 5, DE = 2$,求$AB$。
\problabels{yellow/平面几何, green/长度问题}

\emph{LHQ提供的题目。}

\ans{$AB = 30$}

\subsection{相似三角形} \label{subsec:00B3-sim}

\begin{figure}[htbp]
  \centering \image{00B3-sim}
  \caption{方法~\ref{subsec:00B3-sim} 图} \label{fig:00B3-sim}
\end{figure}

如图~\ref{fig:00B3-sim},延长$AC$至$E'$使得$CE' = CE = 5$,连接$BE'$。

显然有
\[ \triangle BCE \cong \triangle BCE' \Rightarrow BE = BE', \angle CBE = \angle CBE' \]
而$\angle A = \angle ABD = 2\angle CBE$,故
\[ \left\{ \begin{aligned}
  \angle A &= \angle EBE' \\ \angle E' &= \angle E'
\end{aligned} \right. \Rightarrow \triangle BAE' \sim \triangle EBE' \]
而$BE = BE'$,故$AB = AE'$。

不妨设$AD = BD = x$,则
\begin{align*}
  AB &= AE' = x + 2 + 5 + 5 = x + 12 \\
  AC &= AD + CD = x + 2 + 5 = x + 7
\end{align*}
在$\rttri ACB$与$\rttri DCB$中应用勾股定理知
\[ AB^2 - AC^2 = BD^2 - CD^2 \]
即
\[ (x + 12)^2 - (x + 7)^2 = x^2 - 7^2 \]
解得$x_1 = 18, x_2 = -8$(舍),故$AD = 18$。由勾股定理易得$AB = 30$。

\subsection{三角函数} \label{subsec:00B3-tri}

\begin{figure}[htbp]
  \centering \image{00B3-tri}
  \caption{方法~\ref{subsec:00B3-tri} 图} \label{fig:00B3-tri}
\end{figure}

如图~\ref{fig:00B3-tri},过$B$作$BF \perp AB$交直线$AC$于$F$;过$D$作$DH \perp AB$于$H$。不妨设$AH = BH = x, \angle CBE = \alpha$,于是有$\angle ABD = 2\alpha, \angle CDB = 4\alpha$。显然有$DH = x\tan2\alpha$。

显然$D$是$AF$的中点,于是$AD = BD = FD$,可得
\[ \angle F = 90^\circ - \frac12\angle BDF = 90^\circ - 2\alpha \]
而$\angle BCF = 90^\circ$,故$\angle CBF = 2\alpha$。

由于$D$是$AF$的中点,且$DH \parallel BF$,故$DH$是$\triangle ABF$的中位线,故
\[ BF = 2DH = 2x\tan2\alpha \]
因此,
\[ BC = BF\cos2\alpha = 2x\sin2\alpha \]

由于$\angle CBE = \alpha$,故
\begin{align*}
  & 2x\sin2\alpha\tan\alpha \\
  ={}& 2x\cdot2\sin\alpha\cos\alpha\cdot\frac{\sin\alpha}{\cos\alpha} \\
  ={}& 4x\sin^2\alpha = 5
\end{align*}
而
\[ \cos2\alpha = \cos^2\alpha - \sin^2\alpha = 1 - 2\sin^2\alpha \]
即
\[ \sin^2\alpha = \frac12(1 - \cos2\alpha) \]
代入得
\begin{align}
  2(1 - \cos2\alpha)x = 5 \Rightarrow \cos2\alpha = 1 - \frac5{2x} \label{eq:00B3-cos2alpha}
\end{align}
同时由于$\angle BDC = 4\alpha$,故
\begin{align*}
  & 2x\sin2\alpha = 7\tan4\alpha = 7\frac{\sin4\alpha}{\cos4\alpha} \\
  ={}& 7\frac{2\cos2\alpha\sin2\alpha}{\cos^22\alpha - \sin^22\alpha} \Rightarrow x = \frac{7\cos2\alpha}{2\cos^22\alpha - 1} \\
  \Rightarrow{}& \frac7x = \frac{2\cos^22\alpha - 1}{\cos2\alpha} = 2\cos2\alpha - \frac1{\cos2\alpha}
\end{align*}
将式~\ref{eq:00B3-cos2alpha} 代入上式得
\begin{align*}
  \frac7x &= 2\left(1 - \frac5{2x}\right) - \frac1{1 - 5/(2x)} \\
  &= 2 - \frac5x - \frac{2x}{2x - 5} \\
  \Rightarrow{}& \frac{12}x = 2 - \frac{2x}{2x - 5} \\
  \Rightarrow{}& \frac6x = 1 - \frac x{2x - 5} = \frac{x - 5}{2x - 5} \\
  \Rightarrow{}& 12x - 30 = x^2 - 5x
\end{align*}
解得$x_1 = 2, x_2 = 15$。当$x = 2$时,
\[ \cos2\alpha = 1 - \frac5{2x} = -\frac14 < 0 \]
不符题意,故$x = 15 \Rightarrow AB = 2x = 30$。

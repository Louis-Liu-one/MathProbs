
\prob{00C4}{中点上的角平分线}

\begin{figure}[htbp]
  \centering \image{00C4}
  \caption{总第~\ref{sec:00C4} 题图}
  \label{fig:00C4}
\end{figure}

如图~\ref{fig:00C4},等腰 $\triangle ABC$ 内接于 $\odot K$,$AC = BC$,$M$ 是 $AB$ 的中点;分别在 $\wideparen{BC}, \wideparen{AC}$ 上找点 $D, E$ 使得 $\wideparen{BD} = \wideparen{CE}$;连接 $AD, BE$ 交于 $F$,作直线 $CF$ 交 $\odot K$ 于 $G$;在直线 $CF$ 上找一点 $F'$ 使得 $FG = F'G$,连接 $MF, MF'$。求证:$\angle BMF = \angle BMF'$。
\problabels{yellow/平面几何, green/证明题}

\emph{LTY 提供的题目。}

\subsection{调和点列} \label{subsec:00C4-h}

\begin{figure}[htbp]
  \centering \image{00C4-h}
  \caption{方法~\ref{subsec:00C4-h} 图}
  \label{fig:00C4-h}
\end{figure}

如图~\ref{fig:00C4-h},作 $\triangle AFB$ 的外接 $\odot O$,连接 $OA, OB, OF, OF'$;作直线 $OC$,作 $FH \perp OC$ 于 $H$,$F'H' \perp OC$ 于 $H'$。设 $\angle ACB = 2\theta$。

显然
\[ \wideparen{BD} = \wideparen{CE} \Rightarrow \angle BAD = \angle CBE \]
于是
\begin{align*}
  \angle ABF + \angle BAF &= \angle ABF + \angle CBE  \\
  &= \angle ABC = 90^\circ - \theta \\
  \angle AFB &= 90^\circ + \theta
\end{align*}

显然
\[ \angle AOB + \angle OAF + \angle OBF + \angle AFB = 360^\circ \]
而
\[ \angle OAF = \angle OFA, \angle OBF = \angle OFB \]
故
\[ \angle AOB = 360^\circ - 2\angle AFB = 180^\circ - 2\theta \]
因此有
\[ \angle AOB + \angle ACB = 180^\circ \]
故 $A, O, B, C$ 共圆,即 $O$ 在 $\odot K$ 上。显然 $OC$ 是 $\odot K$ 的直径,故
\[ \angle OAC = \angle OBC = \angle OGC = 90^\circ \]
故 $CA, CB$ 切于 $\odot O$。

由于 $OG \perp FF', FG = F'G$,故 $OG$ 垂直平分 $FF'$,于是 $OF = OF'$。而 $OF$ 是 $\odot O$ 的半径,故 $F'$ 在 $\odot O$ 上。

由总第~\ref{sec:00C3} 题知,$C, F, M', F'$ 是调和点列,即
\[ \frac{CF}{CF'} = \frac{M'F}{M'F'} \]
而 $HF \parallel H'F'$,故
\[ \triangle HCF \sim \triangle H'CF' \Rightarrow \frac{CF}{CF'} = \frac{HF}{H'F'} \]
由平行线等比例分线段知
\[ \frac{M'F}{M'F'} = \frac{MH}{MH'} \]
故
\[ \left\{ \begin{aligned}
  \angle MHF &= \angle MH'F' \\
  \frac{HF}{H'F'} &= \frac{MH}{MH'}
\end{aligned} \right. \Rightarrow \triangle MHF \sim \triangle MH'F' \]
故有
\[ \angle HMF = \angle H'MF' \Rightarrow \angle BMF = \angle BMF' \]
命题得证。


\prob{0029}{圆形分区}

\begin{figure}[htbp]
  \centering
  \image{0029}
  \caption{0029:圆形分区} \label{fig:0029}
\end{figure}

在一圆上有$n$个点两两相连,若没有三线交于一点的情况,则可以将圆分成$F$个区域。例如3个点可以分出4个区域,6个点分出31个区域,图~\ref{fig:0029} 中5个点分出16个区域。求$F$(使用含$n$的代数式表示)。
\problabels{yellow/平面几何, green/数量关系问题}

\ans{$F = \sfrac1{24}(n^4 - 6n^3 + 23n^2 - 18n + 24)$}

\subsection{Euler定理}

基本思路:求出图形的点数、边数,然后利用Euler定理得出面数。

\subsubsection{点数}

$n$个点中,每4个点可以组成一个四边形,两条对角线交于一点,因此$n$个点可以交于$C_n^4$个点。加上原本的$n$个点,即有

\[ V = n + C_n^4 \]

\noindent 个点。

\subsubsection{边数}

若$p$条线段交于$q$个点,因为每交于一个点就将已有的两条边分成四条边,即增加两条边。再加上原本的$p$条边,就有

\[ p + 2q \]

\noindent 条边。在题目中,由于$n$个点中任选两个点构成一条线段,因此$p = C_n^2$。而$q = C_n^4$,即$n$个点交出的$C_n^4$个点(不算原本的$n$个点),因此就有

\[ C_n^2 + 2C_n^4 \]

\noindent 条边。由于$n$个点又将圆周分成了$n$条边,所以总边数为

\[ E = n + C_n^2 + 2C_n^4 \]

\subsubsection{Euler定理}

根据Euler定理,

\[ V + F - E = 2 \]

即

\[ F = E - V + 2 \]

代入$E$、$V$得

\begin{align*}
  F &= (n + C_n^2 + 2C_n^4) - (n + C_n^4) + 2 \\
  &= C_n^2 + C_n^4 + 2 \\
\end{align*}

不算圆周外的面,可得

\begin{align*}
  F &= C_n^2 + C_n^4 + 1 \\
  &= \frac{n!}{2!(n - 2)!} + \frac{n!}{4!(n - 4)!} + 1 \\
  &= \frac12n(n - 1) + \frac1{24}n(n - 1)(n - 2)(n - 3) + 1 \\
  &= \frac1{24}(n^4 - 6n^3 + 23n^2 - 18n + 24) \\
\end{align*}

综上,$F = \sfrac1{24}(n^4 - 6n^3 + 23n^2 - 18n + 24)$。


\prob{0050}{分式方程I}

求关于$x$的方程
\begin{align*}
  & \frac{(x + 4)(x + 2)}{(x - 4)(x - 2)} + \frac{(x - 4)(x - 2)}{(x + 4)(x + 2)} \\
  ={}& \frac{(x + 3)(x + 1)}{(x - 3)(x - 1)} + \frac{(x - 3)(x - 1)}{(x + 3)(x + 1)}
\end{align*}
的实数根。
\problabels{yellow/代数, green/方程相关问题}

\ans{$x = 0$或$x = \pm\sqrt7$}

\subsection{移项}

移项得

\begin{align*}
  & \frac{(x + 4)(x + 2)}{(x - 4)(x - 2)} - \frac{(x + 3)(x + 1)}{(x - 3)(x - 1)} \\
  ={}& \frac{(x - 3)(x - 1)}{(x + 3)(x + 1)} - \frac{(x - 4)(x - 2)}{(x + 4)(x + 2)}
\end{align*}

等式两边通分得

\begin{align*}
  & \Big((x - 1)(x + 2)(x - 3)(x + 4) \\
  &- (x + 1)(x - 2)(x + 3)(x - 4)\Big) \\
  &/ \Big((x - 1)(x - 2)(x - 3)(x - 4)\Big) \\
  ={}& \Big((x - 1)(x + 2)(x - 3)(x + 4) \\
  &- (x + 1)(x - 2)(x + 3)(x - 4)\Big) \\
  &/ \Big((x + 1)(x + 2)(x + 3)(x + 4)\Big) \\
\end{align*}

可见两边分子相同。当两边分子为0时,

\begin{align*}
  & (x - 1)(x + 2)(x - 3)(x + 4) \\
  &- (x + 1)(x - 2)(x + 3)(x - 4) \\
  ={}& ((x^2 + x) - 2)((x^2 + x) - 12) \\
  &- ((x^2 - x) - 2)((x^2 - x) - 12) \\
  ={}& (x^2 + x)^2 - 14(x^2 + x) + 24 \\
  &- (x^2 - x)^2 + 14(x^2 - x) - 24 \\
  ={}& 2x^2\cdot2x - 28x \\
  ={}& 4x^3 - 28x = 4x(x^2 - 7) \\
  ={}& 4x(x + \sqrt7)(x - \sqrt7) = 0 \\
\end{align*}

故此时$x = 0$或$x = \pm\sqrt7$。

当两边分母相同时,

\begin{align*}
  & (x + 1)(x + 2)(x + 3)(x + 4) \\
  &- (x - 1)(x - 2)(x - 3)(x - 4) \\
  ={}& ((x^2 + 5x) + 4)((x^2 + 5x) + 6) \\
  &- ((x^2 - 5x) + 4)((x^2 - 5x) + 6) \\
  ={}& (x^2 + 5x)^2 + 10(x^2 + 5x) + 24 \\
  &- (x^2 - 5x)^2 - 10(x^2 - 5x) - 24 \\
  ={}& 20x^3 + 100x = 20x(x^2 + 5) = 0 \\
\end{align*}

故此时$x = 0$。

综上,原方程实数根为$x = 0$或$x = \pm\sqrt7$。

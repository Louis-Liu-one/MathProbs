
\prob{0054}{勾股定理逆定理}

\begin{figure}[htbp]
  \centering
  \image{0054}
  \caption{0054:勾股定理逆定理} \label{fig:0054}
\end{figure}

证明勾股定理逆定理:若三角形中两边平方和等于第三边的平方,则该两边夹角为直角。
\problabels{yellow/平面几何, green/证明题}

\subsection{余弦定理}

基本思路:通过余弦定理证明。注意,余弦定理的证明仅用到勾股定理,而没有用到其逆定理,故不构成循环论证。

如图~\ref{fig:0054},由余弦定理知,

\[ \cos\angle C = \frac{a^2 + b^2 - c^2}{2ab} \]

又由题知,$a^2 + b^2 = c^2$,故$\cos\angle C = 0$,$\angle C = 90^\circ$。

证毕。

\subsection{构造全等三角形} \label{subsec:0054-eqtri}

\begin{figure}[htbp]
  \centering
  \image{0054-eqtri}
  \caption{\nameref{subsec:0054-eqtri}:证明$\triangle ABC$与$\triangle A'BC$全等。}
  \label{fig:0054-eqtri}
\end{figure}

基本思路:通过证明$\triangle ABC$与某直角三角形全等,进而证明命题。

如图~\ref{fig:0054-eqtri},在直线$BC$下方找一点$A'$,使得$A'C = AC = b$且$A'C \perp BC$于$C$,连接$A'B$。

由勾股定理知$A'C^2 + BC^2 = A'B^2$,即$a^2 + b^2 = A'B^2$,而$a^2 + b^2 = c^2$,故$A'B = AB = c$。

由于在$\triangle ABC$与$\triangle A'BC$中,

\[ \begin{cases}
  AB = A'B = c \\
  BC = BC = a \\
  AC = A'C = b \\
\end{cases} \]

可知$\triangle ABC \cong \triangle A'BC$。由此可知$\angle ACB = \angle A'CB$。而$\angle A'CB = 90^\circ$,故$\angle ACB = 90^\circ$。

证毕。

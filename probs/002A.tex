
\prob{002A}{Heron公式}

\begin{figure}[htbp]
  \centering
  \image{002A}
  \caption{002A:Heron公式} \label{fig:002A}
\end{figure}

证明Heron公式:对任意$\triangle ABC$,设其面积为$S$,三边分别为$a$、$b$、$c$,又设$p = \sfrac12(a + b + c)$,则

\[ S = \sqrt{p(p - a)(p - b)(p - c)} \]
\problabels{yellow/平面几何, green/证明题}

\subsection{余弦定理}

基本思路:通过余弦定理求出$\sin\angle A$,然后用$a$、$b$、$c$表示$S$,整理得Heron公式。

设$\alpha = \angle A$。

由余弦定理知

\[ \cos\alpha = \frac{b^2 + c^2 - a^2}{2bc} \]

由$\cos^2\alpha + \sin^2\alpha = 1$知

\begin{align*}
  \sin\alpha &= \sqrt{1 - \cos^2\alpha} \\
  &= \sqrt{1 - \left(\frac{b^2 + c^2 - a^2}{2bc}\right)} \\
  &= \frac{\sqrt{(2bc)^2 - (b^2 + c^2 - a^2)^2}}{2bc} \\
\end{align*}

又由$S = \sfrac12bc\sin\alpha$知

\begin{align*}
  S &= \frac12bc\sin\alpha \\
  &= \frac12bc\cdot\frac{\sqrt{(2bc)^2 - (b^2 + c^2 - a^2)^2}}{2bc} \\
  &= \frac14\sqrt{(2bc)^2 - (b^2 + c^2 - a^2)^2} \\
  &= \frac14\Big((2bc + b^2 + c^2 - a^2) \\
  &\phantom{=} \cdot(2bc - b^2 - c^2 + a^2)\Big)^{\sfrac12} \\
  &= \frac14\sqrt{((b + c)^2 - a^2)(-(b - c)^2 + a^2)} \\
  &= \frac14\Big((a + b + c)(-a + b + c) \\
  &\phantom{=} \cdot(a - b + c)(a + b - c)\Big)^{\sfrac12} \\
  &= \Bigg(\frac1{16}(a + b + c)(-a + b + c) \\
  &\phantom{=} \cdot(a - b + c)(a + b - c)\Bigg)^{\sfrac12} \\
  &= \Bigg(\frac{a + b + c}2\cdot\frac{-a + b + c}2 \\
  &\phantom{=} \cdot\frac{a - b + c}2\cdot\frac{a + b - c}2\Bigg)^{\sfrac12} \\
\end{align*}

设$p = \sfrac12(a + b + c)$,则

\begin{align*}
  \frac{a + b + c}2 &= p \\
  \frac{-a + b + c}2 &= p - a \\
  \frac{a - b + c}2 &= p - b \\
  \frac{a + b - c}2 &= p - c \\
\end{align*}

因此

\[ S = \sqrt{p(p - a)(p - b)(p - c)} \]

证毕。

\subsection{内切圆} \label{subsec:002A-circ}

\begin{figure}[htbp]
  \centering
  \image{002A-circ}
  \caption{\nameref{subsec:002A-circ}:通过证明相似三角形,然后利用一系列关系和定理证明命题。}
  \label{fig:002A-circ}
\end{figure}

基本思路:通过作三角形的内切圆得出相似三角形,然后通过代数计算并运用射影定理等证明命题。

如图~\ref{fig:002A-circ},作$\angle BAC$的平分线、$\angle ABC$的平分线、$\angle ACB$的平分线,三线交于点$O$;作$OD \perp BC$于$D$,$OE \perp AC$于$E$,$OF \perp AB$于$F$;延长$AB$到点$M$使得$BM = CE$;作$BN \perp AB$,$ON \perp OA$,两线交于点$N$,$ON$与$AB$交于点$L$;连接$AN$。

由于$O$是$\triangle ABC$的内心,所以由三角形内心的性质,可以设

\begin{align*}
  AE = AF &= x \\
  BD = BF &= y \\
  CD = CE = BM &= z \\
  OD = OE = OF &= r \\
  FL &= h \\
  p &= \frac12(a + b + c) \\
\end{align*}

则可知

\begin{align*}
  BL &= y - h \\
  x + y &= c \\
  y + z &= a \\
  x + z &= b \\
\end{align*}

因此,

\begin{align*}
  p &= \frac12(a + b + c) \\
  &= \frac12(x + y + y + z + x + z) \\
  &= x + y + z \\
  x &= p - a \\
  y &= p - b \\
  z &= p - c \\
  AM &= AF + BF + BM \\
  &= x + y + z \\
  &= p \\
\end{align*}

又由于$\angle AON = \angle ABN = 90^\circ$,可知$A, O, B, N$四点共圆,根据圆周角定理\footnote{圆周角定理的证明参见第~\ref{sec:0025} 题。},

\[ \angle ONA = \angle OBA \]

因此,

\begin{align*}
  &\phantom{=} \angle OBA + \angle OAB + \angle BAN \\
  &= \angle ONA + \angle OAB + \angle BAN \\
  &= 90^\circ \\
\end{align*}

又由于$OA$、$OB$、$OC$分别平分$\angle BAC$、$\angle ABC$、$\angle ACB$,且$\angle BAC + \angle ABC + \angle ACB = 180^\circ$,所以

\begin{align*}
  &\phantom{=} \angle OBA + \angle OAB + \angle ECO \\
  &= \frac12\angle ABC + \frac12\angle BAC + \frac12\angle ACB \\
  &= \frac12\cdot 180^\circ = 90^\circ \\
\end{align*}

因此

\begin{align*}
  &\phantom{=} \angle OBA + \angle OAB + \angle BAN \\
  &=\angle OBA + \angle OAB + \angle ECO \\
  &\phantom{=} \angle BAN = \angle ECO \\
\end{align*}

又由于$\angle ABN = \angle CEO = 90^\circ$,所以可得

\[ \triangle ABN \sim \triangle CEO \]

即

\[ \frac{BN}{AB} = \frac{OE}{CE} = \frac{OE}{BM} \]

所以

\[ \frac{BM}{AB} = \frac{OE}{BN} = \frac{OF}{BN} \]

又因为$\angle OFB = \angle FBN$,所以$OF \parallel BN$,因此,

\[ \frac{OF}{BN} = \frac{FL}{BL} \]

因此,

\begin{align*}
  \frac{BM}{AB} &= \frac{FL}{BL} \\
  \frac{BM + AB}{AB} &= \frac{FL + BL}{BL} \\
  \frac{AM}{AB} &= \frac{BF}{BL} \\
  AM\cdot BL &= AB\cdot BF \\
  p(y - h) &= cy \\
  ph &= py - cy \\
  &= y(p - c) \\
  &= yz \\
  pxh &= xyz \\
\end{align*}

由射影定理\footnote{射影定理可以由相交弦定理推知。相交弦定理的证明参见第~\ref{sec:0028} 题。}可知,在$\triangle AOL$中,

\begin{align*}
  AF\cdot FL &= OF^2 \\
  xh &= r^2 \\
\end{align*}

因此,

\begin{align*}
  pxh &= xyz \\
  pr^2 &= xyz \\
  p^2r^2 &= pxyz \\
\end{align*}

又因为

\begin{align*}
  S &= S_{\triangle AOB} + S_{\triangle AOC} + S_{\triangle BOC} \\
  &= \frac12AB\cdot OF + \frac12AC\cdot OE \\
  &+ \frac12BC\cdot OD \\
  &= \frac12cr + \frac12br + \frac12ar \\
  &= r\cdot\frac12(a + b + c) \\
  &= pr \\
  S^2 &= p^2r^2 \\
\end{align*}

所以

\begin{align*}
  p^2r^2 &= pxyz \\
  S^2 &= p(p - a)(p - b)(p - c) \\
  S &= \sqrt{p(p - a)(p - b)(p - c)} \\
\end{align*}

即为Heron公式。证毕。

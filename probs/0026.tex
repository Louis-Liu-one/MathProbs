
\prob{0026}{弦切角定理}

\begin{figure}[htbp]
  \centering
  \image{0026}
  \caption{0026:弦切角定理} \label{fig:0026}
\end{figure}

证明弦切角定理:弦切角与它所夹的弧所对的圆周角相等。
\problabels{yellow/平面几何, green/证明题}

\subsection{等腰三角形} \label{subsec:0026-eqtri}

\begin{figure}[htbp]
  \centering
  \image{0026-eqtri}
  \caption{\nameref{subsec:0026-eqtri}:利用圆周角定理结合角度关系证明。}
  \label{fig:0026-eqtri}
\end{figure}

基本思路:寻找等腰三角形,利用圆周角定理计算角度,得出结论。

如图~\ref{fig:0026-eqtri},连接$OB$、$OC$。

\begin{align*}
  &\because   OC\text{是圆}O\text{的半径}, \\
  &\mathalignsep P_1P_2\text{是圆}O\text{过点}C\text{的切线} \\
  &\therefore OC \perp P_1P_2 \\
  &\therefore \angle OCP_1 = 90^\circ \\
  &\therefore \angle OCB + \angle BCP_1 = 90^\circ \\
  &\because   OB = OC \\
  &\therefore \angle OCB = \angle OBC \\
  &\because   \angle OCB + \angle OBC + \angle BOC = 180^\circ \\
  &\therefore \angle OCB = 90^\circ - \frac12\angle BOC \\
  &\because   \text{据圆周角定理\footnotemark 得,}\angle BOC = 2\angle BAC \\
  &\therefore \angle OCB = 90^\circ - \angle BAC \\
  &\therefore \angle BCP_1 = \angle BAC \\
\end{align*}

\footnotetext{圆周角定理的证明参见第~\ref{sec:0025} 题。}

证毕。


\prob{0098}{外角部分相等}

\begin{figure}[htbp]
  \centering
  \image{0098}
  \caption{总第~\ref{sec:0098} 题图} \label{fig:0098}
\end{figure}

如图~\ref{fig:0098},在$\triangle ABC$中,$D$是$AB$的中点。延长$CA, CB$到点$E, F$,使得$DE = DF$。过$E, F$分别作$EP \perp CE, FP \perp CF$,两垂线交于点$P$,连接$AP, BP$。求证:$\angle PAE = \angle PBF$。
\problabels{yellow/平面几何, green/证明题}

\subsection{构造全等三角形} \label{subsec:0098-eqtri}

\begin{figure}[htbp]
  \centering
  \image{0098-eqtri}
  \caption{解法~\ref{subsec:0098-eqtri} 图} \label{fig:0098-eqtri}
\end{figure}

如图~\ref{fig:0098-eqtri},作线段$AP, BP$的中点$M, N$,连接$DM, DN, EM, FN$。

由直角三角形的性质易知$AM = EM = \sfrac12AP, BN = FN = \sfrac12BP$。同时,$DM, DN$均为$\triangle APB$的中位线,故$DN = \sfrac12AP, DM = \sfrac12BP$且$DN \parallel AP, DM \parallel BP$。于是有$EM = DN, FN = DM$。

\begin{align*}
  \because{}& \text{在$\triangle DME$与$\triangle FND$中,} \\
  & \left\{ \begin{aligned}
    DM &= FN \\
    ME &= ND \\
    DE &= FD \\
  \end{aligned} \right. \\
  \therefore{}& \triangle DME \cong \triangle FND \\
  \therefore{}& \angle DME = \angle DNF \\
  \because  {}& DN \parallel AP, DM \parallel BP \\
  \therefore{}& \angle AMD = \angle APB = \angle BND \\
  \therefore{}& \angle AME = \angle BNF \\
  \because  {}& AM = EM, BN = FN \\
  \therefore{}& \angle PAE = 90^\circ - \frac12\angle AME, \\
  & \angle PBF = 90^\circ - \frac12\angle BNF \\
  \because  {}& \angle AME = \angle BNF \\
  \therefore{}& \angle PAE = \angle PBF
\end{align*}

证毕。


\prob{002B}{正弦定理}

\begin{figure}[htbp]
  \centering
  \image{002B}
  \caption{002B:正弦定理} \label{fig:002B}
\end{figure}

证明正弦定理:在任意一个平面三角形中,各边和它所对角的正弦值之比相等。
\problabels{yellow/平面几何, green/证明题}

\subsection{面积法}

基本思路:通过将三角形的面积用不同的方式表示,然后列出等式证明。

由于$\alpha$的正弦为$\sin\alpha$,易知三角形$AB$边上的高是$b\sin\alpha$,又由于底为$c$,所以三角形的面积

\[ S = \frac12bc\sin\alpha \]

同理得

\begin{align*}
  S &= \frac12ac\sin\beta \\
  &= \frac12ab\sin\gamma \\
\end{align*}

因此,

\[ \frac12bc\sin\alpha = \frac12ac\sin\beta = \frac12ab\sin\gamma \]

每个单项式同除$\sfrac12abc$可知

\[ \frac{\sin\alpha}a = \frac{\sin\beta}b = \frac{\sin\gamma}c \]

即为正弦定理。证毕。

\subsection{作垂线} \label{subsec:002B-vert}

\begin{figure}[htbp]
  \centering
  \image{002B-vert}
  \caption{\nameref{subsec:002B-vert}:通过作垂线运用正弦函数的定义证明。}
  \label{fig:002B-vert}
\end{figure}

基本思路:通过作垂线构造直角三角形,然后运用正弦函数的定义证明命题。

如图~\ref{fig:002B-vert},作三角形$AB$边上的高,设为$h$。

图中有两个直角三角形:$\mathrm{Rt}\triangle ADC$与$\mathrm{Rt}\triangle BDC$,由正弦函数的定义知,

\begin{align*}
  \sin\alpha &= \frac hb \\
  \sin\beta &= \frac ha \\
\end{align*}

于是知

\begin{align*}
  \frac{\sin\alpha}{\sin\beta} &= \frac ab \\
  \frac{\sin\alpha}a &= \frac{\sin\beta}b \\
\end{align*}

同样通过作垂线的方法知

\[ \frac{\sin\beta}b = \frac{\sin\gamma}c \]

整理得

\[ \frac{\sin\alpha}a = \frac{\sin\beta}b = \frac{\sin\gamma}c \]

即为正弦定理。证毕。

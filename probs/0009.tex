
\prob{0009}{线段和}

\begin{figure}[htbp]
  \centering
  \image{0009}
  \caption{0009:线段和} \label{fig:0009}
\end{figure}

如图~\ref{fig:0009},在$\triangle ABC$中,$AB = AC$,$D$是$\triangle ABC$外一点,且$\angle ABD = 60^\circ$,$\angle ADB = 90^\circ - \sfrac12\angle BDC$,求证:$AC = BD + CD$。
\problabels{yellow/平面几何, green/证明题}

\subsection{翻折$BD$边} \label{subsec:0009-BD}

\begin{figure}[htbp]
  \centering
  \image{0009-BD}
  \caption{\nameref{subsec:0009-BD}:翻折$\triangle ABD$至$\triangle AB'D$,从而构造等边三角形。}
  \label{fig:0009-BD}
\end{figure}

基本思路:通过翻折三角形构造等边三角形,从而证明命题。

如图~\ref{fig:0009-BD},延长$CD$到$B'$,使得$B'D = BD$。连接$AB'$。

\begin{align*}
  \because  {}& \angle ADB = 90^\circ - \frac12\angle BDC \\
  \therefore{}& 2\angle ADB + \angle BDC = 180^\circ \\
  \because  {}& \angle ADB + \angle ADB' + \angle BDC = 180^\circ \\
  \therefore{}& \angle ADB = \angle ADB' \\
  \because  {}& \text{在 $\triangle ADB$ 与 $\triangle CDB'$ 中} \\
  & \begin{cases}
    AD = AD \\
    \angle ADB = \angle ADB' \\
    DB = DB'
  \end{cases} \\
  \therefore{}& \triangle ADB \cong \triangle ADB' \\
  \therefore{}& AB = AB', \angle ABD = \angle AB'D \\
  \because  {}& \angle ABD = 60^\circ \\
  \therefore{}& \angle AB'D = 60^\circ \\
  \because  {}& AB = AC \\
  \therefore{}& AB' = AC \\
  \because  {}& \text{在 $\triangle AB'C$ 中} \\
  & \begin{cases}
    \angle AB'C = 60^\circ \\
    AB' = AC
  \end{cases} \\
  \therefore{}& \text{$\triangle AB'C$ 是等边三角形} \\
  \therefore{}& AC = B'C \\
  \because  {}& B'C = B'D + CD \\
  \therefore{}& AC = BD + CD
\end{align*}

证毕。

\subsection{翻折$CD$边} \label{subsec:0009-CD}

\begin{figure}[htbp]
  \centering
  \image{0009-CD}
  \caption{\nameref{subsec:0009-CD}:翻折$\triangle ACD$至$\triangle AC'D$,从而构造等边三角形。}
  \label{fig:0009-CD}
\end{figure}

\emph{XJX提供的方法。}

基本思路:通过翻折三角形构造等边三角形,从而证明命题。

如图~\ref{fig:0009-CD},延长$BD$到$C'$,使得$C'D = CD$。连接$AC'$。

\begin{align*}
  \because  {}& \angle ADB = 90^\circ - \frac12\angle BDC \\
  \therefore{}& 2\angle ADB + \angle BDC = 180^\circ \\
  \because  {}& \angle ADB + \angle ADC' = 180^\circ \\
  \therefore{}& \angle ADC' = \angle ADB + \angle BDC = \angle ADC \\
  \because  {}& \text{在 $\triangle ADC$ 与 $\triangle ADC'$ 中} \\
  & \begin{cases}
    AD = AD \\
    \angle ADC = \angle ADC' \\
    DC = DC'
  \end{cases} \\
  \therefore{}& \triangle ADC \cong \triangle ADC' \\
  \therefore{}& AC = AC' \\
  \because  {}& AB = AC \\
  \therefore{}& AB = AC' \\
  \because  {}& \text{在}\triangle ABC'\text{中} \\
  & \begin{cases}
    \angle ABC' = 60^\circ \\
    AB = AC'
  \end{cases} \\
  \therefore{}& \text{$\triangle ABC'$ 是等边三角形} \\
  \therefore{}& BC' = AB = AC \\
  \because  {}& BC' = BD + C'D \\
  \therefore{}& AC = BD + CD
\end{align*}

证毕。

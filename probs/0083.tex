
\prob{0083}{Gauss引理}

证明:若某整系数多项式在有理数集上可约,则其在整数集上可约。
\problabels{yellow/数论, green/证明题}

\subsection{部分使用反证法}

设$f(x) = a_nx^n + a_{n - 1}x^{n - 1} + \dots + a_1x + a_0$是整系数多项式,且存在
\begin{align*}
  g(x) &= b_rx^r + b_{r - 1}x^{r - 1} + \dots + b_1x + b_0 \\
  h(x) &= c_sx^s + c_{s - 1}x^{s - 1} + \dots + c_1x + c_0
\end{align*}
其中$b, c$都是既约的有理数,$r + s = n$,使得$f(x) = g(x)h(x)$,即要证明$f(x)$可以写成2个整系数多项式的乘积。

设$g(x)$的系数的分母的最小公倍数为$m_g$,$h(x)$的系数的分母的最小公倍数为$m_h$,于是得到两整系数多项式$g'(x) = m_gg(x), h'(x) = m_hh(x)$。设$g'(x)$的系数的分子的最大公因数为$d_g$,$h'(x)$的系数的分子的最大公因数为$d_h$,于是又得到两整系数多项式
\begin{align*}
  g''(x) &= \frac{g'(x)}{d_g} = \frac{m_g}{d_g}g(x) \\
  h''(x) &= \frac{h'(x)}{d_h} = \frac{m_h}{d_h}h(x) \\
\end{align*}
它们的系数互质,设它们的系数为$b'', c''$。令$d = d_gd_h, m = m_gm_h$,于是有
\[ f(x) = g(x)h(x) = \frac dmg''(x)h''(x) \]
有理数$d/m$可能不是既约的,约去后得既约的$d'/m'$,即
\[ m'f(x) = d'g''(x)h''(x) \]

若$m' = 1$,则$f(x)$可以表示为两整系数多项式的乘积,命题得证;若$m' \ne 1$,则其有质因数$p$,故等号右边亦有质因数$p$。而$d'$与$m'$互质,故其无质因数$p$,故$p$整除$g''(x)h''(x)$的每一项系数。

而$p$不可能整除$g''(x)$的每一项系数,因为其系数互质。设$g''$的系数中第一个不能被$p$整除的系数为$b''_i$,则$b''_0, b''_1, \dots, b''_{i - 1}$均被$p$整除。同理,设$h''$的系数中第一个不能被$p$整除的系数为$c''_j$,则$c''_0, c''_1, \dots, c''_{j - 1}$均被$p$整除。

于是$g''(x)h''(x)$的$x^{i + j}$项的系数为
\[ a''_{i + j} = b''_{i + j}c''_0 + b''_{i + j - 1}c''_1 + \dots + b''_1c''_{i + j - 1} + b''_0c''_{i + j} \]
约定当$k > r$时,$b''_k = 0$;当$k > s$时,$c''_k = 0$。其中有一项$b''_ic''_j$不能被$p$整除,其余项均被$p$整除,故$p \nmid a''_{i + j}$,于是$p$不整除$g''(x)h''(x)$的每一项系数,原假设不成立,故$m' = 1$,命题得证。


\prob{0095}{线段比例III}

\begin{figure}[htbp]
  \centering
  \image{0095}
  \caption{总第~\ref{sec:0095} 题图} \label{fig:0095}
\end{figure}

如图~\ref{fig:0095},四边形$ABFE$是矩形,$D$是对角线$AF, BE$的交点。过$E$的一直线$CG$交$BF$的延长线于$C$,交$BA$的延长线于$G$。连接$DC, DG$,$DC = DG$。求证:
\[ \frac{AB}{FC} = \frac{FC}{GA} = \frac{GA}{AE} \]
\problabels{yellow/平面几何, green/证明题}

\subsection{切割线定理} \label{subsec:0095-tan}

\begin{lemma} \label{lemma:0095-ltan}
  \begin{figure}
    \centering
    \image{0095-ltan}
    \caption{引理~\ref{lemma:0095-ltan} 图}
    \label{fig:0095-ltan}
  \end{figure}
  如图~\ref{fig:0095-ltan},从半径为$r$的圆$O$外一点$P$引出一条直线,交圆于$A, B$。连接$OP$,可知
  \[ AP \cdot BP = OP^2 - r^2 \]
\end{lemma}

\begin{proof}
  \begin{figure}
    \centering
    \image{0095-ltan-p}
    \caption{引理~\ref{lemma:0095-ltan} 证明图}
    \label{fig:0095-ltan-p}
  \end{figure}
  如图~\ref{fig:0095-ltan-p},从$P$引圆的切线$PC$,切圆于$C$,连接$OC$。由切割线定理知,$AP \cdot BP = CP^2$。由于$PC$切圆$O$于$C$,故$PC \perp OC$,由勾股定理知
  \[ CP^2 = OP^2 - OC^2 = OP^2 - r^2 \]
  代入$AP \cdot BP = CP^2$即得。
\end{proof}

\begin{figure}[htbp]
  \centering
  \image{0095-tan}
  \caption{解法~\ref{subsec:0095-tan} 图} \label{fig:0095-tan}
\end{figure}

如图~\ref{fig:0095-tan},作矩形$ABFE$的外接圆,$D$为圆心,令其半径为$r$。

易证$\triangle CFE \sim \triangle EAG$,同时注意到有$\triangle EAG \sim \triangle CBG$,于是
\begin{align}
  \frac{AB}{FC} = \frac{EF}{FC} = \frac{GB}{BC} = \frac{GA}{AE} \label{eq:0095-tan}
\end{align}

由引理~\ref{lemma:0095-ltan} 知,
\begin{align*}
  FC \cdot BC &= DC^2 - r^2 \\
  GB \cdot GA &= DG^2 - r^2 \\
\end{align*}
而$DC = DG$,于是
\[ FC \cdot BC = GB \cdot GA \Rightarrow \frac{FC}{GA} = \frac{GB}{BC} \]
结合式~\ref{eq:0095-tan} 可知
\[ \frac{AB}{FC} = \frac{FC}{GA} = \frac{GA}{AE} \]
证毕。

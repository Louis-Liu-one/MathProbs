
\prob{006E}{矩形与最小值}

\begin{figure}[htbp]
  \centering
  \image{006E}
  \caption{006E:第一象限的矩形与最小值} \label{fig:006E}
\end{figure}

如图~\ref{fig:006E},在第一象限内有矩形$ABCD$,其中$A, B$分别在$y, x$正半轴上,且$AB = CD = 2, AD = BC = 1$,求$OD$的最大值。
\problabels{yellow/解析几何, green/最值问题}

\ans{$OD$的最大值为$1 + \sqrt2$。}

\subsection{构造三角形} \label{subsec:006E-tri}

基本思路:通过构造三角形使得$OD$在一个另外两边都为定值三角形中。

\begin{figure}[htbp]
  \centering
  \image{006E-tri}
  \caption{\nameref{subsec:006E-tri}:构造两边为定值的三角形。}
  \label{fig:006E-tri}
\end{figure}

如图~\ref{fig:006E-tri},作$AB$的中点$E$,连接$OE, DE$。

由于$\triangle AOB$是直角三角形,且$OE$是其斜边上的中线,故$2OE = AB = 2$,即$OE = AE = 1$。

而由$AE = AD = 1, \angle BAD = 90^\circ$,有$DE = \sqrt2$。

由此易知,当$O, D, E$三点共线时,$OD$取最大值$OE + DE$,即其最大值为$1 + \sqrt2$。

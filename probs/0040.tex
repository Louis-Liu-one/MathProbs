
\prob{0040}{格点中点}

证明:在平面直角坐标系中任选5个横纵坐标均为整数的点,其中至少有一对点的中点横纵坐标都是整数。
\problabels{yellow/解析几何, green/证明题}

\subsection{奇偶性}

基本思路:利用点的横纵坐标的奇偶性质证明。

根据简单的分析可知,这5个点的横坐标中至少有3个奇偶性相同,这3个点又至少有2个点的纵坐标奇偶性相同。

令这两个点分别为$(x_1, y_1)$和$(x_2, y_2)$,则我们可知$x_1$与$x_2$奇偶性相同,$y_1$与$y_2$奇偶性亦相同。即$x_1 + x_2$和$y_1 + y_2$均为偶数。

又由中点坐标公式知这两个点的中点为

\[ \left(\frac{x_1 + x_2}2, \frac{y_1 + y_2}2\right) \]

因为$x_1 + x_2$和$y_1 + y_2$均为偶数,所以$\sfrac{x_1 + x_2}2$和$\sfrac{y_1 + y_2}2$均为整数,因此选中的5个点中至少有一对点,满足其中点横纵坐标都是整数。

证毕。

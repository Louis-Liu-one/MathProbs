
\prob{000D}{平行于角平分线}

\begin{figure}[htbp]
  \centering
  \image{000D}
  \caption{000D:平行于角平分线} \label{fig:000D}
\end{figure}

如图~\ref{fig:000D},在任意三角形$ABC$中,$D$在线段$BC$上且$AD$平分$\angle BAC$,$Q$是线段$BC$的中点,$P$在线段$AC$上且$PQ \parallel AD$,求$AB$、$AP$与$CP$的数量关系。
\problabels{yellow/平面几何, green/数量关系问题}

\ans{$AB + AP = CP$}

\subsection{倍长“中线”} \label{subsec:000D-mid}

\begin{figure}[htbp]
  \centering
  \image{000D-mid}
  \caption{\nameref{subsec:000D-mid}:运用倍长中线的思想构造8形全等,从而找出一个平行四边形和一个等腰三角形,将三条线段拼在一起。}
  \label{fig:000D-mid}
\end{figure}

基本思路:构造8形全等从而构造出一个平行四边形,将三条线段拼在一起。

如图~\ref{fig:000D-mid},延长$PQ$到$P'$使得$PQ = P'Q$,连接$BP'$,与$AD$的延长线交于$A'$。

\begin{align*}
  &\because   Q\text{是线段}BC\text{的中点} \\
  &\therefore BQ = CQ \\
  &\because   \text{在}\triangle CPQ\text{与}\triangle BP'Q\text{中} \\
  &\mathalignsep \begin{cases}
    CQ = BQ \\
    \angle CQP = \angle BQP' \\
    PQ = P'Q \\
  \end{cases} \\
  &\therefore \triangle CPQ \cong \triangle BP'Q \\
  &\therefore CP = BP', \angle P' = \angle CPP' \\
  &\therefore AC \parallel BP' \\
  &\therefore \angle A'AC = \angle AA'B \\
  &\because   AD\text{平分}\angle BAC \\
  &\therefore \angle A'AC = \angle A'AB \\
  &\therefore \angle A'AB = \angle AA'B \\
  &\therefore AB = A'B \\
  \text{又}&\because AA' \parallel PP', AC \parallel BP' \\
  &\therefore \text{四边形}A'APP'\text{为平行四边形} \\
  &\therefore AP = A'P' \\
  &\because   A'B + A'P' = BP' \\
  &\therefore AB + AP = CP \\
\end{align*}

综上,$AB$、$AP$与$CP$的数量关系为$AB + AP = CP$。

\subsection{角平分线定理}

基本思路:通过角平分线定理以及中点、平行等线段的比例关系推出$AB$、$AP$与$CP$的数量关系。

根据角平分线、中点、平行三个条件可以列出方程组

\[ \begin{cases}
  \sfrac{AB}{BD} = \sfrac{CP + AP}{CQ + DQ} \\
  BD + DQ = CQ \\
  \sfrac{CP}{AP} = \sfrac{CQ}{DQ} \\
\end{cases} \]

可得

\[ \frac{AB}{CQ - DQ} = \frac{CP + AP}{CQ + DQ} \]

即

\[ \frac{AB}{CP + AP} = \frac{CQ - DQ}{CQ + DQ} \]

将等号右边上下同时除以$DQ$可得

\[ \frac{AB}{CP + AP} = \frac{\sfrac{CQ}{DQ} - 1}{\sfrac{CQ}{DQ} + 1} \]

又根据$\sfrac{CP}{AP} = \sfrac{CQ}{DQ}$可得

\[ \frac{AB}{CP + AP} = \frac{\sfrac{CP}{AP} - 1}{\sfrac{CP}{AP} + 1} \]

将等号右边上下同时乘以$AP$可得

\begin{align*}
  \frac{AB}{CP + AP} &= \frac{CP - AP}{CP + AP} \\
  AB &= CP - AP \\
  AB + AP &= CP \\
\end{align*}

综上,$AB$、$AP$与$CP$的数量关系为$AB + AP = CP$。

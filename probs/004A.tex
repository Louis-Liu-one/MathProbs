
\prob{004A}{线段平移最值}

\begin{figure}
  \centering
  \image{004A}
  \caption{004A:线段平移最值} \label{fig:004A}
\end{figure}

如图~\ref{fig:004A},在平面直角坐标系$xOy$中有$A(0, 1)$、$B(0, 3)$。$x$轴上有长度为$3$的线段$CD$,求$AC + BD$的最小值。
\problabels{yellow/解析几何, green/最值问题}

\ans{$AC + BD$的最小值为$5$。}

\subsection{移动线段} \label{subsec:004A-mv}

\begin{figure}
  \centering
  \image{004A-mv}
  \caption{\nameref{subsec:004A-mv}:移动线段$AC$。} \label{fig:004A-mv}
\end{figure}

基本思路:通过移动线段$AC$使得$C$与$D$重合,进而将其转换为易求问题。

如图~\ref{fig:004A-mv},将线段$AC$先向右平移$3$个单位,再作其关于$x$轴的对称线段。易知$A(0, 1)$的对应点为$A'(3, -1)$,$C$的对应点与$D$重合。由平移的性质知$AC = A'D$,故求出$A'D + BD$的最小值即可。

又由于$B$、$A'$分别是$x$轴上下两定点,$D$为$x$轴上动点,故其最小值为

\begin{align*}
  A'B &= \sqrt{(x_{A'} - x_B)^2 + (y_{A'} - y_B)^2} \\
  &= \sqrt{3^2 + 4^2} = 5 \\
\end{align*}

综上,$AC + BD$的最小值为$5$。

\subsection{求导}

基本思路:通过将$AC + BD$表示为一个函数,然后通过求导求该函数的极值。

设$C(x, 0)$,则$D(x + 3, 0)$,又由$A(0, 1)$、$B(0, 3)$知

\begin{align*}
  AC &= \sqrt{(x_A - x_C)^2 + (y_A - y_C)^2} \\
  &= \sqrt{x^2 + 1} \\
  BD &= \sqrt{(x_B - x_D)^2 + (y_B - y_D)^2} \\
  &= \sqrt{(x + 3)^2 + 9} \\
  AC + BD &= \sqrt{x^2 + 1} + \sqrt{(x + 3)^2 + 9} \\
\end{align*}

又对其求导并应用链式法则得

\begin{align*}
  &\phantom{=} \frac{\dif}{\dif x}(AC + BD) \\
  &= \frac{\dif}{\dif x}(\sqrt{x^2 + 1} + \sqrt{(x + 3)^2 + 9}) \\
  &= \frac{\dif}{\dif x}\sqrt{x^2 + 1} + \frac{\dif}{\dif x}\sqrt{(x + 3)^2 + 9} \\
  &= \frac{\dif}{\dif\ (x^2 + 1)}\sqrt{x^2 + 1}\cdot \frac{\dif}{\dif x}(x^2 + 1) \\
  &+ \frac{\dif}{\dif\ \left((x + 3)^2 + 9\right)}\sqrt{(x + 3)^2 + 9} \\
  &\cdot \frac{\dif}{\dif x}\left((x + 3)^2 + 9\right) \\
  &= \frac1{2\sqrt{x^2 + 1}}\cdot2x + \frac1{2\sqrt{(x + 3)^2 + 9}}\cdot(2x + 6) \\
  &= \frac x{\sqrt{x^2 + 1}} + \frac{x + 3}{\sqrt{(x + 3)^2 + 9}} \\
\end{align*}

由于其导数为$0$时函数取极值,故当

\[ \frac x{\sqrt{x^2 + 1}} + \frac{x + 3}{\sqrt{(x + 3)^2 + 9}} = 0 \]

时$AC + BD$取极值,即

\begin{align*}
  -\frac x{\sqrt{x^2 + 1}} &= \frac{x + 3}{\sqrt{(x + 3)^2 + 9}} \\
  \frac{x^2}{x^2 + 1} &= \frac{(x + 3)^2}{(x + 3)^2 + 9} \\
  \frac{x^2 + 1}{x^2} &= \frac{(x + 3)^2 + 9}{(x + 3)^2} \\
  \frac1{x^2} &= \frac9{(x + 3)^2} \\
  9x^2 &= (x + 3)^2 \\
  8x^2 - 6x - 9 &= 0 \\
\end{align*}

解关于$x$的一元二次方程得$x = \sfrac32$或$x = -\sfrac34$。代入原函数得$AC + BD = 2\sqrt{13}$或$AC + BD = 5$。由于$2\sqrt{13} > 5$,故$x = \sfrac32$为平方后的余根,故$AC + BD$的最小值为$5$。

综上,$AC + BD$的最小值为$5$。


\prob{000E}{四边形求角I}

\begin{figure}[htbp]
  \centering
  \image{000E}
  \caption{总第~\ref{sec:000E} 题图} \label{fig:000E}
\end{figure}

如图~\ref{fig:000E},在四边形$ABCD$中,$2\angle CAD + \angle BAC = 180^\circ$,$BD = CD$,求$\angle CAD$与$\angle CBD$的数量关系。
\problabels{yellow/平面几何, green/数量关系问题}

\ans{$\angle CAD = \angle CBD$}

\subsection{翻折法} \label{subsec:000E-fold}

\begin{figure}[htbp]
  \centering
  \image{000E-fold}
  \caption{\nameref{subsec:000E-fold}:通过翻折$\triangle ABD$到$\triangle AB'D$,构造一个等腰直角三角形,进而得出数量关系。}
  \label{fig:000E-fold}
\end{figure}

基本思路:通过翻折$DB$构造等腰三角形,进而得出数量关系。

如图~\ref{fig:000E-fold},延长$CA$到$B'$,使得$AB = AB'$;连接$B'D$。

\begin{align*}
  &\because   2\angle CAD + \angle BAC = 180^\circ \\
  &\therefore 180^\circ - \angle CAD = \angle CAD + \angle BAC \\
  &\because   \angle CAD + \angle B'AD = 180^\circ \\
  &\therefore \angle BAD = \angle B'AD \\
  &\therefore \triangle ABD \cong \triangle AB'D \ \text{(证明省略)} \\
  &\therefore BD = B'D, \angle B' = \angle ABD \\
  &\because   BD = CD \\
  &\therefore B'D = CD, \angle BCD = \angle CBD \\
  &\therefore \angle B'CD = \angle B' \\
  &\therefore \angle ABD = \angle B'CD, \\
  &\mathalignsep \angle ACB = \angle BCD - \angle B'CD = \angle CBD - \angle ABD \\
  &\therefore \angle ABD = 45^\circ \\
  &\because   \angle ABD + \angle CBD + \angle ACB + \angle BAC = 180^\circ \\
  &\therefore \angle ABD + \angle CBD \\& + \angle CBD - \angle ABD + \angle BAC = 180^\circ \\
  &\therefore 2\angle CBD + \angle BAC = 180^\circ \\
  &\because   2\angle CAD + \angle BAC = 180^\circ \\
  &\therefore \angle CAD = \angle CBD \\
\end{align*}

综上,$\angle CAD = \angle CBD$。\footnote{得到$\angle ABD = \angle B'CD$后,也可用三角形相似证明。设$AC$与$BD$交于$P$,则$\triangle ABP \sim \triangle DCP$,然后证明$\triangle ADP \sim \triangle BCP$也可得到$\angle CAD = \angle CBD$。}

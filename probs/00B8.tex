
\prob{00B8}{五等边}

\begin{figure}[htbp]
  \centering \image{00B8}
  \caption{总第~\ref{sec:00B8} 题图}
  \label{fig:00B8}
\end{figure}

如图~\ref{fig:00B8},有等边$\triangle ABC$,其外有等边$\triangle AA_1A_2$、等边$\triangle BB_1B_2$、等边$\triangle CC_1C_2$;连接$A_2B_1, B_2C_1, C_2A_1$,作此三条线段的中点$M_1, M_2, M_3$。求证:$\triangle M_1M_2M_3$是等边三角形。
\problabels{yellow/平面几何, green/证明题}

\subsection{中位线与全等} \label{subsec:00B8-meq}

\begin{figure}[htbp]
  \centering \image{00B8-meq}
  \caption{方法~\ref{subsec:00B8-meq} 图}
  \label{fig:00B8-meq}
\end{figure}

如图~\ref{fig:00B8-meq},连接$AC_1, BC_2, AB_1, A_1B_1, B_1C_2$;作线段$AB_1, A_1B_1$的中点$M'_1, M'_2$,并连接$M_1M'_1, M_2M'_1, M_1M'_2, M_3M'_2$;延长$AM_2$到$B'_1$使得$AM_2 = B'_1M_2$,连接$AB'_1, B_1B'_1, B_2B'_1$。

显然$\triangle ACC_1 \cong \triangle BCC_2$,且两者有旋转$60^\circ$的关系,故$AC_1 = BC_2$且两者夹角为$60^\circ$。

显然$\triangle AM_2C_1 \cong \triangle B'_1M_2B_2$,故$AC_1 \parallel B_2B'_1$。

于是$BC_2 = B_2B'_1$且两者夹角为$60^\circ$,$B_1B = B_1B_2$且两者夹角为$60^\circ$。由此显然有
\[ \angle B_1BC_2 = \angle B_1B_2B'_1 \Rightarrow \triangle B_1BC_2 \cong \triangle B_1B_2B'_1 \]
故$B_1B'_1 = B_1C_2$且两者夹角为$60^\circ$。

显然$M'_1M_2$是$\triangle B_1AB'_1$的中位线,$M'_2M_3$是$\triangle B_1A_1C_2$的中位线,而$B_1B'_1 = B_1C_2$且两者夹角为$60^\circ$,故$M'_1M_2 = M'_2M_3$且两者夹角为$60^\circ$。

显然$M'_1M_1$是$\triangle AB_1A_2$的中位线,$M'_2M_1$是$\triangle A_1B_1A_2$的中位线,而$A_2A = A_2A_1$且两者夹角为$60^\circ$,故$M'_1M_1 = M'_2M_1$且两者夹角为$60^\circ$。

因此易得
\begin{align*}
  & \left\{ \begin{aligned}
    M_1M'_1 &= M_1M'_2 \\
    \angle M_1M'_1M_2 &= \angle M_1M'_2M_3 \\
    M'_1M_2 &= M'_2M_3
  \end{aligned} \right. \\
  \Rightarrow{}& \triangle M_1M'_1M_2 \cong \triangle M_1M'_2M_3
\end{align*}
且两者有旋转$60^\circ$的关系。故$M_1M_2 = M_1M_3$且$\angle M_2M_1M_3 = 60^\circ$,即$\triangle M_1M_2M_3$是等边三角形。证毕。

\subsection{解析几何} \label{subsec:00B8-dec}

\begin{lemma} \label{lemma:00B8-decl}
  在平面直角坐标系中,将$P(a + 2x, b + 2y)$绕$O(a, b)$逆时针旋转$60^\circ$后得到的点是$P'(a + x - \sqrt3y, b + \sqrt3x + y)$。
\end{lemma}

\begin{proof}
  将$O$平移至$(0, 0)$,则$P$平移至$(2x, 2y)$。

  设$2x = k\cos\theta, 2y = k\sin\theta$,则
  \[ P'(k\cos(\theta + 60^\circ), k\sin(\theta + 60^\circ)) \]
  即
  \[ P'\left(\frac12k\cos\theta - \frac{\sqrt3}2k\sin\theta, \frac{\sqrt3}2k\cos\theta + \frac12k\sin\theta\right) \]
  代入$x, y$立得$P'(x - \sqrt3y, \sqrt3x + y)$,平移回原位置后得$P'(a + x - \sqrt3y, b + \sqrt3x + y)$。
\end{proof}

\begin{figure}[htbp]
  \centering \image{00B8-dec}
  \caption{方法~\ref{subsec:00B8-dec} 图}
  \label{fig:00B8-dec}
\end{figure}

如图~\ref{fig:00B8-dec},以$AB$中点为中心,直线$AB$为$x$轴建立平面直角坐标系。不妨设
\begin{align*}
  & A(-1, 0), B(1, 0), C(0, \sqrt3), \\
  & A_1(2x_1 - 1, 2y_1), \\
  & B_1(2x_2 + 1, 2y_2), \\
  & C_1(2x_3, 2y_3 + \sqrt3)
\end{align*}
由引理~\ref{lemma:00B8-decl} 立得
\begin{align*}
  A_2&(x_1 - \sqrt3y_1 - 1, \sqrt3x_1 + y_1) \\
  B_2&(x_2 - \sqrt3y_2 + 1, \sqrt3x_2 + y_2) \\
  C_2&(x_3 - \sqrt3y_3, \sqrt3x_3 + y_3 + \sqrt3) \\
\end{align*}
令$M_k$的坐标为$(X_k/2, Y_k/2)$,则
\begin{align*}
  X_1 &= x_1 - \sqrt3y_1 + 2x_2, Y_1 = \sqrt3x_1 + y_1 + 2y_2, \\
  X_2 &= x_2 - \sqrt3y_2 + 2x_3 + 1, \\
  Y_2 &= \sqrt3x_2 + y_2 + 2y_3 + \sqrt3, \\
  X_3 &= x_3 - \sqrt3y_3 + 2x_1 - 1, \\
  Y_3 &= \sqrt3x_3 + y_3 + 2y_1 + \sqrt3
\end{align*}
于是有
\begin{align*}
  & M_1M_2 = M_1M_3 \\
  \Leftrightarrow{}& (X_1 - X_2)^2 + (Y_1 - Y_2)^2 \\
  &= (X_1 - X_3)^2 + (Y_1 - Y_3)^2 \\
  \Leftrightarrow{}& X_2^2 - X_3^2 + Y_2^2 - Y_3^2 \\
  &= 2X_1X_2 - 2X_1X_3 + 2Y_1Y_2 - 2Y_1Y_3 \\
  \Leftrightarrow{}& (X_2 + X_3)(X_2 - X_3) + (Y_2 + Y_3)(Y_2 - Y_3) \\
  &= 2X_1(X_2 - X_3) + 2Y_1(Y_2 - Y_3) \\
  \Leftrightarrow{}& (X_2 + X_3 - 2X_1)(X_2 - X_3) \\
  &+ (Y_2 + Y_3 - 2Y_1)(Y_2 - Y_3) = 0
\end{align*}
令
\begin{align*}
  P_1 &= X_2 + X_3 - 2X_1, & P_2 &= X_2 - X_3 \\
  Q_1 &= Y_2 + Y_3 - 2Y_1, & Q_2 &= Y_2 - Y_3
\end{align*}
即要证明$P_1P_2 + Q_1Q_2 = 0$。

计算可知
\begin{align*}
  P_1 ={}& -3x_2 + 3x_3 - \sqrt3y_2 - \sqrt3y_3 + 2\sqrt3y_1 \\
  P_2 ={}& x_2 + x_3 - 2x_1 - \sqrt3y_2 + \sqrt3y_3 + 2 \\
  Q_1 ={}& \sqrt3x_2 + \sqrt3x_3 \\
    &- 2\sqrt3x_1 - 3y_2 + 3y_3 + 2\sqrt3 \\
  Q_2 ={}& \sqrt3x_2 - \sqrt3x_3 + y_2 + y_3 - 2y_1
\end{align*}
注意到$Q_1 = \sqrt3P_2, P_1 = -\sqrt3Q_2$,故显然$P_1P_2 + Q_1Q_2 = 0 \Rightarrow M_1M_2 = M_1M_3$。

再以$BC$中点为中心,直线$BC$为$x$轴建立平面直角坐标系。同理可证$M_1M_2 = M_2M_3$,故$M_1M_2 = M_2M_3 = M_3M_1$,于是$\triangle M_1M_2M_3$是等边三角形。证毕。


\prob{00B8}{五等边}

\begin{figure}[htbp]
  \centering \image{00B8}
  \caption{总第~\ref{sec:00B8} 题图}
  \label{fig:00B8}
\end{figure}

如图~\ref{fig:00B8},有等边$\triangle ABC$,其外有等边$\triangle AA_1A_2$、等边$\triangle BB_1B_2$、等边$\triangle CC_1C_2$;连接$A_2B_1, B_2C_1, C_2A_1$,作此三条线段的中点$M_1, M_2, M_3$。求证:$\triangle M_1M_2M_3$是等边三角形。
\problabels{yellow/平面几何, green/证明题}

\subsection{中位线与全等} \label{subsec:00B8-meq}

\begin{figure}[htbp]
  \centering \image{00B8-meq}
  \caption{方法~\ref{subsec:00B8-meq} 图}
  \label{fig:00B8-meq}
\end{figure}

如图~\ref{fig:00B8-meq},连接$AC_1, BC_2, AB_1, A_1B_1, B_1C_2$;作线段$AB_1, A_1B_1$的中点$M'_1, M'_2$,并连接$M_1M'_1, M_2M'_1, M_1M'_2, M_3M'_2$;延长$AM_2$到$B'_1$使得$AM_2 = B'_1M_2$,连接$AB'_1, B_1B'_1, B_2B'_1$。

显然$\triangle ACC_1 \cong \triangle BCC_2$,且两者有旋转$60^\circ$的关系,故$AC_1 = BC_2$且两者夹角为$60^\circ$。

显然$\triangle AM_2C_1 \cong \triangle B'_1M_2B_2$,故$AC_1 \parallel B_2B'_1$。

于是$BC_2 = B_2B'_1$且两者夹角为$60^\circ$,$B_1B = B_1B_2$且两者夹角为$60^\circ$。由此显然有
\[ \angle B_1BC_2 = \angle B_1B_2B'_1 \Rightarrow \triangle B_1BC_2 \cong \triangle B_1B_2B'_1 \]
故$B_1B'_1 = B_1C_2$且两者夹角为$60^\circ$。

显然$M'_1M_2$是$\triangle B_1AB'_1$的中位线,$M'_2M_3$是$\triangle B_1A_1C_2$的中位线,而$B_1B'_1 = B_1C_2$且两者夹角为$60^\circ$,故$M'_1M_2 = M'_2M_3$且两者夹角为$60^\circ$。

显然$M'_1M_1$是$\triangle AB_1A_2$的中位线,$M'_2M_1$是$\triangle A_1B_1A_2$的中位线,而$A_2A = A_2A_1$且两者夹角为$60^\circ$,故$M'_1M_1 = M'_2M_1$且两者夹角为$60^\circ$。

因此易得
\begin{align*}
  & \left\{ \begin{aligned}
    M_1M'_1 &= M_1M'_2 \\
    \angle M_1M'_1M_2 &= \angle M_1M'_2M_3 \\
    M'_1M_2 &= M'_2M_3
  \end{aligned} \right. \\
  \Rightarrow{}& \triangle M_1M'_1M_2 \cong \triangle M_1M'_2M_3
\end{align*}
且两者有旋转$60^\circ$的关系。故$M_1M_2 = M_1M_3$且$\angle M_2M_1M_3 = 60^\circ$,即$\triangle M_1M_2M_3$是等边三角形。证毕。

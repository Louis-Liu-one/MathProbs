
\prob{0090}{整数根方程IV}

对于某两非零且互不为相反数的整数$a, b$,关于$n$的方程
\begin{align}
  \frac1a + \frac1b = \frac n{a + b} \label{eq:0090}
\end{align}
有整数解,求所有的整数解$n$。
\problabels{yellow/数论, green/方程相关问题}

\ans{$n = 4$}

\subsection{解二次方程}

式~\ref{eq:0090} 化简为
\[ a^2 + (2 - n)ab + b^2 = 0 \]
对$a$应用二次方程求根公式得
\begin{align*}
  a &= \frac{-(2 - n)b \pm \sqrt{(2 - n)^2b^2 - 4b^2}}2 \\
  &= \frac b2\left(n - 2 \pm \sqrt{(n - 2)^2 - 4}\right)
\end{align*}
注意到$a, b, n$为整数,于是$(n - 2)^2 - 4$是平方数。设其为$k^2$,则
\begin{align*}
  (n - 2)^2 - 4 &= k^2 \\
  (n - 2 + k)(n - 2 - k) &= 4 \\
\end{align*}
而两乘数皆整数,故易知有$n - 2 = \pm2$即$n = 0$或$n = 4$。当$n = 0$时,式~\ref{eq:0090} 左边不为0,右边为0,矛盾,故$n = 4$。经检验,$n = 4$在$a = b$时成立。

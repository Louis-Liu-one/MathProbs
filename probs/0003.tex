
\prob{0003}{双等边}

\begin{figure}[htbp]
  \centering
  \image{0003}
  \caption{0003:双等边} \label{fig:0003}
\end{figure}

如图~\ref{fig:0003},$\triangle ABC$与$\triangle CDE$是等边三角形,$\angle CBD = 60^\circ$,求证:$PA = PE$。
\problabels{yellow/平面几何, green/证明题}

\subsection{证明等腰三角形} \label{subsec:0003-istri}

\begin{figure}[htbp]
  \centering
  \image{0003-istri}
  \caption{\nameref{subsec:0003-istri}:将$\triangle ABP$翻着至$\triangle A'BP$,从而将问题转化为等腰三角形。}
  \label{fig:0003-istri}
\end{figure}

基本思路:将$PA$移到另一个位置,使得其能成为等腰三角形的一条腰,其中$PE$是等腰三角形的另一条腰。证明其为等腰三角形即可。

如图~\ref{fig:0003-istri},延长$BD$至$A'$,使得$BA' = BC$。连接$A'P$、$A'C$、$A'E$。

\begin{align*}
  &\because   BA' = BC, \angle A'BC = 60^\circ \\
  &\therefore \triangle A'BC \text{是等边三角形} \\
  &\therefore BC = BA' = CA' \\
  &\because   AB = AC = BC \\
  &\therefore AB = A'B \\
  &\therefore \triangle ABP \cong \triangle A'BP \ \text{(证明省略)} \\
  &\therefore \angle APB = \angle A'PB, PA = PA' \\
  \text{又}&\because \angle A'CB = \angle DCE \\
  &\therefore \angle BCD = \angle A'CE \\
  &\therefore \triangle BCD \cong \triangle A'CE \ \text{(证明省略)} \\
  &\therefore \angle CBD = \angle CA'E = 60^\circ \\
  &\because   \angle A'CB = 60^\circ \\
  &\therefore \angle A'CB = \angle CA'E \\
  &\therefore BC \parallel A'E \\
  &\therefore \angle APB = \angle AEA', \angle A'PB = \angle PA'E \\
  &\because   \angle APB = \angle A'PB \\
  &\therefore \angle AEA' = \angle PA'E \\
  &\therefore PA' = PE \\
  &\because   PA = PA' \\
  &\therefore PA = PE \\
  &\text{证毕。} \\
\end{align*}


\prob{0093}{从线段相等到角}

\begin{figure}[htbp]
  \centering \image{0093-1}
  \caption{总第~\ref{sec:0093} 题图I} \label{fig:0093-1}
\end{figure}

\begin{figure}[htbp]
  \centering \image{0093-2}
  \caption{总第~\ref{sec:0093} 题图II} \label{fig:0093-2}
\end{figure}

在$\triangle ABC$中,$D, E$分别在边$AC, AB$上,$BD, CE$分别平分$\angle ABC, \angle ACB$且交于$O$。若$OD = OE$,求证:$AB = AC$(图~\ref{fig:0093-1})或$\angle A = 60^\circ$(图~\ref{fig:0093-2})。
\problabels{yellow/平面几何, green/证明题}

\subsection{分情况讨论} \label{subsec:0093-sit}

连接$OA$,可知$OA$平分$\angle BAC$。考虑$AD = AE$与$AD \ne AE$两种情况。

\begin{figure}[htbp]
  \centering \image{0093-sit1}
  \caption{方法~\ref{subsec:0093-sit} 图I} \label{fig:0093-sit1}
\end{figure}

如图~\ref{fig:0093-sit1},此时$AD = AE$,由此易知$\triangle AOD \cong \triangle AOE$于是可知$\triangle ABD \cong \triangle ACE \Rightarrow AB = AC$。

\begin{figure}[htbp]
  \centering \image{0093-sit2}
  \caption{方法~\ref{subsec:0093-sit} 图II} \label{fig:0093-sit2}
\end{figure}

考虑$AD \ne AE$的情况。如图~\ref{fig:0093-sit2},此时$AD \ne AE$。在线段$AB$上截取$AD' = AD$,连接$OD'$。易证$\triangle AOD \cong \triangle AOD'$,于是$OD = OD' = OE$,可知$\angle ODA + \angle OEA = 180^\circ$。

\begin{align*}
  \because  {}& \angle ODA + \angle OEA = 180^\circ \\
  \therefore{}& \angle ABD + \angle ACE + \angle BOE + \angle COD = 180^\circ \\
  \because  {}& \angle BOE = \angle COD = \angle OBC + \angle OCB \\
  \therefore{}& \angle ABD + \angle ACE + 2(\angle OBC + \angle OCB) = 180^\circ \\
  \because  {}& \angle ABD = \angle OBC, \angle ACE = \angle OCB \\
  \therefore{}& 3(\angle OBC + \angle OCB) = 180^\circ \\
  \therefore{}& \angle ABC + \angle ACB = 2(\angle OBC + \angle OCB) = 120^\circ \\
  \therefore{}& \angle BAC = 60^\circ
\end{align*}

因此,当$AD = AE$时$AB = AC$,当$AD \ne AE$时$\angle BAC = 60^\circ$,证毕。

\subsection{正弦定理} \label{subsec:0093-sin}

\begin{figure}[htbp]
  \centering \image{0093-sin}
  \caption{方法~\ref{subsec:0093-sin} 图} \label{fig:0093-sin}
\end{figure}

如图~\ref{fig:0093-sin},连接$OA$。设$\angle BAC = 2\alpha, \angle ABC = 2\beta, \angle ACB = 2\gamma$,可知
\begin{align*}
  \angle OAB = \angle OAC &= \alpha \\
  \angle ODA &= 2\gamma + \beta \\
  \angle OEA &= 2\beta + \gamma
\end{align*}

在$\triangle AOD$与$\triangle AOE$中应用正弦定理,可知
\begin{align*}
  \frac{OA}{\sin(2\gamma + \beta)} &= \frac{OD}{\sin\alpha} \\
  \frac{OA}{\sin(2\beta + \gamma)} &= \frac{OE}{\sin\alpha}
\end{align*}
而$OD = OE$,于是有
\[ \sin(2\beta + \gamma) = \sin(2\gamma + \beta) \]
显然,有
\[ 2\beta + \gamma = 2\gamma + \beta\ \text{或}\ (2\beta + \gamma) + (2\gamma + \beta) = 180^\circ \]
于是有$\angle BAC = 60^\circ$或$\angle ABC = \angle ACB \Rightarrow AB = AC$。证毕。

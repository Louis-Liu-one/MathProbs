
\prob{002D}{高和中线}

\begin{figure}[htbp]
  \centering
  \image{002D}
  \caption{002D:高和中线} \label{fig:002D}
\end{figure}

如图~\ref{fig:002D},$OA \perp OA'$,$OB \perp OB'$,$OA = OA'$,$OB = OB'$,$OP \perp AB$于$P$,$Q$是线段$A'B'$的中点,求证:$P$、$O$、$Q$三点共线。
\problabels{yellow/平面几何, green/证明题}

\subsection{倍长中线} \label{subsec:002D-mid}

\begin{figure}[htbp]
  \centering
  \image{002D-mid}
  \caption{\nameref{subsec:002D-mid}:通过倍长中线构造两对全等三角形,从而证明。}
  \label{fig:002D-mid}
\end{figure}

基本思路:通过倍长中线构造全等三角形,从而证明$\angle POQ = 180^\circ$,进而证明命题。

如图~\ref{fig:002D-mid},延长$OQ$到点$O'$,使得$OQ = O'Q$;连接$O'B'$。

\begin{align*}
  &\because   \triangle OQA' \cong \triangle O'QB' \ \text{(证明省略)} \\
  &\therefore \angle O'B'Q = \angle OA'B', A'O = O'B' \\
  &\therefore \angle O'B'Q + \angle OB'A' = \angle OA'B' + \angle OB'A' \\
  &\therefore \angle O'B'O = \angle OA'B' + \angle OB'A' \\
  &\because   \angle OA'B' + \angle OB'A' + \angle A'OB' = 180^\circ \\
  &\therefore \angle O'B'O + \angle A'OB' = 180^\circ \\
  &\because   \angle AOB + \angle A'OB' \\
  &\mathalignsep + \angle AOA' + \angle BOB' = 360^\circ \\
  &\therefore \angle AOB + \angle A'OB' \\
  &\mathalignsep = 360^\circ - \angle AOA' - \angle BOB' \\
  &\because   OA \perp OA', OB \perp OB' \\
  &\therefore \angle AOA' = \angle BOB' = 90^\circ \\
  &\therefore \angle AOB + \angle A'OB' = 180^\circ \\
  &\therefore \angle AOB = \angle O'B'O \\
  &\because   AO = A'O \\
  &\therefore AO = O'B' \\
  &\because   \text{在}\triangle AOB\text{与}\triangle O'B'O\text{中} \\
  &\mathalignsep \begin{cases}
    AO = O'B' \\
    \angle AOB = \angle O'B'O \\
    OB = B'O \\
  \end{cases} \\
  &\therefore \triangle AOB \cong \triangle O'B'O \\
  &\therefore \angle B'OQ = \angle B \\
  \text{又}&\because OP \perp AB \\
  &\therefore \angle OPA = 90^\circ \\
  &\therefore \angle B + \angle BOP = 90^\circ \\
  &\therefore \angle B'OQ + \angle BOP = 90^\circ \\
  &\because   \angle POQ = \angle B'OQ + \angle BOP + \angle BOB' \\
  &\therefore \angle POQ = 90^\circ + 90^\circ = 180^\circ \\
  &\therefore P, O, Q\text{三点共线} \\
\end{align*}

证毕。

\subsection{旋转} \label{subsec:002D-rot}

\begin{figure}[htbp]
  \centering
  \image{002D-rot}
  \caption{\nameref{subsec:002D-rot}:通过旋转三角形构造中位线,然后利用位置关系证明。}
  \label{fig:002D-rot}
\end{figure}

基本思路:通过构造一个被旋转的三角形构造一个三角形的中位线,然后利用一系列线段的位置关系证明命题。

如图~\ref{fig:002D-rot},延长$AO$到点$A''$使得$OA = OA' = OA''$,连接$A''B$;设线段$A''B$的中点为$Q'$,连接$OQ'$。

\begin{align*}
  &\because   \angle A'OA'' = \angle BOB' = 90^\circ \\
  &\therefore \angle A'OA'' + \angle A''OB' = \angle BOB' + \angle A''OB' \\
  &\therefore \angle A'OB' = \angle A''OB \\
  &\therefore \triangle A'OB' \cong \triangle A''OB \ \text{(证明省略)} \\
  &\therefore A'B' = A''B, \angle B' = \angle OBQ' \\
  &\because   B'Q = \frac12A'B', BQ' = \frac12A''B \\
  &\therefore B'Q = BQ' \\
  &\because   \text{在}\triangle B'OQ\text{与}\triangle BOQ'\text{中} \\
  &\mathalignsep \begin{cases}
    B'O = BO \\
    \angle B' = \angle OBQ' \\
    B'Q = BQ' \\
  \end{cases} \\
  &\therefore \triangle B'OQ \cong \triangle BOQ' \\
  &\therefore \angle B'OQ = \angle BOQ' \\
  &\therefore \angle B'OQ + \angle B'OQ' = \angle BOQ' + \angle B'OQ' \\
  &\therefore \angle QOQ' = \angle BOB' = 90^\circ \\
  &\therefore OQ' \perp OQ \\
  \text{又}&\because OA'' = OA, Q'A'' = Q'B \\
  &\therefore \text{线段}OQ'\text{是}\triangle AA''B\text{的中位线} \\
  &\therefore OQ' \parallel AB \\
  &\because   OP \perp AB \\
  &\therefore OQ' \perp OP \\
  &\therefore P, O, Q\text{三点共线} \\
\end{align*}

证毕。


\prob{00B7}{从增函数到不等式链}

求证对于任意非负实数$a, b$,函数
\[ f(x) = \left(\frac{a^x + b^x}2\right)^{1/x} \]
在实数上是单调不减函数,并由此证明对于任意整数$i$,
\begin{align*}
  & \sqrt[i]{\frac2{1/a^i + 1/b^i}} \le \dots \le \frac2{1/a + 1/b} \\
  \le{}& \sqrt{ab} \le \frac{a + b}2 \le \sqrt{\frac{a^2 + b^2}2} \\
  \le{}& \sqrt[3]{\frac{a^3 + b^3}2} \le \dots \le \sqrt[i]{\frac{a^i + b^i}2}
\end{align*}
\problabels{yellow/代数, green/证明题}

\subsection{由定义入手} \label{subsec:00B7-def}

由定义知要证明对于任意满足$x_1 < x_2$的实数$x_1, x_2$,有
\[ \left(\frac{a^{x_1} + b^{x_1}}2\right)^{1/x_1} \le \left(\frac{a^{x_2} + b^{x_2}}2\right)^{1/x_2} \]

构造函数
\[ g(x) = x^k, (k < 0\ \text{或}\ k > 1) \]
显然有
\[ \frac{\dif^2 g}{\dif x^2} = k(k - 1)x^{k - 2} > 0 \]
故$g$为凹函数,于是对于任意实数$A, B$,有
\[ g\left(\frac{A + B}2\right) \le \frac{g(A) + g(B)}2 \]
即
\[ \left(\frac{A + B}2\right)^k \le \frac{A^k + B^k}2 \]
由于$x_2 > x_1 \ne 0$,故$x_2/x_1$不在$[0, 1]$中,故可令$A = a^{x_1}, B = b^{x_1}, k = x_2/x_1$,则
\[ \left(\frac{a^{x_1} + b^{x_1}}2\right)^{x_2/x_1} \le \frac{a^{x_2} + b^{x_2}}2 \]
即
\[ \left(\frac{a^{x_1} + b^{x_1}}2\right)^{1/x_1} \le \left(\frac{a^{x_2} + b^{x_2}}2\right)^{1/x_2} \]
故$f$是单调不减函数。

显然
\begin{align*}
  \ln f(x) &= \ln\left(\frac{a^x + b^x}2\right)^{1/x} \\
  &= \frac1x\ln\frac{a^x + b^x}2 = \frac{\ln((a^x + b^x)/2)}x
\end{align*}
当$x \to 0$时,分式上下皆$\to 0$,故应用L'Hôspital法则知
\begin{align*}
  \lim_{x \to 0}\ln f(x) &= \lim_{x \to 0}\frac2{a^x + b^x}\cdot\frac{a^x\ln a + b^x\ln b}2 \\
  &= \lim_{x \to 0}\frac{a^x\ln a + b^x\ln b}{a^x + b^x} \\
  &= \frac{\ln a + \ln b}2 \\
  \lim_{x \to 0}f(x) &= \mathe^{(\ln a + \ln b)/2} = \sqrt{ab}
\end{align*}
由前知$f$是实数上的单调不减函数,故
\begin{align*}
  f(-i) &\le \dots \le f(-1) \le \lim_{x \to 0}f(x) \\
  &\le f(1) \le f(2) \le \dots \le f(i)
\end{align*}
即
\begin{align*}
  & \sqrt[i]{\frac2{1/a^i + 1/b^i}} \le \dots \le \frac2{1/a + 1/b} \\
  \le{}& \sqrt{ab} \le \frac{a + b}2 \le \sqrt{\frac{a^2 + b^2}2} \\
  \le{}& \sqrt[3]{\frac{a^3 + b^3}2} \le \dots \le \sqrt[i]{\frac{a^i + b^i}2}
\end{align*}
证毕。

\subsection{直接求导}

直接将$f$求导得
\begin{align*}
  \frac{\dif f}{\dif x} &= \frac\dif{\dif x}\mathe^{1/x\cdot\ln((a^x + b^x)/2)} \\
  &= \mathe^{1/x\cdot\ln((a^x + b^x)/2)}\cdot\frac\dif{\dif x}\left(\frac1x\ln\frac{a^x + b^x}2\right) \\
  &= f(x)\left(\frac1x\cdot\frac\dif{\dif x}\ln\frac{a^x + b^x}2 + \frac{a^x + b^x}2\cdot\frac\dif{\dif x}\frac1x\right) \\
  &= f(x)\left(\frac{a^x\ln a + b^x\ln b}{x\left(a^x + b^x\right)} - \frac1{x^2}\ln\frac{a^x + b^x}2\right) \\
  &= x^2f(x)\left(\frac{a^x\ln a + b^x\ln b}{a^x + b^x}x - \ln\frac{a^x + b^x}2\right)
\end{align*}
令$A = a^x, B = b^x$,则
\begin{align*}
  \frac{\dif f}{\dif x} ={}& x^2f(x)\left(\frac{A\ln A + B\ln B}{A + B} - \ln\frac{A + B}2\right) \\
  ={}& \frac{2x^2f(x)}{A + B}\bigg(\frac{A\ln A + B\ln B}2 \\
  &- \frac{A + B}2\ln\frac{A + B}2\bigg)
\end{align*}
令
\[ g(x) = x\ln x \]
则
\[ \frac{\dif f}{\dif x} = \frac{2x^2f(x)}{A + B}\left(\frac{g(A) + g(B)}2 - g\left(\frac{A + B}2\right)\right) \]
而
\[ \frac{\dif^2 g}{\dif x^2} = \frac\dif{\dif x}\left(\ln x + 1\right) = \frac1x \]
当其中$x > 0$时上式$> 0$,故此时$g$为凸函数,故
\[ \frac{g(A) + g(B)}2 \ge g\left(\frac{A + B}2\right) \]
同时显然
\[ \frac{2x^2f(x)}{A + B} \ge 0 \]
故
\[ \frac{\dif f}{\dif x} \ge 0 \]
因此$f$为单调不减函数。与方法~\ref{subsec:00B7-def} 同理得
\begin{align*}
  & \sqrt[i]{\frac2{1/a^i + 1/b^i}} \le \dots \le \frac2{1/a + 1/b} \\
  \le{}& \sqrt{ab} \le \frac{a + b}2 \le \sqrt{\frac{a^2 + b^2}2} \\
  \le{}& \sqrt[3]{\frac{a^3 + b^3}2} \le \dots \le \sqrt[i]{\frac{a^i + b^i}2}
\end{align*}
证毕。
